\documentclass[12pt]{article}
\RequirePackage[l2tabu, orthodox]{nag}
\usepackage[main=english]{babel}
\usepackage[rm={lining,tabular},sf={lining,tabular},tt={lining,tabular,monowidth}]{cfr-lm}
\usepackage{amsthm,amssymb,latexsym,gensymb,mathtools,mathrsfs}
\usepackage[T1]{fontenc}
\usepackage[utf8]{inputenc}
\usepackage[pdftex]{graphicx}
\usepackage{epstopdf,enumitem,microtype,dcolumn,booktabs,hyperref,url,fancyhdr}
\usepackage{algorithmic}
\usepackage[ruled,vlined,commentsnumbered,titlenotnumbered]{algorithm2e}
\usepackage{bbm}

% Plotting
\usepackage{pgfplots}
\usepackage{xinttools} % for the \xintFor***
\usepgfplotslibrary{fillbetween}
\pgfplotsset{compat=1.8}
\usepackage{tikz}

% Custom Commands
\newcommand*{\norm}[1]{\left\lVert#1\right\rVert}
\newcommand*{\abs}[1]{\left\lvert#1\right\rvert}
\newcommand*{\suchthat}{\,\mathrel{\big|}\,}
\newcommand{\E}{\mathbb{E}}
\newcommand{\Var}{\mathrm{Var}}
\newcommand{\R}{\mathbb{R}}
\newcommand{\N}{\mathcal{N}}
\newcommand{\Ker}{\mathrm{Ker}}
\newcommand{\Cov}{\mathrm{Cov}}
\newcommand{\Prob}{\mathbb{P}}
\DeclarePairedDelimiterX\innerp[2]{(}{)}{#1\delimsize\vert\mathopen{}#2}
\DeclareMathOperator*{\argmax}{argmax}
\DeclareMathOperator*{\argmin}{argmin}
\DeclarePairedDelimiter{\ceil}{\lceil}{\rceil}

\setlist{topsep=1ex,parsep=1ex,itemsep=0ex}
\setlist[1]{leftmargin=\parindent}
\setlist[enumerate,1]{label=\arabic*.,ref=\arabic*}
\setlist[enumerate,2]{label=(\alph*),ref=(\alph*)}

% For embedding images
\graphicspath{ {./images/} }

% Specifically for paper formatting 
\renewcommand{\baselinestretch}{1.2} % Spaces manuscript for easy reading

% Formatting definitions, propositions, etc. 
\newtheorem{definition}{Definition}
\newtheorem{prop}{Proposition}
\newtheorem{lemma}{Lemma}
\newtheorem{thm}{Theorem}
\newtheorem{corollary}{Corollary}

% Title and author
\title{Conditional Expectation and Martingales}
\author{Andrew Roberts}

\begin{document}

\maketitle
\tableofcontents
\newpage

\section{Defining Conditional Expectation}
Here we slowly build up to the general definition of conditional expectation in order to gain intuition. 

\subsection{A Simple Example}

\subsubsection{Constructing the Probability Space for a Two-Coin Flip Experiment}
We begin with the simple example of independently flipping two coins, with respective probabilities $p$ and $q$ of landing heads. Coding $0$ as tails and $1$ as heads, the probability space associated with 
this experiment is
\[\Omega = \{(0, 0), (0, 1), (1, 0), (1, 1)\}\]
The typical $\sigma$-algebra to consider here is the full power set 
\[\mathcal{F} = 2^{\Omega} = \{\text{All subsets of } \Omega\}\]
Writing out all of the subsets would be inconvenient, but note that there are $2^4 = 16$ sets in the collection $2^{\Omega}$. Now that we have the measurable space $(\Omega, \mathcal{F})$, we can 
define a probability measure $\mathbb{P}: \mathcal{F} \to [0, 1]$, which assigns weights to each set in the $\sigma$-algebra. The probability measure corresponding to the two-coin experiment is 
\begin{align*}
\Prob(\{(1, 1)\}) &= pq \\
\Prob(\{(0, 0)\}) &= (1 - p)(1 - q) \\
\Prob(\{(0, 1)\}) &= (1 - p)q \\
\Prob(\{(1, 0)\}) &= p(1 - q)
\end{align*}
This follows from the fact that any sequence of two independent flips has probability equal to the product of the probabilities of the individual flips. 
For example, the probability of flipping two heads in a row should be the product of each independently turning up heads. 
In this case, defining $\Prob(\{\omega\})$ for all singleton sets $\{\omega\} \in \mathcal{F}$ is sufficient to define $\Prob$ on all of $\mathcal{F}$, since the probabilities of disjoint events must add. 
For example, consider 
\[A = \{(0, 0), (0, 1)\} = \{\text{First flip is tails and second flip is heads or tails}\}\]
The probability of $A$ is then sum of the probabilities of the disjoint singleton sets $\{(0, 0)\}$ and 
$\{0, 1\}$. 
\[\Prob(A) = \Prob(\{\omega \in \Omega: \omega \in \{(0, 0), (0, 1)\}\}) = \Prob(\{(0, 0)\}) + \Prob(\{(0, 1)\}) = (1 - p)(1 - q) + (1 - p)q = 1 - p\] 
We have thus constructed a valid probability space $(\Omega, \mathcal{F}, \Prob)$. 

\subsubsection{A Smaller $\sigma$-algebra}
Although the power set $\sigma$-algebra is the natural choice here, for the purposes of illustration let's consider an alternative 
$\sigma$-algebra. Since $\sigma$-algebras are supposed to model \textit{knowledge} or \textit{information}, let's try to define the $\sigma$-algebra $\mathcal{G}$ that models the knowledge of the the total number 
of heads in the experiment. Intuitively, this is less information than is captured by $\mathcal{F}$. For example., the power set $\sigma$-algebra allowed us to make statements about the probability of sets $\{0, 1\}$ and 
$\{1, 0\}$. However, each of these sets have the same number of heads (one), so according to $\mathcal{G}$ we should not be able to distinguish between these two events. $\mathcal{G}$ should allow us to make statements 
about the probability of events like ``the experiment produced one head'' but it should not let us say anything about which coin produced that head. Therefore, the sets that have the same number of heads
\[\{(0, 0)\}, \{(0, 1), (1, 0)\}, \{(1, 1)\}\] 
should be in $\mathcal{G}$. The $\sigma$-algebra can now be completed by considering unions, intersections, and complements of these sets.
\begin{align*}
\mathcal{G} = &\{\{(0, 0)\}, \{(0, 1), (1, 0)\}, \{(1, 1)\}, \\
		       &\{(0, 0), (0, 1), (1, 0)\}, \{(1, 1), (0, 1), (1, 0)\}, \{(0, 0), (1, 1)\} \}
\end{align*}
If should make intuitive sense that these are the sets we end up with; e.g. $\{(0, 0), (0, 1), (1, 0)\}$ is the set corresponding to the event ``the experiment produced zero or one heads''. None of these sets will allow us to make 
probabilistic statements about the specific sequence in which the heads appeared in the coin flips. We can again define a probability measure $\Prob: \mathcal{G} \to [0, 1]$ (note that this is the restriction of the previous 
probability measure to the smaller domain $\mathcal{G}$) on the measurable space $(\Omega, \mathcal{G})$. 
\begin{align*}
\Prob(\{(1, 1)\}) &= pq \\
\Prob(\{(0, 0)\}) &= (1 - p)(1 - q) \\
\Prob(\{(0, 1), (1, 0)\}) &= p(1 - q) + (1 - p)q
\end{align*}
The probability $p(1 - q) + (1 - p)q$ is simply the probability of flipping a single head (either the first is heads and the second is tails, or vice versa). 
The main point about $\mathcal{G}$ is that the sets $\{(0, 1)\}$ and $\{1, 0\}$ are indistinguishable. Notice that the probability measure assigns a probability to $\{(0, 1), (1, 0)\}$ by averaging over the probabilities 
of $\{(0, 1)\}$ and $\{(0, 1)\}$, but it cannot assign probabilities to these latter two sets! 

\subsubsection{Smaller $\sigma$-algebra generated by a random variable}
In the previous section, we stated the goal of defining a $\sigma$-algebra $\mathcal{G}$ that captured knowledge only of the number of heads that resulted from the experiment. We can arrive at the same $\sigma$-algebra by 
considering the random variable $Y: \Omega \to \{0, 1, 2\}$ that returns the number of heads in the experiment; e.g. $Y((1, 1)) = 2$ and $Y(0, 1) = 1$. My claim is that $\mathcal{G}$ is the smallest $\sigma$-algebra to which $Y$
is measurable. Intuitively, this should be the case since being able to make probabilistic statements about the values $Y$ can assume only requires knowledge about how many heads occurred in the experiment. Let's show this more formally. $Y$ can assume three values, and hence we need to consider the inverse image of these three values. 
\begin{align*}
Y^{-1}(0) &= \{(0, 0)\} \\
Y^{-1}(1) &= \{(1, 0), (0, 1)\} \\
Y^{-1}(2) &= \{(1, 1)\}
\end{align*}
So the three sets on the righthand side are the three subsets of $\Omega$ to which we must assign probabilities in order to define a distribution on $Y$. This is the minimal collection of sets with this property, so the minimal 
$\sigma$-algebra follows from taking unions, intersections, and complements of these sets. But this is the same exact thing we did previously to arrive at $\mathcal{G}$! Thus, we have shown that 
\[\mathcal{G} = \sigma(Y)\]
or in words: $\mathcal{G}$ is the $\sigma$-algebra generated by $Y$. This just means that $\mathcal{G}$ is the smallest possible $\sigma$-algebra to which $Y$ is measurable. The smallest $\sigma$-algebra can also be 
thought of as the ``coarsest'' in that it contains less fine-grained information compared to, for example, $\mathcal{F}$. $Y$ is also measurable with respect to $\mathcal{F}$, but $\mathcal{F}$ contains more sets than are needed 
to define a distribution on $Y$. Note also that \textit{any} random variable $X: \mathcal{\Omega} \to \R$ that is measurable with respect to $\mathcal{G}$ must have the property $X((1, 0)) = X((0, 1))$. This is because $\mathcal{G}$
does not contain enough information to differentiate between the two outcomes $(1, 0)$ and $(0, 1)$. 

\subsubsection{Defining conditional expectation with respect to the smaller $\sigma$-algebra}
We are now ready to give a definition of conditional expectation in this simple setting. In this setting and the more general one, conditional expectation is defined as conditional with respect to a $\sigma$-algebra. This makes sense given the fact that $\sigma$-algebra models knowledge, so we seek to define an expectation with respect to knowledge of certain events. Let $X: \Omega \to \R$ be any random variable that is $\mathcal{F}$ measurable. 
The conditional expectation of $X$ given $\mathcal{G}$ will itself be a random variable, let's call it $Z$. Intuitively, if we condition on $\mathcal{G}$ then we are assuming knowledge of only the events in $\mathcal{G}$. Given 
these events, the conditional expectation $Z$ will give the ``best guess'' of $X$ for each event in $\mathcal{G}$. Thus, $Z$ must be $\mathcal{G}$ measurable but does not in general need to be $\mathcal{F}$ measurable. 
We can think of $Z(\omega)$ in the following way: 
\begin{itemize}
\item The random experiment produces an outcome $\omega \in \Omega$. 
\item We only have knowledge of which set in $\mathcal{G}$ that $\omega$ falls into. Suppose $\omega \in G$ for some $G \in \mathcal{G}$. 
\item $Z(\omega)$ averages over the $\mathcal{F}$-information in $G$ to give the best guess of $X$. 
\end{itemize}
To make this more clear, we first present an example. After I will give the definition of the conditional expectation of $X$ given $\mathcal{G}$ in this setting. For now, consider 
$X = \mathbbm{1}\{\text{Second flip is heads}\}$. So we want to condition the random variable representing knowledge that the second flip is heads on knowledge of the total number of heads.
If we know there are zero heads, then our best guess (and the correct one) at the value of $X$ is $0$. Similarly, if we know there are two heads then we can correctly infer that $X = 1$. However, 
if we know there is a single head, then we are not able to pin down the exact value of $X$. So instead $Z$ averages over the two possibilities: that either the first flip or the second flip contributed the 
single head. Thus, we define $Z(\omega) := $
\begin{align*}
&0, &&\text{ if } \omega \in \{(0, 0)\} \\
&1, &&\text{ if } \omega \in \{(1, 1)\} \\
&\frac{(1 - p)q}{(1 - p)q + p(1 - q)} \cdot 0 + \frac{p(1 - q)}{(1 - p)q + p(1 - q)} \cdot 1, &&\text{ if } \omega \in \{(1, 0), (0, 1)\}
\end{align*}
The third entry looks a bit complicated, but it is simply a weighted average of the values $X$ can assume on the set $\{(1, 0), (0, 1)\}$. We have to re-weight the probabilities to sum to $1$, as is usual when defining 
conditional probability. Note that in the case where both coins are fair $p = q = \frac{1}{2}$ then each of the weights in the above expression reduce to $\frac{1}{2}$. 

Now that I have made the point that the conditional expectation is a random variable, I will replace $Z(\omega)$ the more common notation $\E[X|\mathcal{G}](\omega)$. This latter notation is clearly must better, 
but to those accustomed to undergraduate-level probability, it may initially be weird to think of something denoted $\E[X|\mathcal{G}]$ as a random variable. 

We now consider generalizing the definition of $\E[X|\mathcal{G}]: \Omega \to \R$ to any $\mathcal{F}$-measurable random variable $X$. We define $\E[X|\mathcal{G}](\omega) := $
\begin{align*}
&X(0, 0), &&\text{ if } \omega \in \{(0, 0)\} \\
&X(1, 1), &&\text{ if } \omega \in \{(1, 1)\} \\
&\frac{(1 - p)q}{(1 - p)q + p(1 - q)} \cdot X(0, 1) + \frac{p(1 - q)}{(1 - p)q + p(1 - q)} \cdot X(1, 0), &&\text{ if } \omega \in \{(1, 0), (0, 1)\}
\end{align*}
Notice that the weighted sum can also be written as 
\[\frac{(1 - p)q}{(1 - p)q + p(1 - q)} \cdot X(0, 1) + \frac{p(1 - q)}{(1 - p)q + p(1 - q)} \cdot X(1, 0) = \frac{\E[X \mathbbm{1}_G]}{\Prob(G)}\]
where $G := \{(1, 0), (0, 1)\}$, as can be verified by considering the calculations for the expectation and probability in the expression. Thus, in general we can think of the conditional expectation as averaging 
over the set $G \in \mathcal{G}$, re-weighting as necessary by its probability $\Prob(G)$. This expression will be utilized when we consider the next generalization of the conditional expectation definition, so 
we give it a name
\[\E[X|G] := \frac{\E[X \mathbbm{1}_G]}{\Prob(G)},\]
Note that although the notation is similar, $G$ is a set, so this is very different from conditioning on a $\sigma$-algebra as in $\E[X|\mathcal{G}]$. 

\subsubsection{What does it mean to condition on a random variable?}
We have given a rigorous (though currently very limited) definition of conditional expectation with respect to a $\sigma$-algebra $\mathcal{G}$. How does this relate to the usual convention of writing conditional 
expectations with respect to random variables? For example, consider $Y$, the number of heads in the experiment. What does $\E[X|Y]$ mean? When we use this notation, the true rigorous meaning underlying it 
is 
\[\E[X|Y] := \E[X|\sigma(Y)],\]
namely, that we condition with respect to the $\sigma$-algebra generated by the random variable. In the case of this example, we know that $\mathcal{G} = \sigma(Y)$ so $\E[X|Y]$ is in fact the conditional expectation 
we have been working with throughout this example!

Note that in undergraduate-level probability we often think of the conditional expectation as a number $\E[X|Y = y]$. Given a different $y$ this will be a different number, but it  is still just a number. By instead defining the 
random variable $\E[X|Y]$ we have essentially have a formula as a function of $Y$ that outputs the same numbers once the experiment occurs. This simple change in thinking allows the conditional expectation to be a much 
more powerful concept. 

\subsection{A More General Definition}
In the previous section, we considered defining the conditional expectation $\E[X|\mathcal{G}]$ in a simple case where the $\sigma$-algebra $\mathcal{G}$ was generated by a finite collection of sets. Naturally, the next 
generalization will be the case where $\mathcal{G}$ is generated by a countable collection of sets. In particular, assume that $\{G_i\}_{i = 1}^{\infty}$ form a partition of the sample space $\Omega$; i.e. they are 
pairwise disjoint and their union is $\Omega$. We suppose $\mathcal{G} = \sigma\left(\{G_i\}_{i = 1}^{\infty}\right)$. Recalling the notation 
\[\E[X|G] := \frac{\E[X\mathbbm{1}_G]}{\Prob(G)}, \text{ for } G \in \mathcal{G}\]
we simply define $\E[X|\mathcal{G}]: \Omega \to \R$ by $\E[X|\mathcal{G}](\omega) := \E[X|G_i]$ if $\omega \in G_i$. This is essentially the same definition as in the previous case, we now simply have a countable 
number of $G_i$. Once again, if all we know is that $\omega \in \mathcal{G}$ then the idea is to average $X$ over $G_i$ to produce a best guess for $X$, re-weighting by $\Prob(G_i)$ as is necessary.

\subsubsection{An Example: Infinite Coin Flips}
\textbf{TODO}: this section is incomplete.
As an example of this generalization, consider the experiment that consists of an infinite number of coin tosses. The probability space $\Omega$ consists of all sequences $\omega: \mathbb{N} \to \{0, 1\}$ of the form 
\[\omega = (\omega_1, \omega_2, \omega_3, \dots), \text{ where } \omega_i \in \{0, 1\}, \text{ for all } i,\]
that is, all infinite sequences of zeros and ones. Constructing a $\sigma$-algebra that allows us to make probabilistic statements about all of the coin tosses is is a bit tricky. For example, we might want to be able to talk about 
the probability that every coin flip is heads. However, it is not hard to construct smaller $\sigma$-algebras that allow us to make statements about finitely many of the coin tosses. Intuitively, we want to define a $\sigma$-algebra
like 
\[\mathcal{F}_n = \{\text{Knowledge of the first $n$ coin tosses}\}\]
for each $n \in \mathbb{N}$. This $\sigma$-algebra should contain all events that are determined entirely by the first $n$ coin tosses. Formally, we can describe the sets in $\mathcal{F}_n$ by:
\[A \in \mathcal{F}_n \iff \exists A_n \subset \{0, 1\}^n \text{ such that } A = \{\omega \in \Omega: (\omega_1, \dots, \omega_n) \in A_n\}\] 
As we increase $n$ then $\mathcal{F}_n$ will contain knowledge of more coin tosses, and should therefore be a larger $\sigma$-algebra. Indeed, it is not difficult to show that 
\[\mathcal{F}_n \subset \mathcal{F}_{n + 1}\]
A natural candidate for a larger $\sigma$-algebra $\mathcal{F}_0$ is the collection resulting from tossing all of the sets in each $\mathcal{F}_n$ into one large collection; that is, 
\[\mathcal{F}_0 := \bigcup_{n \in \mathbb{N}} \mathcal{F}_n\]
Unfortunately, it turns out that $\mathcal{F}_0$ is not actually a $\sigma$-algebra; it is missing some sets that should be included. Thus, the next natural step is to consider the smallest $\sigma$-algebra containing $\mathcal{F}_0$. 
\[\mathcal{F} := \sigma(\mathcal{F}_0)\]
Given the measurable space $(\Omega, \mathcal{F})$ it is now possible to define a probability measure $\mathcal{P}$ that models the concept of ``fair'' coin tosses. I will not give specifics on this but see 
\href{https://www.ee.iitm.ac.in/~krishnaj/EE5110_files/notes/lecture8_Infinite\%20Coin\%20Toss.pdf}{here} for details. 

\subsubsection{Next Steps}
In the interest of taking the next step in generalizing to arbitrary $\sigma$-algebras, not necessarily generated by a countable collection of sets, let's consider the salient property of our current definition of conditional expectation 
that makes it a reasonable definition. I will again denote the conditional expectation as $Z = \E[X|\mathcal{G}]$ as I believe it makes the following explanation easier to understand. Recall that the conditional expectation $Z$ 
takes the knowledge that an outcome $\omega$ is in some $G_i \in \mathcal{G}$ and then produces a guess for the value of $X$ by averaging over the finer-grain $\mathcal{F}$-information in $G_i$ (since this information is not 
available; all that is known is that $\omega \in G_i$). The key thing to note here is that this ``averaging over'' is done with respect to $\mathcal{F}$. Recall the definition 
\[Z(\omega) = \frac{\E[X \mathbbm{1}_{G_i}]}{\Prob(G_i)}, \text{ if } \omega \in G_i\]
The expectation in the numerator is defined with respect to $\mathcal{F}$, since $X$ is $\mathcal{F}$-measurable. 

\subsection{General Definition of Conditional Expectation}

\section{Review for Final Exam}

\subsection{Conditional Expectation}
Throughout this section let $(\Omega, \mathcal{F}, \Prob)$ be a probability space and $\mathcal{G}$ a sub-$\sigma$-algebra of $\mathcal{F}$. 

\begin{definition}
Let $X: \Omega \to \R$ be an integrable $\mathcal{F}$-measurable random variable (denoted $X \in \mathcal{F}$). The conditional expectation of $X$ with respect to $\mathcal{G}$ is a random variable 
$Z \in \mathcal{G}$ satisfying $\E[X \mathbbm{1}_{G}] = \E[Z \mathbbm{1}_{G}]$ for all $G \in \mathcal{G}$. The conditional expectation is typically denoted
as $\E[X|\mathcal{G}] \equiv Z$. 
\end{definition}

The conditional expectation satisfies some properties that we would intuitively hope are true: linearity, monotonicity, conditional expectation of a constant is a.s. equal to that constant. Here are some perhaps less-obvious 
properties: 
\begin{itemize}
\item Law of iterated expectations: $\E \E[X|\mathcal{G}] = \E X$. Note that the outer expectation of the left is wrt $\mathcal{G}$, while the one on the right is wrt $\mathcal{F}$. 
\item If $X \in \mathcal{G}$ then $\E[X|\mathcal{G}] \overset{a.s.}{=} X$. This just says that, given knowledge of $\mathcal{G}$ completely determines $X$, since $X \in \mathcal{G}$. Essentially, $X$ can be treated like a constant. 
\item If $\mathcal{G}$ is independent of $X$ then $\E[X|\mathcal{G}] = \E[X]$. So $\mathcal{G}$ provides no additional information about $X$. 
\item $\abs{\E[X|\mathcal{G}]} \leq \E[\abs{X}|\mathcal{G}]$. More generally, the generalization of Jensen's inequality for conditional expectation holds as we would expect. 
\item If $X$ and $XY$ are integrable, $Y \in \mathcal{G}$ then $\E[XY|\mathcal{G}] = Y\E[X|\mathcal{G}]$ ($Y$ can be treated as a constant). 
\item The tower property: If $\mathcal{F}_1 \subset \mathcal{F}_2$ then 
\[\E[\E[X|\mathcal{F}_1]|\mathcal{F}_2] = \E[X|\mathcal{F}_1] = \E[\E[X|\mathcal{F}_2]|\mathcal{F}_1] \]
In other words, not matter the order of conditioning the best best guess of $X$ is limited by the less granular $\sigma$-algebra, since once you condition on $\mathcal{F}_1$ the information in 
the more granular $\mathcal{F}_2$ is lost. 
\end{itemize}

\subsection{Martingale Basics}

\begin{definition}
A collection of increasing sub-$\sigma$-algebras of $\mathcal{F}$, 
\[\mathcal{F}_1 \subset \mathcal{F}_2 \subset \dots \subset \mathcal{F}\]
 is called a \textit{filtration}. A sequence of random variables $\{X_n\}$ is said to be \textit{adapted} to the filtration $\{\mathcal{F}_n\}$ provided that $X_n \in \mathcal{F}_n$ for all $n$. 
 If instead $X_n \in \mathcal{F}_{n-1}$ for all $n$, then we say that $\{X_n\}$ is $\{\mathcal{F}_n\}$-predictable. 
\end{definition}
We can think of a filtration as capturing knowledge through time, where $\mathcal{F}_n$ captures the knowledge up through time $n$. A sequence of random variables being adapted to the filtration encodes the 
notion that $X_n$ only depends on information up through time $n$ (it does not require looking into the future). A sequence of random variables being predictable means that $X_n$ is completely determined by 
information up through the previous time step $n - 1$. A good example is a betting strategy, where $X_n$ is how much money you are going to bet in round $n$. It is reasonable to think that how much you are going to 
bet is completely determined by how you have fared in the previous rounds. 

A few more notes on filtrations and $\sigma$-algebras. First, define the $\sigma$-algebra $\sigma_\infty = \sigma \left\{\bigcup_{n} \mathcal{F}_n \right\}$, 
which includes all information as time ``goes to infinity''. Next, we typically are considering a sequence $\{X_n\}$ and do not explicitly mention a filtration, which means that we are implicitly assuming the natural filtration
\[\mathcal{F}_n := \sigma\left\{X_0, X_1, \dots, X_n \right\} := \sigma\left(\bigcup_{k = 0}^{n} \sigma(X_k) \right)\]
Recall that the union of $\sigma$-algebras is not necessarily a $\sigma$-algebra, hence the need for the $\sigma$ on the outside of the union. 

\begin{definition}
Let $\{X_n\}$ be an $\{\mathcal{F}_n\}$-adapted sequence of integrable random variables. Then $\{X_n, \mathcal{F}_n\}$ is called a \textit{martingale} if 
\[\E[X_{n + 1}|\mathcal{F}_n] = X_n \text{, for all } n\]
If is called a sub-martingale if $\E[X_{n + 1}|\mathcal{F}_n] \geq X_n$ and a super-martingale if $\E[X_{n + 1}|\mathcal{F}_n] \leq X_n$. We call it an $L^p$-martingale (or sub/super) if, in addition, $\E[\abs{X_n}^p] < \infty$ for all $n$. 
\end{definition}
A martingale encodes the notion of a ``fair game'' in which, given information up through round $n$, the expected payout of future rounds is $0$ (i.e. your expected earnings is whatever your current earnings are). It can also be seen 
as a generalization of a random walk (where in a random walk $X_n$ is simply a sum of $n$ mean-zero independent random variables). One key initial observation is that martingales have constant mean (they are ``mean stationary''). 
Indeed, applying the defining relation of martingales and the law of iterated expectations
\[\E X_{n+1} = \E \E[X_{n + 1}|\mathcal{F}_n] = \E X_n \text{ for all } n\]


To explore the idea of martingales generalizing random walks, consider a random walk
\[X_n = \sum_{k = 0}^{n} Y_k\]
where the $Y_k$ are independent, mean-zero. The key property of a random walk is that it has independent increments $X_{n + 1} - X_n$ (clearly, since this difference equals $Y_k$ and the $Y_k$ are independent). We can thus 
view a random walk as some starting condition plus its increments up to the current time: 
\[X_n = Y_0 + \sum_{k = 1}^{n} Y_k\]
where we note that $X_0 = Y_0$. The initial condition is in general a mean-zero random variable, independent of the increments, though it is often simply set to zero, $X_0 \equiv 0$ so that the process starts at the origin. Martingales 
generalize this notion by allowing processes which look like 
\[X_n = \sum_{k = 0}^{n} U_k\]
where the $U_k$ are mean-zero, but not necessarily independent. In the case of martingales, the $U_k$ are uncorrelated (a weaker condition), which can also be thought of as an orthogonality condition. So while random walks have 
independent increments, martingales of this form have orthogonal increments. I should emphasize at this point that martingales are more general than sums of mean zero random variables; for example, we will see product forms of 
martingales in addition to the sum forms discussed here. That being said, martingales can always be decomposed as a sum of its increments in this way. Indeed, let $\{X_n\}$ be a generic martingale. Then 
\[X_n = X_0 + \sum_{k = 1}^{n} (X_k - X_{k-1})\]
which simply follows from the terms cancelling in the telescoping sum. Now, we can define $U_0 := X_0$ and $U_n := X_n - X_{n - 1}$, $n \geq 1$ so that this can be re-written as 
\[X_n = \sum_{k = 0}^{n} U_k\]
Note that since martingales have constant expectation then 
\[\E U_n = \E[X_n - X_{n-1}] = 0\]
so the increments are indeed mean zero. Also note that $U_n = X_n - X_{n-1} \in \mathcal{F}_n$ so that $\{U_n\}$ is $\{\mathcal{F}_n\}$-adapted. Finally note that the increments satisfy a property analogous to the defining relation for 
martingales. 
\[\E[U_{n + 1}|\mathcal{F}_n] = \E[X_{n + 1} - X_n|\mathcal{F}_n] = \E[X_{n + 1}] - X_n = \E[X_{n + 1} - X_n] = 0\]
so that the conditional expectations of the increments, in addition to the unconditional ones, are also $0$. The above argument is a very common one; it uses linearity of conditional expectation, the fact that $X_n \in \mathcal{F}_n$ and hence 
gets pulled out like a constant, and $X_{n + 1}$ is independent of $\mathcal{F}_n$ so that its conditional expectation equals the unconditional expectation. We have thus shown that martingales can be decomposed as partial sums of 
a sequence $\{U_n\}$ satisfying the properties just outlined. It makes sense to give this sequence a name. 

\begin{definition}
An $\{F_n\}$-adapted sequence $\{U_n\}$ is called a \textit{martingale difference sequence} provided that $\E[U_{n+1}|\mathcal{F}_n] = 0$. The standard sub/super terminology is also used when the equality is replaced with 
$\geq$ or $\leq$, respectively. 
\end{definition}

Let's now consider some martingales other than random walks. It turns out that the square of an $L_2$ martingale is not quite a martingale, but by subtracting off the mean of the process we can obtain a martingale. 
To show this, let $Y_k$ be independent mean zero random variables with variances $\sigma_k^2$, $S_n = \sum_{k = 1}^{n} Y_k$ their partial sum, and denote $s_n^2 := \sum_{k = 1}^{n} \sigma_k^2$. Note that the mean zero restriction on the $Y_k$ is not actually a restriction; these random variables could have 
non-zero mean and then we would just center them to show the analogous result. The claim 
is that $X_n := S_n^2$ is a sub-martingale but $X_n^\prime := S_n^2 - s_n^2$ is a martingale.  To show this, consider 
\begin{align*}
\E[X_{n + 1}|\mathcal{F}_n] &= \E\left[\left(Y_{n + 1} + S_n \right)^2 \bigg| \mathcal{F}_n \right] \\
					   &= \E[Y_{n + 1}^2|\mathcal{F}_n] + \E[S_n^2|\mathcal{F}_n] + 2\E[Y_{n+1}S_n|\mathcal{F}_n] \\
					   &= \E[Y_{n + 1}^2] + S_n^2 + 2 S_n \E[Y_{n+1}|\mathcal{F}_n] \\
					   &= \sigma_{n + 1}^2 + S_n^2 + 2S_n \E[Y_{n+1}] \\
					   &= X_n + \sigma_{n + 1}^2 \geq S_n^2
\end{align*}
so this is indeed a sub-martingale. Notice that the last three steps use the standard measurability and independence arguments discussed above (note that $S_n^2 \in \mathcal{F}_n$). 
An almost identical calculation shows that the correction $X_n^\prime := S_n^2 - s_n^2$ results in a martingale. 

It turns out that many martingales also take a product form. If $Y_k$ are independent with mean $1$, then 
\[X_n = \prod_{k = 1}^{n} Y_k\]
can easily be shown to be a martingale. The argument is very similar as the above, where the final term is split off of the product, rather than the sum and then measurability and independence 
arguments are applied. We finish this introduction with a couple more examples of special types of martingales. First, let $\{\mathcal{F}_n\}$ be a filtration, $Z$ a random variable with finite 
expectation Then define 
\[X_n := \E[Z|\mathcal{F}_n]\]
I claim this defines a martingale. We can think of the random process $\{X_n\}$ as representing the acquisition of more and more information about $Z$ over time. Before proving that it actually holds, 
consider the defining relation $\E[X_{n+1} |\mathcal{F}_n] = X_n$. This is a bit odd to interpret given that $X_n := \E[Z|\mathcal{F}_n]$. Loosely speaking, I like to think of $X_n$ as the ``best guess'' of 
$Z$ given knowledge up to time $n$. The defining relation thus says that considering a ``better guess'' $\E[Z|\mathcal{F}_{n+1}]$ does not gain us any thing if we are conditioning on $\mathcal{F}_n$; i.e.
conditioning on $\mathcal{F}_n$ drops all additional knowledge gained from $\mathcal{F}_{n+1}$ (this is essentially just the tower law, and indeed the proof relies on the tower law as I show below).
\[\E[X_{n + 1}|\mathcal{F}_n] = \E[\E(Z|\mathcal{F}_{n+1})|\mathcal{F}_n] = \E[Z|\mathcal{F}_n] = X_n\] 
This martingale seems pretty odd, but it turns out to have some very nice and important properties. 

Using similar arguments, we can actually devise a way to adjust any integrable, adapted sequence $\{Y_k\}$ into a martingale. Indeed, set $X_0 = Y_0$ and define 
\[X_n = \sum_{k = 1}^{n} \left(Y_k - \E[Y_k|\mathcal{F}_{k-1}] \right), n \geq 1\]
$\{X_n, \mathcal{F}_n\}$ is a martingale. This makes sense; we are just adjusting the increments (i.e. martingale differences) to have conditional mean zero so that the defining relation 
will hold. 

\subsection{Martingale Algebra}
Whenever defining a new mathematical object, it is typical to study classes of transformations that preserve the structure of the object. In this vein, one can show the following. 
Let $\{X_n\}$, $\{Q_n\}$ be martingales. 
\begin{enumerate}
\item linear combinations of martingales are martingales; i.e. a $\{a X_n + b Q_n\}$. If $X_n$ and $Q_n$ are instead submartingales, then \textit{positive} linear combinations 
$(a, b > 0)$ preserve the submartingale structure. The same is true of supermartingales.  
\item $\max\{X_n, Q_n\}$ is a submartingales. this makes sense since submartingales are increasing random processes, in the sense that expectations of future states given the 
past are higher than the current state. 
\item Similarly, $\min\{X_n, Q_n\}$ is a supermartingale. 
\item Changing the sign of a supermartingale (i.e. $a \cdot X_n$, where $a < 0$) results in submartingale, and vice versa. 
\item A Jensen-type result: if $\{X_n\}$ is a martingale and $g$ a convex function then $\{g(X_n)\}$ is a sub-martingale. 
\item If $\{X_n\}$ is a martingale then so is $\{\abs{X_n}\}$ as well as the positive and negative parts $\{X_n^+\}$ and $\{X_n^-\}$. 
\item If $X_n \in L_p(\Omega)$ and $p > 1$ then $\{\abs{X_n}^p\}$ (this is a consequence of the Jensen-type result due to convexity). 
\end{enumerate}

\subsection{Orthogonality}
We have already noted that martingales have orthogonal (i.e. uncorrelated) increments. We make this more precise here. To even talk about orthogonality, we must 
have $X_n \in L_2(\Omega)$, hence throughout this section we assume $\{X_n\}$ is an $L_2$ martingale. We showed before that any martingale can be represented 
as the partial sums of a martingale difference sequence $\{U_k\}$, such that 
\[X_n = \sum_{k = 0}^{n} U_k\]
Since $\E U_k = 0$ then the claim of orthogonal increments is a claim that $\E U_k U_j = 0$ for $j > k$. Conditioning on the less informative (``coarser'') $\sigma$-algebra, 
\[\E[U_k U_j | \mathcal{F}_k] = U_k \E[U_j | \mathcal{F}_k] = U_k \E[U_j] = U_k \cdot 0 = 0\]
where the second step uses independence. Now taking expectations of both sides and applying the law of iterated expectations yields the result 
\[\E \E[U_k U_j|\mathcal{F}_k] = \E U_k U_j = 0\]
This orthogonality implies 
\[\E \left(\sum_{i = k + 1}^{j} U_i \right)^2 = \sum_{i = k + 1}^{j} \E U_i^2\]

\subsection{Decompositions}




\end{document}


