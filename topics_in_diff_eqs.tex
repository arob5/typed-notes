\documentclass[12pt]{article}
\RequirePackage[l2tabu, orthodox]{nag}
\usepackage[main=english]{babel}
\usepackage[rm={lining,tabular},sf={lining,tabular},tt={lining,tabular,monowidth}]{cfr-lm}
\usepackage{amsthm,amssymb,latexsym,gensymb,mathtools,mathrsfs}
\usepackage[T1]{fontenc}
\usepackage[utf8]{inputenc}
\usepackage[pdftex]{graphicx}
\usepackage{epstopdf,enumitem,microtype,dcolumn,booktabs,hyperref,url,fancyhdr}
\usepackage{algorithmic}
\usepackage[ruled,vlined,commentsnumbered,titlenotnumbered]{algorithm2e}
\usepackage{bbm}

% Plotting
\usepackage{pgfplots}
\usepackage{xinttools} % for the \xintFor***
\usepgfplotslibrary{fillbetween}
\pgfplotsset{compat=1.8}
\usepackage{tikz}

% Custom Commands
\newcommand*{\norm}[1]{\left\lVert#1\right\rVert}
\newcommand*{\abs}[1]{\left\lvert#1\right\rvert}
\newcommand*{\suchthat}{\,\mathrel{\big|}\,}
\newcommand{\E}{\mathbb{E}}
\newcommand{\Var}{\mathrm{Var}}
\newcommand{\R}{\mathbb{R}}
\newcommand{\N}{\mathcal{N}}
\newcommand{\Ker}{\mathrm{Ker}}
\newcommand{\Cov}{\mathrm{Cov}}
\newcommand{\Prob}{\mathbb{P}}
\newcommand{\btheta}{\boldsymbol{\theta}}
\newcommand{\bx}{\mathbf{x}}
\newcommand{\bb}{\mathbf{b}}
\newcommand{\bw}{\mathbf{w}}
\newcommand{\be}{\mathbf{e}}

% Differential Equations
\newcommand{\state}{x}
\newcommand{\stateb}{\mathbf{x}}


\DeclarePairedDelimiterX\innerp[2]{(}{)}{#1\delimsize\vert\mathopen{}#2}
\DeclareMathOperator*{\argmax}{argmax}
\DeclareMathOperator*{\argmin}{argmin}
\DeclarePairedDelimiter{\ceil}{\lceil}{\rceil}


\setlist{topsep=1ex,parsep=1ex,itemsep=0ex}
\setlist[1]{leftmargin=\parindent}
\setlist[enumerate,1]{label=\arabic*.,ref=\arabic*}
\setlist[enumerate,2]{label=(\alph*),ref=(\alph*)}

% For embedding images
\graphicspath{ {./images/} }

% Specifically for paper formatting 
\renewcommand{\baselinestretch}{1.2} % Spaces manuscript for easy reading

% Formatting definitions, propositions, etc. 
\newtheorem{definition}{Definition}
\newtheorem{prop}{Proposition}
\newtheorem{lemma}{Lemma}
\newtheorem{thm}{Theorem}
\newtheorem{corollary}{Corollary}

% Title and author
\title{Topics in Differential Equations and Dynamical Systems}
\author{Andrew Roberts}

\begin{document}

\maketitle
\tableofcontents
\newpage

% Fredholm Alternative and Adjoint Operators
\section{Fredholm Alternative and Adjoint Operators}

\subsection{Fredholm Alternative}
The \textit{Fredholm alternative} refers to a few different theorems, which concern linear algebra, integral equations, and PDEs, among other things. 
We begin with the simplest, finite-dimensional case. 

\subsubsection{Finite Linear Systems}

\subsection{Adjoints of Differential Operators}
Motivate that adjoint equation essentially arises from integration by parts. But since it's not necessarily defined on the whole space, and since 
boundary conditions might mess up the exact adjoint equivalence, we instead define what's known as the formal adjoint. 

% Numerical Methods
\section{Numerical Methods for ODEs}
Here we consider numerical integrators for solving the initial value problem (IVP) 
\begin{align}
&\dot{x}(t) = f(x(t), t), &&x(t_0) = x_0, \label{IVP}
\end{align}
where $x(t) \in \R^n$ and $f: \R^{n+1} \to \R^n$. We will adopt the notational convention of suppressing the 
functional dependence on $t$, writing \ref{IVP} as $\dot{x} = f(x,t)$. Note that this setup considers non-autonomous 
systems, where the vector field $f(x,t)$ is itself allowed to change with time. The numerical schemes we will consider 
here seek to approximate the trajectory $\{x(t) : t_0 \leq t \leq T\}$ by constructing a discretized approximation 
$\{x_k\}_{k=0}^{K}$, where 
\begin{align}
x_k \approx x(t_k), \qquad t_k := k \Delta t + t_0,  \qquad K := \frac{T-t_0}{\Delta t},  \qquad k = 0, 1, \dots, K.
\end{align}
Here, $\Delta t > 0$ is some fixed \textbf{step size}, although there also adaptive algorithms that consider tuning the step size as the algorithm progresses. 
One thing to keep in mind as we progress through numerical integration schemes is that one major concern is that errors can accumulate 
over time. The approximation error introduced at time step $t_k$ will propagate to time step $t_{k+1}$, and so on. Can tiny errors quickly snowball into 
large issues as time progresses, or will the algorithm be relatively robust to small errors? These are the kind of questions we must answer in 
order to be confident that the numerical approximations returned by these algorithms are faithful to the exact trajectories. 

The reason these algorithms are called \textbf{numerical integrators} follows from considering the integral formulation of \ref{IVP}, 
\begin{align}
x(t) = x_0 + \int_{t_0}^{t} f(x(s), s) ds. 
\end{align}
Given the value $x(t_k)$ we can consider the analogous integral that produces the value $x(t_{k+1})$, 
\begin{align}
x(t_{k+1}) = x(t_k) + \int_{t_k}^{t_{k+1}} f(x(s), s) ds = x(t_k) + \int_{t_k}^{t_k + \Delta t} f(x(s), s) ds \label{IVP_integral}
\end{align}

If we could compute the integral in \ref{IVP_integral} exactly, then we could exactly calculate the value of $x(t_k)$ at any chosen point $t_k$; i.e., 
$x_k = x(x(t_k))$. Since this is typically not possible, numerical solution techniques typically can be viewed as approximating this integral in some way.  


\subsection{Forward Euler}
Note that $f(x,t)$ is a (potentially time-varying) vector field, giving the instantaneous velocity of the particle $x(t)$. Perhaps the most obvious numerical 
approach is thus to simply take a small step in the direction indicated by the vector field,
\begin{align}
x_{k+1} := x_k + \Delta t f(x_k, t_k). \label{forward_Euler}
\end{align}
The algorithm given by repeating the update \ref{forward_Euler} is known as \textbf{forward Euler}. It is reminiscent of a gradient descent update with 
step size $\Delta t$, except that the \textit{descent direction} is replaced by the direction indicated by the vector field $f$. Looking at this from the 
integration point of view \ref{IVP_integral}, suppose we have current value $x_k$ and seek to produce $x_{k+1}$. The forward Euler update follows from 
the approximation 
\begin{align*}
x(t_{k+1}) = x(t_k) + \int_{t_k}^{t_k + \Delta t} f(x(s),s) ds \approx x_k + \int_{t_k}^{t_k + \Delta t} f(x_k,t_k) ds = x_k + \Delta t f(x_k, t_k) =: x_{k+1}
\end{align*}
Notice that there are two sources of approximation error here: 
\begin{enumerate}
\item The true value $x(t_k)$ is replaced by the approximation $x_k$ produced by the previous step of the algorithm. 
\item The integral is approximated by assuming that its value is approximately constant in the interval $[t_k, t_k + \Delta t]$, with the constant value given by 
evaluating the integrand at the left endpoint.
\end{enumerate}
The choice to evaluate at the left endpoint is what gives the \textit{forward} Euler method its name. 


\end{document}


