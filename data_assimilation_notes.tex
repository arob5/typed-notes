\documentclass[12pt]{article}
\RequirePackage[l2tabu, orthodox]{nag}
\usepackage[main=english]{babel}
\usepackage[rm={lining,tabular},sf={lining,tabular},tt={lining,tabular,monowidth}]{cfr-lm}
\usepackage{amsthm,amssymb,latexsym,gensymb,mathtools,mathrsfs}
\usepackage[T1]{fontenc}
\usepackage[utf8]{inputenc}
\usepackage[pdftex]{graphicx}
\usepackage{epstopdf,enumitem,microtype,dcolumn,booktabs,hyperref,url,fancyhdr}
\usepackage{algorithmic}
\usepackage[ruled,vlined,commentsnumbered,titlenotnumbered]{algorithm2e}
\usepackage{bbm}

% Plotting
\usepackage{pgfplots}
\usepackage{xinttools} % for the \xintFor***
\usepgfplotslibrary{fillbetween}
\pgfplotsset{compat=1.8}
\usepackage{tikz}

% Custom Commands
\newcommand*{\norm}[1]{\left\lVert#1\right\rVert}
\newcommand*{\abs}[1]{\left\lvert#1\right\rvert}
\newcommand*{\suchthat}{\,\mathrel{\big|}\,}
\newcommand{\E}{\mathbb{E}}
\newcommand{\Var}{\mathrm{Var}}
\newcommand{\R}{\mathbb{R}}
\newcommand{\N}{\mathcal{N}}
\newcommand{\Ker}{\mathrm{Ker}}
\newcommand{\Cov}{\mathrm{Cov}}
\newcommand{\Prob}{\mathbb{P}}
\DeclarePairedDelimiterX\innerp[2]{(}{)}{#1\delimsize\vert\mathopen{}#2}
\DeclareMathOperator*{\argmax}{argmax}
\DeclareMathOperator*{\argmin}{argmin}
\DeclarePairedDelimiter{\ceil}{\lceil}{\rceil}

\setlist{topsep=1ex,parsep=1ex,itemsep=0ex}
\setlist[1]{leftmargin=\parindent}
\setlist[enumerate,1]{label=\arabic*.,ref=\arabic*}
\setlist[enumerate,2]{label=(\alph*),ref=(\alph*)}

% For embedding images
\graphicspath{ {./images/} }

% Specifically for paper formatting 
\renewcommand{\baselinestretch}{1.2} % Spaces manuscript for easy reading

% Formatting definitions, propositions, etc. 
\newtheorem{definition}{Definition}
\newtheorem{prop}{Proposition}
\newtheorem{lemma}{Lemma}
\newtheorem{thm}{Theorem}
\newtheorem{corollary}{Corollary}

% Title and author
\title{Data Assimilation: Filtering and Smoothing}
\author{Andrew Roberts}

\begin{document}

\maketitle
\tableofcontents
\newpage

% General Framework
\section{The General Framework}

\subsection{The Stochastic Dynamics Model}
We consider the following stochastic dynamics model. 
\begin{align}
&x_{t + 1} = g(x_t) + \xi_t, && \xi_t \overset{iid}{\sim} \mathcal{N}(0, \Sigma_x) \label{stochastic_dynamics_model} \\
&y_{t + 1} = h(x_{t+1}) + \epsilon_{t+1}, && \epsilon_t \overset{iid}{\sim} \mathcal{N}(0, \Sigma_\epsilon) \nonumber 
\end{align}
The first line describes a stochastic Markov dynamical system, while the second defines an observation model. Moreover, we assume a random initial condition 
\[x_0 \sim \mathcal{N}(\mu_0, \Sigma_0)\]
and assume independence $x_0 \perp \{\xi_t\} \perp \{\epsilon_t\}$. We suppose the states $x_t$ and observations $y_t$ are $P$ and $N$-dimensional vectors, respectively. 
In large-scale settings, the case $D > N$ is quite common, as the state vector may be derived, for example, from a fine discretization of space. 

\subsection{Filtering and Smoothing}
I will follow Stuart et al's notation and denote $Y_t := \{y_1, y_2, \dots, y_t\}$. Given this notation, the \textit{filtering problem} is concerned with characterizing the following distribution:
\[\pi_{t}(x_t) := p(x_t|Y_t)\]
That is, the distribution of the current state given observations up through the current time. The primary goal of the filtering problem is to iteratively update the filtering distribution $\pi_t$ in 
an efficient manner as new data arrives. When a new observation $y_{t + 1}$ arrives, we can think of this update as consisting of two steps. 
\begin{enumerate}
\item \textbf{Prediction:} Iterate the dynamical model one time step forward to obtain the predictive distribution $\hat{\pi}_{t+1}(x_{t+1}) := p(x_{t+1}|Y_t)$. We can think of this as applying 
an operator $\mathcal{P}$ which transforms the current filtering distribution into the predictive distribution: $\hat{\pi}_{t+1} = \mathcal{P}\pi_t$. 
\item \textbf{Analysis:} Incorporate the new observation $y_{t+1}$ via Bayes' rule to obtain the new filtering distribution $\pi_{t+1}(x_{t+1})$. This can be thought of again as 
applying a map $\mathcal{A}_t$ (which in this case depends on $t$, since it depends on the data), so that $\pi_{t + 1} = \mathcal{A}_t \hat{\pi}_{t+1}$. 
\end{enumerate}
Putting these together, the main goal of filtering is to characterize the updated distribution 
\[\pi_{t + 1} = \mathcal{A}_t \mathcal{P} \pi_t \]

We now derive the form taken by the two operators. 

\bigskip
\noindent
\textbf{Prediction.} The prediction step simply means iterating the stochastic dynamics forward. Thus, obtaining $\hat{\pi}_{t+1}(x_{t+1})$ requires marginalizing 
over the current state $x_t$, weighted by the current filtering distribution $\pi_t(x_t)$. 
\begin{align*}
\hat{\pi}_{t+1}(x_{t+1}) &= p(x_{t+1}|Y_t) \\
				   &= \int p(x_t, x_{t+1}|Y_t) dx_t \\ 
				   &= \int p(x_{t+1}|x_t, Y_t) p(x_t|Y_t) dx_t \\
				   &= \int p(x_{t+1}|x_t) p(x_t|Y_t) dx_t \\
				   &= \int \mathcal{N}_P(x_{t+1}|g(x_t), \Sigma_g) \pi_t(x_t) dx_t
\end{align*}
Note that the fact that we need only integrate over the previous state $x_t$ depends completely on the Markov assumption in the dynamics model. Note also that this 
step is independent of $t$ due to the assumption that the dynamical model is time-homogenous. 

\bigskip
\noindent
\textbf{Analysis.} The analysis step entails combining the predictive distribution $\hat{\pi}_{t+1}(x_{t+1})$ with the observed data $y_{t+1}$. Hence, this step 
is simply an application of Bayes' rule. The analysis operator $\mathcal{A}_t$ is thus a likelihood operator, as the predictive distribution is simply weighted by 
the likelihood $p(y_{t+1}|x_{t+1})$. 
\begin{align*}
\pi_{t+1}(x_{t+1}) &= p(x_{t+1}|Y_{t+1}) \\
			   &= p(x_{t+1}|Y_t, y_{t+1}) \\
			   &= \frac{p(y_{t+1}|x_{t+1}, Y_t)p(x_{t+1}|Y_t)}{p(y_{t+1}|Y_t)} \\
			   &= \frac{p(y_{t+1}|x_{t+1})p(x_{t+1}|Y_t)}{p(y_{t+1}|Y_t)} \\
			   &= \frac{\mathcal{N}_N(y_{t+1}|h(x_{t+1}), \Sigma_{\epsilon}) \hat{\pi}_{t+1}(x_{t+1})}{p(y_{t+1}|Y_t)}
\end{align*}
Thus, $\mathcal{A}_t$ simply represents multiplication of $\hat{\pi}_{t+1}(x_{t+1})$ by the likelihood $\mathcal{N}_N(y_{t+1}|h(x_{t+1}), \Sigma_{\epsilon})$
and then normalization in order to result in a valid probability distribution. 

% The Linear Gaussian Model 
\section{Linear Gaussian Model}
The linear Gaussian model is the special case of the stochastic dynamics model \ref{stochastic_dynamics_model} where the one-step operator is linear 
\[g(x_t) = Gx_t, \text{ where } G \in \R^{P \times P}\]
as well as the observation operator 
\[h(x_t) = Hx_t, \text{ where } H \in \R^{N \times P}\]

\subsection{The Kalman Filter}
Under the linear Gaussian assumptions here, the prediction $\mathcal{P}$ and analysis $\mathcal{A}_t$ operators preserve Gaussianity. This implies that 
all of the predictive and filtering distributions are Gaussian. The Kalman filter is simply a recursive update that yields the updated mean and covariance matrix 
of the filtering distribution when new data arrives. In particular, note that this means that in the linear Gaussian setting the filtering distributions are available in 
closed-form. 

\begin{thm}
Under the linear Gaussian assumption, the filtering distributions $\{\pi_t\}_{t \in \mathbb{N}}$ and predictive distributions $\{\hat{\pi}_{t}\}_{t \in \mathbb{N}}$ are all 
Gaussian. 
\begin{align*}
\hat{\pi}_{t+1}(x_{t+1}) &= p(x_{t+1}|Y_{t}) = \mathcal{N}_P(x_{t+1}|\hat{\mu}_{t+1}, \hat{\Sigma}_{t+1}) \\
\pi_{t+1}(x_{t+1}) &= p(x_{t+1}|Y_{t+1}) = \mathcal{N}_P(x_{t+1}|\mu_{t+1}, \Sigma_{t+1})
\end{align*}
Moreover, the means and covariance matrices of these formulas satisfy the following recursions. 
\begin{align*}
\hat{\mu}_{t + 1} &= G \mu_{t} \\
\hat{\Sigma}_{t + 1} &= G \Sigma_t G^\top + \Sigma_g \\
\Sigma_{t + 1} &= \left[\hat{\Sigma}^{-1}_{t + 1} + H^\top \Sigma_{\epsilon} H \right]^{-1} \\
\mu_{t + 1} &= \Sigma_{t + 1} \left[\hat{\Sigma}^{-1}_{t + 1}G \mu_t + H^\top \Sigma_{\epsilon}^{-1} y_{t+1} \right] 
\end{align*}
\end{thm}


\end{document} 



