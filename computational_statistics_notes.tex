\documentclass[12pt]{article}
\RequirePackage[l2tabu, orthodox]{nag}
\usepackage[main=english]{babel}
\usepackage[rm={lining,tabular},sf={lining,tabular},tt={lining,tabular,monowidth}]{cfr-lm}
\usepackage{amsthm,amssymb,latexsym,gensymb,mathtools,mathrsfs}
\usepackage[T1]{fontenc}
\usepackage[utf8]{inputenc}
\usepackage[pdftex]{graphicx}
\usepackage{caption}
\usepackage{subcaption}
\usepackage{epstopdf,enumitem,microtype,dcolumn,booktabs,hyperref,url,fancyhdr}

% Plotting
\usepackage{pgfplots}
\usepackage{xinttools} % for the \xintFor***
\usepgfplotslibrary{fillbetween}
\pgfplotsset{compat=1.8}
\usepackage{tikz}

% Custom Commands
\newcommand*{\norm}[1]{\left\lVert#1\right\rVert}
\newcommand*{\abs}[1]{\left\lvert#1\right\rvert}
\newcommand*{\suchthat}{\,\mathrel{\big|}\,}
\newcommand{\E}{\mathbb{E}}
\newcommand{\Var}{\mathrm{Var}}
\newcommand{\R}{\mathcal{R}}
\newcommand{\N}{\mathcal{N}}
\newcommand{\Ker}{\mathrm{Ker}}
\newcommand{\Cov}{\mathrm{Cov}}
\newcommand{\Prob}{\mathbb{P}}
\DeclarePairedDelimiterX\innerp[2]{(}{)}{#1\delimsize\vert\mathopen{}#2}
\DeclareMathOperator*{\argmax}{argmax}
\DeclareMathOperator*{\argmin}{argmin}
\def\R{\mathbb{R}}
\DeclarePairedDelimiter\ceil{\lceil}{\rceil}
\DeclarePairedDelimiter\floor{\lfloor}{\rfloor}

\setlist{topsep=1ex,parsep=1ex,itemsep=0ex}
\setlist[1]{leftmargin=\parindent}
\setlist[enumerate,1]{label=\arabic*.,ref=\arabic*}
\setlist[enumerate,2]{label=(\alph*),ref=(\alph*)}

% Specifically for paper formatting 
\renewcommand{\baselinestretch}{1.2} % Spaces manuscript for easy reading

% Formatting definitions, propositions, etc. 
\newtheorem{definition}{Definition}
\newtheorem{prop}{Proposition}
\newtheorem{lemma}{Lemma}
\newtheorem{thm}{Theorem}
\newtheorem{corollary}{Corollary}

\begin{document}

\begin{center}
\Large
Notes on Computational Statistics/Monte Carlo Methods
\end{center}

\begin{flushright}
Andrew Roberts
\end{flushright} 

\section{Sampling}

\subsection{Linchpin Sampling}
Suppose we are interested in sampling from the marginal distribution of a random variable $X$, and we know how to sample from the distributions or $X|Y$ 
and $Y$, for some other random variable $Y$. We can thus exploit the relationship
\[p(X, Y) = p(X|Y)p(Y)\]
to generate samples from the joint distribution of $(X, Y)$. The idea is then to just ``throw away'' the samples from $Y$, leaving us with a random sample from the marginal
distribution of $X$. Here, $Y$ is called the \textbf{linchpin variable}, given that it is vital in allowing us to sample from the marginal of $X$ even though we aren't interested in 
$Y$ itself. The algorithm for generating a single sample from the joint distribution is given below. 
\begin{itemize}
\item Sample $y \sim \pi_Y$
\item Sample $x \sim \pi_{X|Y=y}$
\item Output $(x, y)$
\end{itemize}

The correctness of this algorithm follows almost by definition of what it means to sample from the joint and marginal distributions. Although perhaps a bit overkill, 
I will state it a bit more formally below. 
\begin{prop}
Suppose $(x, y)$ is constructed as above. Then, 
\begin{enumerate}
\item $(x, y) \sim \pi_{X, Y}$
\item $x \sim X$
\end{enumerate}
\end{prop}

\begin{proof}
We begin by proving the first claim. Note that by definition of the algorithm $y \overset{d}{=} Y$ so that $y$ has distribution $\pi_Y$. Now, with Borel sets 
$A \in \mathcal{B}(\mathcal{X})$, $B \in \mathcal{B}(\mathcal{Y})$ we have 
\[\Prob((x, y) \in A \times B) = \Prob(x \in A | y \in B)\Prob(y \in B) = \pi_{X|Y\in B}(A)\pi_Y(B) = \pi_{X, Y}(A, B)\]
which shows $(x, y) \sim \pi_{X, Y}$. 

We now show the second claim, which pretty much follows directly from the joint distribution result above. 
\[\pi_X(A) = \Prob(X \in A) = \Prob(X \in A, Y \in \mathcal{Y}) = \Prob(x \in A, y \in \mathcal{Y}) = \Prob(x \in A)\]
The third equality uses the equality of the joint distributions proved above. 
\end{proof}
This proof is largely overkill; it is probably best just to realize that the result follows trivially from the fact that the joint distribution can be factored as a product
of a conditional and marginal. As a final point, we recall that the marginal distribution of $X$ can be obtained by integrating $Y$ out of the joint distribution. 
\[\pi_X(A) = \int_\mathcal{Y} \pi_{X, Y}(A, y) \pi_Y(dy)\]
This ``integrating out`` step is analogous to the dropping the samples from $Y$ in the above algorithm to obtain marginal samples from $X$. In other words, 
ignoring $Y$ is just like averaging over $Y$. 

\subsection{Space-Filling Sampling Algorithms}
The purpose of this section is to introduce various sampling algorithms that are often used in the fields of experimental design and computer experiments. I will introduce the 
basic setup in the latter field as the primary motivation here. The setting is as follows: suppose we seek to model a complex process by running a deterministic computer simulation. 
Mathematically, we represent the computer code as a function $f: \mathcal{X} \to \mathcal{Y}$, where $\mathcal{X}$ is a set of valid inputs to the computer program and 
$\mathcal{Y}$ is the set of outputs resulting from running the simulation on a given input value. There are a variety of tasks one might seek to accomplish with such a simulation, with a 
straightforward example being finding an input $x \in \mathcal{X}$, or set of inputs, that produces the ``best'' output $f(x)$. More commonly, we would like to conduct a full \textit{uncertainty analysis}
to gain a sufficient understanding of the variability in the outputs due to the variability in the inputs. We might even want to go a step further and conduct a \textit{sensitivity analysis}, which 
attributes the variability in the outputs to the variability to specific input dimensions. 

 If the set $\mathcal{X}$ is a small, finite set and $f$ does not 
take long to run then we can obviously just try out every input. In practice, thought, this is rarely the case. It is common that $\mathcal{X} = \R^d$ for some large $d$ and moreover 
running the simulation on a single input $x$ (that is, calculating the value $f(x)$) might take hours or days. The task is thus to choose a set of input values $\mathbb{X} = (x_1, \dots, x_N) \subset \mathcal{X}$
at which to run the simulation in such a way that this small number of runs gives us as much information as possible about the behavior of the simulation $f$. With a computational costly simulation, one
must be very careful about choosing the set of points $\mathbb{X}$. The algorithms described in this section highlight just a few ways of approaching this problem. I emphasize that the optimal algorithm for selecting
$X$ may be highly dependent on the specific question at hand. 


\subsection{Two Approaches}
\subsubsection{Probabilistic Analysis: The Monte Carlo Approach}
As currently described, this problem is completely deterministic. However, in practice we often have some notion as to the quality of certain input variables. We can capture this prior knowledge via a prior 
distribution $\pi_0$ over $\mathcal{X}$. If nothing is known about the inputs a priori then $\pi_0$ may just be the uniform distribution. Assuming a distribution over the input space then induces a distribution 
over the output space; that is, the simulation is now some unknown transformation of the random inputs $f(X)$, where $X \sim \pi_0$. 

\subsubsection{Response Surface Methodology and Surrogate Models}



\section{Notion of ``Exploration vs. Exploitation'' in Computational Statistics}
Include stochastic gradient based methods, EM, Gibbs sampling here 
In particular, compare (cyclic) coordinate descent and Gibbs (these seem very similar) 

Interesting application of EM algorithm: reinforcement learning: 
http://www.cs.toronto.edu/~fritz/absps/dh97.pdf




\end{document}


