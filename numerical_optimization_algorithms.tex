\documentclass[12pt]{article}
\RequirePackage[l2tabu, orthodox]{nag}
\usepackage[main=english]{babel}
\usepackage[rm={lining,tabular},sf={lining,tabular},tt={lining,tabular,monowidth}]{cfr-lm}
\usepackage{amsthm,amssymb,latexsym,gensymb,mathtools,mathrsfs}
\usepackage[T1]{fontenc}
\usepackage[utf8]{inputenc}
\usepackage[pdftex]{graphicx}
\usepackage{epstopdf,enumitem,microtype,dcolumn,booktabs,hyperref,url,fancyhdr}
\usepackage{algorithmic}
\usepackage[ruled,vlined,commentsnumbered,titlenotnumbered]{algorithm2e}
\usepackage{bbm}

% Plotting
\usepackage{pgfplots}
\usepackage{xinttools} % for the \xintFor***
\usepgfplotslibrary{fillbetween}
\pgfplotsset{compat=1.8}
\usepackage{tikz}

% Custom Commands
\newcommand*{\norm}[1]{\left\lVert#1\right\rVert}
\newcommand*{\abs}[1]{\left\lvert#1\right\rvert}
\newcommand*{\suchthat}{\,\mathrel{\big|}\,}
\newcommand{\E}{\mathbb{E}}
\newcommand{\Var}{\mathrm{Var}}
\newcommand{\R}{\mathbb{R}}
\newcommand{\N}{\mathcal{N}}
\newcommand{\Ker}{\mathrm{Ker}}
\newcommand{\Cov}{\mathrm{Cov}}
\newcommand{\Prob}{\mathbb{P}}
\newcommand{\btheta}{\boldsymbol{\theta}}
\newcommand{\bk}{\mathbf{k}}
\DeclarePairedDelimiterX\innerp[2]{(}{)}{#1\delimsize\vert\mathopen{}#2}
\DeclareMathOperator*{\argmax}{argmax}
\DeclareMathOperator*{\argmin}{argmin}
\DeclarePairedDelimiter{\ceil}{\lceil}{\rceil}

\setlist{topsep=1ex,parsep=1ex,itemsep=0ex}
\setlist[1]{leftmargin=\parindent}
\setlist[enumerate,1]{label=\arabic*.,ref=\arabic*}
\setlist[enumerate,2]{label=(\alph*),ref=(\alph*)}

% For embedding images
\graphicspath{ {./images/} }

% Specifically for paper formatting 
\renewcommand{\baselinestretch}{1.2} % Spaces manuscript for easy reading

% Formatting definitions, propositions, etc. 
\newtheorem{definition}{Definition}
\newtheorem{prop}{Proposition}
\newtheorem{lemma}{Lemma}
\newtheorem{thm}{Theorem}
\newtheorem{corollary}{Corollary}

% Title and author
\title{Numerical Optimization Algorithms}
\author{Andrew Roberts}

\begin{document}

\maketitle
\tableofcontents
\newpage

% Gradient Descent 
\section{Gradient Descent}

% Newton's Method
\section{Newton's Method}

% Gauss-Newton Method}
\section{Gauss-Newton Method}
The Gauss-Newton method is a famous iterative algorithm for solving non-linear least squares problems. Recall that linear least squares objectives are of the form 
\begin{align*}
&J(x) = \norm{Ax - y}^2, && x \in \R^D, y \in \R^N, A \in \R^{N \times D}
\end{align*}
with the additional assumption that $N \geq D$. 
In other words, we might define the residuals $r(x) := Ax - y$ and then the least-squares criterion would be $J(x) = \norm{r(x)}^2$. So long as $A$ has linearly independent columns, this 
problem can be uniquely solved in closed form:
\begin{align}
(A^\top A)^{-1} A^\top y = \text{argmin}_{x \in \R^D} \norm{Ax - y}^2 \label{linear_LS}
\end{align}
The Gauss-Newton method addresses the challenge of solving this minimization problem where the function $r: \R^D \to \R^N$ is no longer linear. For example, if we replace the linear forward 
model $A$ with a non-linear function $f: \R^D \to \R^N$ then the residual vector would be a non-linear function of $x$:
\begin{align*}
J(x) &= \norm{f(x) - y}^2 = \norm{r(x)}^2, && r(x) = f(x) - y
\end{align*}
The insertion of non-linearity means that $\min_{x} J(x)$ no longer has a closed-form solution. The Gauss-Newton method is an iterative numerical scheme that, under certain conditions, converges 
to a local optimum of $J(x)$.

The idea behind Gauss-Newton is quite simple. Given a current iterate $x_t$, linearize $r(x)$ about $x_t$ then replace the non-linear objective with the linearized version and solve to obtain the 
next iterate $x_{t+1}$. The linearized problem has a closed-form solution given by \ref{linear_LS}. This process is then iterated until convergence. Assuming that the derivative of the function $r$ 
exists then the linear approximation is given by 
\begin{align}
r_t^{\text{lin}}(x) &:= r(x_t) + Dr(x_t)\left(x - x_t \right) 
\end{align}
For succinctness, I will write $D_t := Dr(x_t)$. We now solve for the next iterate:
\begin{align*}
x_{t + 1} := \text{argmin}_x r_t^{\text{lin}}(x)
\end{align*}
However, it is actually a bit cleaner to optimize for the \text{direction} $x - x_t$ rather than $x$ itself. Letting $v := x - x_t$ denote the direction, then we seek to solve
\begin{align}
v_t &:= \text{argmin}_v \norm{r(x_t) + D_t v}^2 \label{optimal_direction} \\
	      &= \text{argmin}_v \norm{D_t v - [-r(x_t]}^2
\end{align}
The second line makes it clear that this is simply a linear least squares problem \ref{linear_LS}, with $A = D_t$, $x = v$, and $b = -r(x_t)$. The solution is thus 
\begin{align*}
v_t = -\left(D_t^\top D_t \right)^{-1} D_t^\top r(x_t)
\end{align*}
which requires $D_t$ to have full column rank (thus the assumption $N \geq D$ is essential). We can now define the next iterate as 
\begin{align}
x_{t + 1} &= x_t + \alpha_t v_t \label{line_search}
\end{align}
where $\alpha_t > 0$ is a step size parameter that determines how far to move away from $x_t$ in the direction $v_t$. Optimization methods of the form \ref{line_search} are known as 
\textit{line-search algorithms} due to the fact that the next iterate is found by searching along the line spanned by $v_t$. The step sizes $\alpha_t$ are typically chosen adaptively. While 
$\alpha_t$ can in some sense control how much trust we place in the direction $v_t$ found via the linear approximation $r_t^{\text{lin}}(x)$, we should emphasize that the direction of $v_t$ is 
determined by placing complete trust in the linear approximation; the step size only determines how far we move in that direction. This is opposed to so-called \textit{trust-region} methods
(e.g. such as Levenberg-Marquardt discussed below) which explicitly constrains \ref{optimal_direction} so that the solution lies in a ball where the linear approximation is ``trusted''. 

\subsection{Levenberg-Marquardt Method}
The Levenberg-Marquardt Method is quite similar to the general Gauss-Newton method, but replaces the step-size parameter $\alpha_t$ with new parameters $\delta_t$, $\Sigma$ which explicitly defines 
the ball in which the linear approximation is trusted when optimizing for the direction $v_t$. This method replaces the unconstrained optimization problem \ref{optimal_direction} with the following 
constrained problem:
\begin{align}
v_t &:= \text{argmin}_v \norm{r(x_t) + D_t v}^2 \text{   s.t.} && \norm{v}^2_{\Sigma} \leq \delta_t \label{LM_constrained_opt}
\end{align}
The matrix $\Sigma$ can be thought of as expressing prior information regarding the smoothness of the function $r$. The next iterate is then set to 
\begin{align*}
x_{t+1} &:= x_t + v_t
\end{align*}
Using Lagrange multipliers, it can be shown that \ref{LM_constrained_opt} can be converted to the unconstrained optimization 
\begin{align}
v_t &:= \text{argmin}_v\left( \norm{r(x_t) + D_t v}^2 + \frac{1}{\alpha_t} \norm{v}^2_{\Sigma}\right) \label{LM_unconstrained_opt}
\end{align}
where there is a one-to-one correspondence between $\delta_t$ and $\alpha_t$. We see that $\alpha_t$ acts as a sort of step-size parameter, in very similar fashion to the optimization perspective 
for the gradient descent update. The formulation \ref{LM_unconstrained_opt} demonstrates the the Levenberg-Marquardt update consists of optimizing the increment to $x_t$ via a regularized 
least-squares problem. In the context of a statistical model or inverse problem, where the residual $r(x)$ quantifies the error between the model using parameter $x$ and the observed data, then 
using the Levenberg-Marquardt method can \textit{implicitly} regularize the problem even when there is no regularization term in $r(x)$. For example, in an ill-posed inverse problem $r(x)$ may be very 
sensitive to the observed data, and hence the derivative $D_t$ will also be sensitive to the data. The regularization in \ref{LM_unconstrained_opt} can help to place less trust in $D_t$ in this setting and hence 
help to discourage overfitting when solving the inverse problem.  


% Iteratively Re-Weighted Least Squares and Fisher Scoring
\section{Iteratively Re-Weighted Least Squares and Fisher Scoring}


\end{document}


