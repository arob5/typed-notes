\documentclass[12pt]{article}
\RequirePackage[l2tabu, orthodox]{nag}
\usepackage[main=english]{babel}
\usepackage[rm={lining,tabular},sf={lining,tabular},tt={lining,tabular,monowidth}]{cfr-lm}
\usepackage{amsthm,amssymb,latexsym,gensymb,mathtools,mathrsfs}
\usepackage[T1]{fontenc}
\usepackage[utf8]{inputenc}
\usepackage[pdftex]{graphicx}
\usepackage{epstopdf,enumitem,microtype,dcolumn,booktabs,hyperref,url,fancyhdr}
\usepackage{algorithmic}
\usepackage[ruled,vlined,commentsnumbered,titlenotnumbered]{algorithm2e}
\usepackage{bbm}

% Plotting
\usepackage{pgfplots}
\usepackage{xinttools} % for the \xintFor***
\usepgfplotslibrary{fillbetween}
\pgfplotsset{compat=1.8}
\usepackage{tikz}

% Custom Commands
\newcommand*{\norm}[1]{\left\lVert#1\right\rVert}
\newcommand*{\abs}[1]{\left\lvert#1\right\rvert}
\newcommand*{\suchthat}{\,\mathrel{\big|}\,}
\newcommand{\E}{\mathbb{E}}
\newcommand{\Var}{\mathrm{Var}}
\newcommand{\R}{\mathbb{R}}
\newcommand{\N}{\mathcal{N}}
\newcommand{\Ker}{\mathrm{Ker}}
\newcommand{\Cov}{\mathrm{Cov}}
\newcommand{\Prob}{\mathbb{P}}
\newcommand{\btheta}{\boldsymbol{\theta}}
\newcommand{\bx}{\mathbf{x}}
\newcommand{\bw}{\mathbf{w}}

\DeclarePairedDelimiterX\innerp[2]{(}{)}{#1\delimsize\vert\mathopen{}#2}
\DeclareMathOperator*{\argmax}{argmax}
\DeclareMathOperator*{\argmin}{argmin}
\DeclarePairedDelimiter{\ceil}{\lceil}{\rceil}


\setlist{topsep=1ex,parsep=1ex,itemsep=0ex}
\setlist[1]{leftmargin=\parindent}
\setlist[enumerate,1]{label=\arabic*.,ref=\arabic*}
\setlist[enumerate,2]{label=(\alph*),ref=(\alph*)}

% For embedding images
\graphicspath{ {./images/} }

% Specifically for paper formatting 
\renewcommand{\baselinestretch}{1.2} % Spaces manuscript for easy reading

% Formatting definitions, propositions, etc. 
\newtheorem{definition}{Definition}
\newtheorem{prop}{Proposition}
\newtheorem{lemma}{Lemma}
\newtheorem{thm}{Theorem}
\newtheorem{corollary}{Corollary}

% Title and author
\title{Linear Dynamical Systems and Ordinary Differential Equations}
\author{Andrew Roberts}

\begin{document}

\maketitle
\tableofcontents
\newpage

% Notation 
\section{Linear ODEs}
Linear ODEs are generally of the form 
\begin{align*}
a_0(t)x + a_1(t) \dot{x} + a_2(t) \ddot{x} + \dots + a_n(t) x^{(n)} = b(t)
\end{align*}
This equation is said to be of order $n$ due to the fact that $n$ derivatives of the state vector $x(t) \in \R^P$ appear in the equation. 
The equation is linear in the state vector $x$ and its $n$ derivatives, $\{x, \dot{x}, \ddot{x}, \dots, x^{(n)}\}$. The coefficients 
$a_1(t), \dots, a_n(t)$ may be non-linear functions of $t$. In the case when the coefficients are simply constants 
\begin{align*}
a_0 x + a_1 \dot{x} + a_2 \ddot{x} + \dots + a_n x^{(n)} = b(t)
\end{align*}
the equation is known as an $n^{\text{th}}$ order linear ODE with constant coefficients. The emphasis here is not on finding analytical solutions 
to such systems, but rather understanding the underlying structure. In particular, I will try to emphasize the application of linear algebra in studying 
such systems far more than I emphasize arguments based on calculus. In this spirit, I start by exploring the general structure of the solutions to 
$n^{\text{th}}$ order linear ODEs by taking a linear operator viewpoint. 

\subsection{Space of solutions}


\end{document}


