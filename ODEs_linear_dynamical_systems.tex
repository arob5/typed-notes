\documentclass[12pt]{article}
\RequirePackage[l2tabu, orthodox]{nag}
\usepackage[main=english]{babel}
\usepackage[rm={lining,tabular},sf={lining,tabular},tt={lining,tabular,monowidth}]{cfr-lm}
\usepackage{amsthm,amssymb,latexsym,gensymb,mathtools,mathrsfs}
\usepackage[T1]{fontenc}
\usepackage[utf8]{inputenc}
\usepackage[pdftex]{graphicx}
\usepackage{epstopdf,enumitem,microtype,dcolumn,booktabs,hyperref,url,fancyhdr}
\usepackage{algorithmic}
\usepackage[ruled,vlined,commentsnumbered,titlenotnumbered]{algorithm2e}
\usepackage{bbm}

% Plotting
\usepackage{pgfplots}
\usepackage{xinttools} % for the \xintFor***
\usepgfplotslibrary{fillbetween}
\pgfplotsset{compat=1.8}
\usepackage{tikz}

% Custom Commands
\newcommand*{\norm}[1]{\left\lVert#1\right\rVert}
\newcommand*{\abs}[1]{\left\lvert#1\right\rvert}
\newcommand*{\suchthat}{\,\mathrel{\big|}\,}
\newcommand{\E}{\mathbb{E}}
\newcommand{\Var}{\mathrm{Var}}
\newcommand{\R}{\mathbb{R}}
\newcommand{\N}{\mathcal{N}}
\newcommand{\Ker}{\mathrm{Ker}}
\newcommand{\Cov}{\mathrm{Cov}}
\newcommand{\Prob}{\mathbb{P}}
\newcommand{\btheta}{\boldsymbol{\theta}}
\newcommand{\bx}{\mathbf{x}}
\newcommand{\bb}{\mathbf{b}}
\newcommand{\bw}{\mathbf{w}}

\DeclarePairedDelimiterX\innerp[2]{(}{)}{#1\delimsize\vert\mathopen{}#2}
\DeclareMathOperator*{\argmax}{argmax}
\DeclareMathOperator*{\argmin}{argmin}
\DeclarePairedDelimiter{\ceil}{\lceil}{\rceil}


\setlist{topsep=1ex,parsep=1ex,itemsep=0ex}
\setlist[1]{leftmargin=\parindent}
\setlist[enumerate,1]{label=\arabic*.,ref=\arabic*}
\setlist[enumerate,2]{label=(\alph*),ref=(\alph*)}

% For embedding images
\graphicspath{ {./images/} }

% Specifically for paper formatting 
\renewcommand{\baselinestretch}{1.2} % Spaces manuscript for easy reading

% Formatting definitions, propositions, etc. 
\newtheorem{definition}{Definition}
\newtheorem{prop}{Proposition}
\newtheorem{lemma}{Lemma}
\newtheorem{thm}{Theorem}
\newtheorem{corollary}{Corollary}

% Title and author
\title{Linear Dynamical Systems and Ordinary Differential Equations}
\author{Andrew Roberts}

\begin{document}

\maketitle
\tableofcontents
\newpage

% Notation 
\section{Linear ODEs}
Linear ODEs are generally of the form 
\begin{align}
a_0(t)x + a_1(t) \dot{x} + a_2(t) \ddot{x} + \dots + a_n(t) x^{(n)} = b(t) \label{linear_ODE}
\end{align}
This equation is said to be of order $n$ due to the fact that $n$ derivatives of the state vector $x(t) \in \R^P$ appear in the equation. 
The equation is linear in the state vector $x$ and its $n$ derivatives, $\{x, \dot{x}, \ddot{x}, \dots, x^{(n)}\}$. The coefficients 
$a_1(t), \dots, a_n(t)$ may be non-linear functions of $t$. In the case when the coefficients are simply constants 
\begin{align*}
a_0 x + a_1 \dot{x} + a_2 \ddot{x} + \dots + a_n x^{(n)} = b(t)
\end{align*}
the equation is known as an $n^{\text{th}}$ order linear ODE with constant coefficients. The emphasis here is not on finding analytical solutions 
to such systems, but rather understanding the underlying structure. In particular, I will try to emphasize the application of linear algebra in studying 
such systems far more than I emphasize arguments based on calculus. In this spirit, I start by exploring the general structure of the solutions to 
$n^{\text{th}}$ order linear ODEs by taking a linear operator viewpoint. 

\subsection{Linear Operator Viewpoint and the Solution Space}
We can view the left side of \ref{linear_ODE} as a linear operator that maps between function spaces. In particular, let $\mathcal{L}$ be the operator 
that maps $x(t)$ to $a_0(t)x + a_1(t) \dot{x} + a_2(t) \ddot{x} + \dots + a_n(t) x^{(n)}$. The equation \ref{linear_ODE} can now be written 
\begin{align}
Lx &= b \label{linear_ODE_operator_form}
\end{align}
where I note that $x = x(t)$ and $b = b(t)$ are both functions of $t$, though this is suppressed in the notation to emphasize the fact that we are viewing $x$ and 
$b$ abstractly as vectors. Note that $L$ is indeed a linear operator (i.e. it satisfies $L(\alpha x_1 + \beta x_2) = \alpha L x_1 + \beta L x_2$) due to the fact that differentiation 
is a linear operation. By writing the linear ODE as \ref{linear_ODE_operator_form}, we see that solving linear ODEs can be viewed as solving a linear equation, though the unknown 
$x$ in the equation is infinite-dimensional. 

With this powerful viewpoint, we can immediately establish what solutions of \ref{linear_ODE} must look like. First, suppose we have already obtained a \textit{particular solution}
$x_p$ such that $L x_p = b$. My claim is that the space of solutions of \label{linear_ODE} can be written as $x_p + \text{Null}(L)$, where $\text{Null}(L)$ 
is the null space of the linear operator $L$. If $x_n \in \text{Null}(L)$ then $L x_n = 0$ (where the righthand side is the zero function) so 
\[L(x_p + x_n) = L x_p + L x_n = b + 0 = b\]
so indeed $x_p + x_n$ is a solution. To show that all solutions are of this form, suppose that $x$ is some other solution such that $Lx = b$. Then we can write $x$ as 
$x = x_p + (x - x_p)$ where $L(x - x_p) = Lx - L x_p = b - b = 0$ so $x - x_p \in \text{Null}(L)$. This establishes that all solutions are of the form $x_p + \text{Null}(L)$. 

\subsection{Finding the Particular Solution}

\subsection{Reduction to a First-Order System}
I now establish another connection between linear ODEs \ref{linear_ODE} and linear systems. In particular, I show that \ref{linear_ODE} can be written as a first order linear 
equation 
\[\dot{\bx} = A \bb \]
by extending the state-space $x$ to a vector of states $\bx$. For clarity of notation, we consider a third-order system,
\[x + a_1(t) \dot{x} + a_2(t) \ddot{x} + a_3(t) \dddot{x} = b(t)\]
but the same exact reasoning applies to equations of general order. To keep the algebra clean we have also assumed $a_0(t) = 1$, which can be achieved by dividing both 
sides of the equation by the leading coefficient. I define the extended state vector 
\begin{align*}
\bx :=  \begin{pmatrix} x_1 \\ x_2 \\ x_3 \end{pmatrix} := \begin{pmatrix} x \\ \dot{x} \\ \ddot{x} \end{pmatrix}
\end{align*}
The dimension of $\bx$ in general matches the order of the ODE, as $\bx$ includes the original state $x$ plus all of its derivatives except for the last one $x^{(n)}$. 

\end{document}


