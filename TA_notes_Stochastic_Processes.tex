\documentclass[12pt]{article}
\RequirePackage[l2tabu, orthodox]{nag}
\usepackage[main=english]{babel}
\usepackage[rm={lining,tabular},sf={lining,tabular},tt={lining,tabular,monowidth}]{cfr-lm}
\usepackage{amsthm,amssymb,latexsym,gensymb,mathtools,mathrsfs}
\usepackage[T1]{fontenc}
\usepackage[utf8]{inputenc}
\usepackage[pdftex]{graphicx}
\usepackage{epstopdf,enumitem,microtype,dcolumn,booktabs,hyperref,url,fancyhdr}
\usepackage{algorithmic}
\usepackage[ruled,vlined,commentsnumbered,titlenotnumbered]{algorithm2e}
\usepackage{bbm}

% Plotting
\usepackage{pgfplots}
\usepackage{xinttools} % for the \xintFor***
\usepgfplotslibrary{fillbetween}
\pgfplotsset{compat=1.8}
\usepackage{tikz}

% Custom Commands
\newcommand*{\norm}[1]{\left\lVert#1\right\rVert}
\newcommand*{\abs}[1]{\left\lvert#1\right\rvert}
\newcommand*{\suchthat}{\,\mathrel{\big|}\,}
\newcommand{\E}{\mathbb{E}}
\newcommand{\Var}{\mathrm{Var}}
\newcommand{\R}{\mathbb{R}}
\newcommand{\N}{\mathcal{N}}
\newcommand{\Ker}{\mathrm{Ker}}
\newcommand{\Cov}{\mathrm{Cov}}
\newcommand{\Prob}{\mathbb{P}}
\DeclarePairedDelimiterX\innerp[2]{(}{)}{#1\delimsize\vert\mathopen{}#2}
\DeclareMathOperator*{\argmax}{argmax}
\DeclareMathOperator*{\argmin}{argmin}
\DeclarePairedDelimiter{\ceil}{\lceil}{\rceil}

\setlist{topsep=1ex,parsep=1ex,itemsep=0ex}
\setlist[1]{leftmargin=\parindent}
\setlist[enumerate,1]{label=\arabic*.,ref=\arabic*}
\setlist[enumerate,2]{label=(\alph*),ref=(\alph*)}

% For embedding images
\graphicspath{ {./images/} }

% Specifically for paper formatting 
\renewcommand{\baselinestretch}{1.2} % Spaces manuscript for easy reading

% Formatting definitions, propositions, etc. 
\newtheorem{definition}{Definition}
\newtheorem{prop}{Proposition}
\newtheorem{lemma}{Lemma}
\newtheorem{thm}{Theorem}
\newtheorem{corollary}{Corollary}

% Title and author
\title{TA Notes}
\author{Andrew Roberts}

\begin{document}

\maketitle
\tableofcontents
\newpage

\section{Introduction to Stochastic Processes}
\begin{itemize}
\item Discuss the two perspectives on stochastic processes; random function vs. collection of random variables. 
\end{itemize}

\section{Introduction to Markov Chains}

\subsection{Finite State Space}
\textbf{TODO}
\begin{align}
\Prob(X_k = j) = \sum_{i = 1}^{I} \Prob(X_k = j|X_{k - 1} = i)\Prob(X_{k - 1} = i)
\end{align}

Define $P(i, j) = \Prob(X_{k+1} = j|X_k = i)$ and $\mu^{(k)}_j = \Prob(X_k = j)$. This can then be re-written as 
\begin{align}
\mu^{(k)}_j = \sum_{i = 1}^{I} P(i, j) \mu^{(k-1)}_i \label{discrete_transition}
\end{align}

\subsection{General State Space}
If the transition dynamics can be described by probability densities $p(x, \cdot)$ for each $x \in \mathcal{X}$ (where $p(x, y)$ is the density of transitioning from $x$ to $y$) then the 
density $\mu_k$ describing the distribution at time $k$ is given by applying the law of total probability:
\begin{align}
\mu_k(y) = \int p(x, y) \mu_{k-1}(x) dx \label{continuous_transition}
\end{align}
This is the continuous analog of \ref{discrete_transition}. Having to write separate formulas for the discrete and continuous cases is annoying and not amenable to developing a general 
theory. Note that there could also be Markov chains with transitions that are not purely discrete nor continuous, but rather a mixture of the two. The concept of a probability measure allows 
us to write down a general expression that encompasses both \ref{discrete_transition} and \ref{continuous_transition}. 
\begin{align}
\mu_k(A) = \mu_{k - 1}P(A) := \int P(x, A)\mu_{k - 1}(dx) \label{general_transition}
\end{align}
The notation $\mu_{k - 1}P$ is very reminiscent of the left-multiplication of a row vector with the transition matrix in the finite state space case. Matrix multiplication (a linear operation) has 
been replaced with integration (another linear operation), but the basic intuition is the same. 

To be absolutely clear here, let's establish the connection between the general expression \ref{general_transition} and the continuous case densities exist. Recalling that integrating densities over 
a set yields the probability of the set, we have 
\[\Prob(X_k \in A) = \mu_k(A) = \int_A p_k(y) dy\]
The same exact logic applies to the transition probability $P(x, A)$:
\[\Prob(X_k \in A|X_{k - 1} = x) = P(x, A) = \int_A p(x, y) dy \]
Putting these together, \ref{general_transition} becomes,
\[\Prob(X_k \in A) = \int \left[ \int_A p(x, y) dy \right] d\mu_{k-1}(x)\]
Finally we note when densities exist, integrating a function $f(x)$ with respect to the measure $\mu_{k-1}$ can be realized by integrating with respect to the density $p_{k-1}$:
\[\int f(x) \mu_{k-1}(dx) = \int f(x) p_{k-1}(x) dx\]
In our current setting, the function we are integrating is $f(x) = \int_A p(x, y) dy$. Thus, plugging this in we obtain 
\[\Prob(X_k \in A) = \int \left[ \int_A p(x, y) dy \right] d\mu_{k-1}(x) = \int \left[ \int_A p(x, y) dy \right] p_{k-1}(x) dx\]
Note that the outer integral is over the entire state space $\mathcal{X}$. Perhaps a more insightful way to write this is 
\[\Prob(X_k \in A)  = \int_A \int_{\mathcal{X}} p(x, y) p_{k-1}(x) dx dy\]


The concept of an invariant (i.e. stationary) distribution also has a natural generalization. A 
distribution $\mu$ is invariant if $\mu P = \mu$; that is, if the distribution is unchanged one time step in the future (and hence unchanged for all future times). By plugging in 
\ref{general_transition}, we see that $\mu P = \mu$ is really shorthand for 
\begin{align*}
\mu(A) = \int P(x, A) \mu(dx) \text{ for all } A \in \mathcal{F}
\end{align*}
Note that the ``for all $A \in \mathcal{F}$'' is essential; for the two distributions $\mu$ and $\mu P$ to be equal they must assign the same probability to every set $A$. 

\subsection{Product Distribution, Detailed Balance, Reversibility}
We will find it useful to talk about the joint distribution between current and subsequent states, $X_{k-1}$ and $X_k$. In other words, we want to answer questions like: 
what is the probability that $X_{k -1}$ is in $A$ and $X_k$ is in $B$? Realizing this event requires 1.) already being in $A$ at the $k^{\text{th}}$ time step and, 2.) transitioning 
from $A$ to $B$ in the next time step. Let's consider the case where densities exist for a moment. In this case, from introductory probability we know that 
the probability that $X_{k-1}$ is in $A$ and $X_k$ is in $B$ can be found by integrating the \textit{joint density} $p_{X_k, X_{k-1}}(x, y)$ over the set $A \times B$. 
\begin{align*}
\Prob(X_{k-1} \in A, X_k \in B) &= \int_{A \times B} p_{X_k, X_{k-1}}(x, y) dx dy =  \int_B \int _A p_{X_k, X_{k-1}}(x, y) dx dy 
\end{align*}
In this case, the joint density factors as 
\[p_{X_k, X_{k-1}}(x, y) = p(x, y)p_{k-1}(x)\]
which makes perfect sense since $p_{k-1}(x)$ captures starting in $A$ and $p(x, y)$ captures the transition from $A$ to $B$. 

\subsubsection{A note on exercise 3.4}
The result that is proved as exercise 3.4,
\begin{align}
(\mu P_1)P_2 = \mu P_1 P_2 \label{associativity}
\end{align}
is not very surprising. In fact, we specifically defined $\mu P$ and $P_1 P_2$ so that this would work. If we defined these notions and found that associativity did not hold, then 
we would definitely want to reconsider those definitions/notation. It is helpful to think about the finite state space case, in which case $\mu$ is a row vector and $P_1$, $P_2$ are 
matrices. The statement \ref{associativity} is identical to the finite state space analog from a notational standpoint, so what we have done is cleverly define things in the general state 
space setting so that the same notation applies here (i.e. we have overloaded the notation). 


\subsection{Questions}
\begin{itemize}
\item Should the probability in example 3.4 be divided by $\Prob(X_{k-1} = x_{k-1})$? Or rather the density $f_{X_{k-1}}(x_{k-1})$? 
\end{itemize}


% Introduction to MCMC
\section{Introduction to Markov Chain Monte Carlo}

\subsection{Sampling from Probability Distributions: Motivation}
\begin{itemize}
\item Why important?
\item Bayesian statistics (e.g. linear regression)
\item The Ising model 
\item Approximating integrals. A bunch of different ways to write expectations: 
\[\pi(\phi) = \E_{X \sim \pi}\left[\phi(X)\right] = \E_{\pi}\left[\phi(X) \right] = \int \phi(x) \pi(dx) \]
\item Emphasis on the fact that the normalizing constant is unknown; intuitively we should not need the normalizing constant for sampling. 
\end{itemize}

\subsection{Simple Monte Carlo}
\begin{itemize}
\item See Youssef's notes 
\item Why not just use deterministic numerical integration? 
\item Apparent immunity to the curse of dimensionality; and where the curse of dimensionality comes in. 
\item Rejection sampling. 
\end{itemize}

\subsection{Markov Chain Monte Carlo}
\subsubsection{The Big Idea}
Discuss the departure from independent sampling, the idea of using a Markov chain, and invariant/stationary distribution. Introduce notation that $h_\pi$ is the density of $\pi$. 

\subsubsection{Metropolis-Hastings}
Metropolis-Hastings (MH) is a class of MCMC algorithms, and is far and away the most common MCMC method. Recall that designing an MCMC algorithm comes down to 
specifying the structure of the Markov chain used in the algorithm. Since homogenous Markov chains are fully defined by an initial condition and a transition kernel $P$, then 
essentially the only design freedom is in the choice of $P$. The class of MH algorithms is thus defined by choosing a specific form for $P$. In particular, the MH transition operator
is composed of two steps. Suppose the current state is $X_k$. The MH Markov chain determines the next state $X_{k + 1}$ by 
\begin{enumerate}
\item \textit{Proposing} a new state $X_{k + 1}^\prime$ by sampling from a specified \textit{proposal distribution} $Q(X_k, \cdot)$; i.e.,  $X_{k + 1} \sim Q(X_k, \cdot)$. 
\item Either accepting or rejecting the proposal (based on a criterion discussed below). The proposal is accepted with probability $\alpha(X_k, X_{k + 1})$, in which case the 
new state is set to the proposed state $X_{k + 1} := X_{k+1}^\prime$. If rejected, the new state is set to the current state $X_{k + 1} := X_k$. 
\end{enumerate}
Recall that we require the target distribution $\pi$ to be a stationary distribution of the Markov chain. Therefore, the above procedure will only be useful if this is true. It may seem like there 
are now two main design choices in MH algorithms: 1.) the proposal distribution $Q$, and 2.) the criterion for accepting or rejecting the samples. However, the accept-reject criterion is chosen 
precisely so that $\pi$ is a stationary distribution, no matter what proposal distribution $Q$ is chosen. Therefore, at a high level the only design freedom in a MH algorithm is in the proposal distribution. 
Therefore, let us first discuss what the accept-reject criterion \textit{has to be}, given the constraint that $\pi$ must be a stationary distribution of the chain. Recall that verifying stationarity directly is 
a difficult task, but we proved that \textit{detailed balance} (i.e. reversibility)--which is conveniently much easier to check--implies stationarity. Therefore, we will use this sufficient condition to derive the 
MH accept-reject criterion.  

Suppose the proposal distribution $Q(x, \cdot)$ has density $q(x, \cdot)$. Recall that $h_\pi$ is the density of the target distribution $\pi$. We seek to determine the acceptance probability $\alpha(X_k, X_{k + 1})$. 

\subsubsection{Gibbs Sampling}

\subsubsection{Roadmap for MCMC Theory}
\begin{enumerate}
\item \textbf{Does the chain have $\pi$ as a stationary distribution?} \\
We have already answered this question above. For MH algorithms, this holds by definition. 

\item \textbf{Does the chain converge to $\pi$? Does it converge for all starting points/initial conditions?} \\
In other words, does the distribution of $X_k$ approach $\pi$ as $k$ grows large? In what sense does it converge? 

\item \textbf{Can we use samples from the chain to estimate expectations?} \\
In other words, does the Monte Carlo estimate $\hat{\pi}(\phi) = \sum_{\ell = 1}^{k} \phi(X_\ell)$ converge to $\pi(\phi) = \int \phi(x) \pi(dx)$? If so, in what sense does it converge? 

\bigskip
\textit{These three questions represent the bare minimum in addressing our goals. The whole motivation for utilizing MCMC was to develop a scalable scheme to sample from a target distribution 
and use those samples to estimate expectations with respect to the target distribution. Thus, if it turned out that suitable convergence results failed to hold, then MCMC wouldn't really be useful 
at all. However, once we have established these basic results, there remain many important unanswered questions. The above three questions make no mention of the rate of convergence. In practice, 
we of course can only run the chain for a finite number of steps, so it is vital to ensure that the chain converges fast enough to be practically useful
They also fail to give a sense of the influence of the starting distribution. In practice, we typically only simulate a Markov chain from a single (or maybe a few) different initial conditions, so this 
question is also of great practical importance. The below items offer more ``quantitative'' viewpoints that seek to address these questions.}

\item \textbf{Does the chain converge at a uniform geometric rate?} \\
This property is known as \textbf{uniform ergodicity} and means there exist constants $M$ and $\rho > 1$ such that 
\[\norm{P^k(x, \cdot) - \pi(\cdot)}_{\text{TV}} \leq M\rho^{-k}\]
This says that each iteration $k$ gets the distribution a factor of $\rho$ closer to the desired distribution $\pi$. This is a very strong notion of convergence, and is very difficult to guarantee. The reason it 
is such a strong notion is due to the fact that $M$ is not a function of the starting location $x$ (hence, ``uniform''); i.e. this says that the same rate of convergence holds no matter where you start. Another thing to note is that 
in practice we typically do not know $M$. Thus, this is more of an asymptotic result in that it establishes a rate of convergence, but doesn't give any sort of finite-time guarantees. In other words, it does 
not answer questions like: after $k$ time steps how far is the chain from the target distribution? 

\item \textbf{Does the chain converge at a (perhaps not uniform) geometric rate?} \\
\textbf{Geometric ergodicity} relaxes the notion of ergodicity above by allowing $M$ to depend on the initial condition $x$. 
\[\norm{P^k(x, \cdot) - \pi(\cdot)}_{\text{TV}} \leq M(x)\rho^{-k}\]
Thus, geometric ergodicity allows the rate of convergence to depend on the starting location. As opposed to uniform ergodicity, geometric ergodicity is a much easier condition to satisfy, and indeed 
a condition that almost all reasonable MCMC schemes should satisfy. 

\item \textbf{Can we compute a finite-time error bound?} \\
The theoretical ideal would be having the ability to answer the question posed earlier: after $k$ time steps how far is the chain from the target distribution? In other words, having a explicit bound on the error 
as a function of $k$. This very powerful notion is referred to as \textbf{quantitative convergence} and can be written as 
\[\norm{P^k(x, \cdot) - \pi(\cdot)}_{\text{TV}} \leq f(x, k)\]
where $f$ is some \textit{known} function of the initial condition $x$ and the time step $k$. 


\end{enumerate}


\subsubsection{To include:}
\begin{itemize}
\item The departure from independent sampling. The cost of using correlated samples. 
\item MCMC tool that Youssef shared. 
\end{itemize}

\subsection{Resources}
\begin{itemize}
\item MCMC tool that Youssef shared. 
\end{itemize}

% Ergodic theory for Markov Chains 
\section{Ergodic Theory for Markov Chains}




% SDEs
\section{Continuous Time}
\begin{itemize}
\item Motivate as follows: we can move from non-random discrete dynamical systems (e.g. difference equations) to things like Markov Chains by introducing randomness. We can do the 
same to move from continuous-time dynamical systems (e.g. ODEs) to continuous-time stochastic dynamical systems (SDEs) 
\item Include discussion of random vs stochastic DEs (see Stuart)
\end{itemize}






\end{document}




