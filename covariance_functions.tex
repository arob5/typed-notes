\documentclass[12pt]{article}
\RequirePackage[l2tabu, orthodox]{nag}
\usepackage[main=english]{babel}
\usepackage[rm={lining,tabular},sf={lining,tabular},tt={lining,tabular,monowidth}]{cfr-lm}
\usepackage{amsthm,amssymb,latexsym,gensymb,mathtools,mathrsfs}
\usepackage[T1]{fontenc}
\usepackage[utf8]{inputenc}
\usepackage[pdftex]{graphicx}
\usepackage{epstopdf,enumitem,microtype,dcolumn,booktabs,hyperref,url,fancyhdr}
\usepackage{algorithmic}
\usepackage[ruled,vlined,commentsnumbered,titlenotnumbered]{algorithm2e}

% Plotting
\usepackage{pgfplots}
\usepackage{xinttools} % for the \xintFor***
\usepgfplotslibrary{fillbetween}
\pgfplotsset{compat=1.8}
\usepackage{tikz}

% Custom Commands
\newcommand*{\norm}[1]{\left\lVert#1\right\rVert}
\newcommand*{\abs}[1]{\left\lvert#1\right\rvert}
\newcommand*{\suchthat}{\,\mathrel{\big|}\,}
\newcommand{\E}{\mathbb{E}}
\newcommand{\Var}{\mathrm{Var}}
\newcommand{\R}{\mathbb{R}}
\newcommand{\N}{\mathcal{N}}
\newcommand{\Cov}{\mathrm{Cov}}
\newcommand{\Prob}{\mathbb{P}}
\DeclarePairedDelimiterX\innerp[2]{(}{)}{#1\delimsize\vert\mathopen{}#2}
\DeclareMathOperator*{\argmax}{argmax}
\DeclareMathOperator*{\argmin}{argmin}
\DeclarePairedDelimiter{\ceil}{\lceil}{\rceil}

% Covariance function/kernel. 
\newcommand{\Ker}{k}
\newcommand{\locSpace}{\mathcal{X}}
\newcommand{\loc}{\mathbf{x}}
\newcommand{\locTwo}{\loc^\prime}
\newcommand{\locDum}{\mathbf{u}}
\newcommand{\inputDim}{D}
\newcommand{\rf}{Y}

\setlist{topsep=1ex,parsep=1ex,itemsep=0ex}
\setlist[1]{leftmargin=\parindent}
\setlist[enumerate,1]{label=\arabic*.,ref=\arabic*}
\setlist[enumerate,2]{label=(\alph*),ref=(\alph*)}

% For embedding images
\graphicspath{ {./images/} }

% Specifically for paper formatting 
\renewcommand{\baselinestretch}{1.2} % Spaces manuscript for easy reading

% Formatting definitions, propositions, etc. 
\newtheorem{definition}{Definition}
\newtheorem{prop}{Proposition}
\newtheorem{lemma}{Lemma}
\newtheorem{thm}{Theorem}
\newtheorem{corollary}{Corollary}

% Title and author
\title{Covariance Functions: Stationary and Non-Stationary}
\author{Andrew Roberts}

\begin{document}

\maketitle
\tableofcontents
\newpage

% Spectral Representation 
\section{Spectral Representation}

% Process Convolutions
\section{Process Convolutions}
The idea of \textit{process convolutions} is a popular method in spatial statistics used to constructively generates a covariance function. The main idea is that a random field $\rf(\cdot)$ over an input space 
$\locSpace$ can be defined via a convolution 
\begin{align}
\rf(\loc) &= \int_{\locSpace} \Ker_0(\loc - \locDum) dW(\locDum), \label{process_convolution}
\end{align}
where $\Ker_0(\cdot)$ is a (stationary) kernel and $W(\cdot)$ a stochastic process over $\locSpace$ \cite{Risser}. So the idea is to start with a ``simple'' stochastic process $W(\cdot)$ (often a white-noise process) and produce a 
more complicated process $\rf(\cdot)$ by convolving it with a kernel $\Ker_0(\cdot)$. When $W(\cdot)$ is a Brownian motion or other form of GP, then the resulting stochastic process $W(\cdot)$ in \ref{process_convolution} is a 
GP. We can obtain a simpler equivalent representation of \ref{process_convolution} in the Brownian motion case via
\begin{align}
\rf(\loc) &= \int_{\locSpace} \Ker_0(\loc - \locDum) V(\locDum) d\locDum, \label{process_convolution2}
\end{align}
where $V(\cdot)$ is a Gaussian white-noise process. The covariance function of the process $\rf(\cdot)$ implied by \ref{process_convolution2} is 
\begin{align}
\Ker(\loc, \locTwo) &= \int_{\locSpace} \Ker_0(\loc - \locDum)\Ker_0(\locTwo - \locDum) d\locDum \label{process_convolution_kernel}
\end{align}
Instead of working through all of the technical details regarding the stochastic integration \ref{process_convolution}, we instead opt for the simpler approach of taking \ref{process_convolution_kernel}
as our starting point. Therefore, we need only verify that $\Ker(\cdot, \cdot)$ is PSD in order to establish that it is a valid covariance function (Paciorek). 

\begin{prop} 
Let $\Ker_0: \locSpace \to \R_+$ be a stationary kernel. Then the function $\Ker: \locSpace \times \locSpace \to \R_+$ defined in \ref{process_convolution_kernel} is symmetric PSD. 
\end{prop}

\begin{proof} 
TODO
\end{proof} 


% Bibliography
\begin{thebibliography}{20}
\bibitem{Risser} Risser, Mark. (2016). Review: Nonstationary Spatial Modeling, with Emphasis on Process Convolution and Covariate-Driven Approaches. 
\bibitem{Guinness} Joseph Guinness, Montserrat Fuentes (2016). Isotropic covariance functions on spheres: Some properties and modeling considerations, Journal of Multivariate Analysis, Volume 143
Pages 143-152, ISSN 0047-259X.
\bibitem{Porcu} Porcu, Emilio \& Bevilacqua, Moreno \& Schaback, Robert \& Oates, Chris. (2023). The Mat\'ern Model: A Journey through Statistics, Numerical Analysis and Machine Learning. 10.48550/arXiv.2303.02759. 
\bibitem{Higdon} Higdon, David M.. “Space and Space-Time Modeling using Process Convolutions.” (2002).
\bibitem{Higdon2} Higdon, David M. et al. “Non-Stationary Spatial Modeling.” (2022).
\bibitem{Paciorek} Paciorek CJ, Schervish MJ. Spatial Modelling Using a New Class of Nonstationary Covariance Functions. Environmetrics. 2006;17(5):483-506. doi: 10.1002/env.785. PMID: 18163157; PMCID: PMC2157553.
\end{thebibliography}

\end{document}


