\documentclass[12pt]{article}
\RequirePackage[l2tabu, orthodox]{nag}
\usepackage[main=english]{babel}
\usepackage[rm={lining,tabular},sf={lining,tabular},tt={lining,tabular,monowidth}]{cfr-lm}
\usepackage{amsthm,amssymb,latexsym,gensymb,mathtools,mathrsfs}
\usepackage[T1]{fontenc}
\usepackage[utf8]{inputenc}
\usepackage[pdftex]{graphicx}
\usepackage{caption}
\usepackage{subcaption}
\usepackage{epstopdf,enumitem,microtype,dcolumn,booktabs,hyperref,url,fancyhdr}

% Plotting
\usepackage{pgfplots}
\usepackage{xinttools} % for the \xintFor***
\usepgfplotslibrary{fillbetween}
\pgfplotsset{compat=1.8}
\usepackage{tikz}

% Custom Commands
\newcommand*{\norm}[1]{\left\lVert#1\right\rVert}
\newcommand*{\abs}[1]{\left\lvert#1\right\rvert}
\newcommand*{\suchthat}{\,\mathrel{\big|}\,}
\newcommand{\E}{\mathbb{E}}
\newcommand{\Var}{\mathrm{Var}}
\newcommand{\R}{\mathcal{R}}
\newcommand{\N}{\mathcal{N}}
\newcommand{\Ker}{\mathrm{Ker}}
\newcommand{\Cov}{\mathrm{Cov}}
\newcommand{\Prob}{\mathbb{P}}
\DeclarePairedDelimiterX\innerp[2]{(}{)}{#1\delimsize\vert\mathopen{}#2}
\DeclareMathOperator*{\argmax}{argmax}
\DeclareMathOperator*{\argmin}{argmin}
\def\R{\mathbb{R}}
\DeclarePairedDelimiter\ceil{\lceil}{\rceil}
\DeclarePairedDelimiter\floor{\lfloor}{\rfloor}

\setlist{topsep=1ex,parsep=1ex,itemsep=0ex}
\setlist[1]{leftmargin=\parindent}
\setlist[enumerate,1]{label=\arabic*.,ref=\arabic*}
\setlist[enumerate,2]{label=(\alph*),ref=(\alph*)}

% Specifically for paper formatting 
\renewcommand{\baselinestretch}{1.2} % Spaces manuscript for easy reading

% Formatting definitions, propositions, etc. 
\newtheorem{definition}{Definition}
\newtheorem{prop}{Proposition}
\newtheorem{lemma}{Lemma}
\newtheorem{thm}{Theorem}
\newtheorem{corollary}{Corollary}

\begin{document}

\begin{center}
\Large
Notes on Markov Chains, Stochastic Stability, and MCMC
\end{center}

\begin{flushright}
Andrew Roberts
\end{flushright} 

% Section: Beyond Independent Sampling
\section{Beyond Independent Sampling}
Discuss some of the shortfalls of Accept/Reject and other independent sampling methods here. 

% Section: Markov Chains
\section{Markov Chains}
\subsection{Defining Markov Chains}
\subsubsection{Measure Theoretic Setup}
Let $(\Omega, \mathcal{A}, \Prob)$ be a measure space. Although we will rarely mention this space going forward, it it important to remember that it is always
underlying all of the theory we develop. Now, consider a measurable space $(\mathcal{X}, \mathcal{F})$ which will call a \textit{state space}. $\mathcal{X}$ is a set of states that the Markov 
Chain can assume, and $\mathcal{F}$ contains all of the subsets of the state space to which we will be able to assign probabilities. We will typically restrict ourselves to 
$(\mathcal{X}, \mathcal{F}) = (\R^n, \mathcal{B})$.
We consider a discrete-time stochastic process $\mathbb{X} = \{X_k\}_{k = 1}^{\infty}$ defined on the state space $(\mathcal{X}, \mathcal{F})$. That is, each $X_k$ is a random variable
$X: \Omega \to \mathcal{X}$. So for $A \in \mathcal{F}$, the event $\{X_k \in A\}$ means ``at time step k, the state of the chain is somewhere in A''. Since the system is random, we 
can only talk about the probability of this event occurring, and the probability measure $\Prob$ assigns such a probability. It is important to emphasis that all of the $X_k$ are defined on 
the \textit{same} underlying probability space; that is, the source of randomness is the same for the system as a whole.

\subsubsection{The Markov Property}
\textbf{TODO}

\subsubsection{Homogenous Chains}
\textbf{TODO}

\subsubsection{Transition Kernels}
Under the assumption of homogeneity, a Markov chain is completely defined by its initial distribution $\nu_0 = \mathcal{L}(X_0)$ (this notation means the distribution or 
\textit{law} of $X_0$)

\subsection{Stationary Distributions}
The basic idea of MCMC is to construct a Markov Chain with stationary distribution equal to the target distribution, and then sample using this chain. 
Thus, it is first necessary to state some basic results regarding stationary distributions of Markov chains. 





\end{document}

