\documentclass[12pt]{article}
\RequirePackage[l2tabu, orthodox]{nag}
\usepackage[main=english]{babel}
\usepackage[rm={lining,tabular},sf={lining,tabular},tt={lining,tabular,monowidth}]{cfr-lm}
\usepackage{amsthm,amssymb,latexsym,gensymb,mathtools,mathrsfs}
\usepackage[T1]{fontenc}
\usepackage[utf8]{inputenc}
\usepackage[pdftex]{graphicx}
\usepackage{epstopdf,enumitem,microtype,dcolumn,booktabs,hyperref,url,fancyhdr}
\usepackage{algorithmic}
\usepackage[ruled,vlined,commentsnumbered,titlenotnumbered]{algorithm2e}
\usepackage{bbm}

% Plotting
\usepackage{pgfplots}
\usepackage{xinttools} % for the \xintFor***
\usepgfplotslibrary{fillbetween}
\pgfplotsset{compat=1.8}
\usepackage{tikz}

% Custom Commands
\newcommand*{\norm}[1]{\left\lVert#1\right\rVert}
\newcommand*{\abs}[1]{\left\lvert#1\right\rvert}
\newcommand*{\suchthat}{\,\mathrel{\big|}\,}
\newcommand{\E}{\mathbb{E}}
\newcommand{\Var}{\mathrm{Var}}
\newcommand{\R}{\mathbb{R}}
\newcommand{\N}{\mathcal{N}}
\newcommand{\Ker}{\mathrm{Ker}}
\newcommand{\Cov}{\mathrm{Cov}}
\newcommand{\Prob}{\mathbb{P}}
\DeclarePairedDelimiterX\innerp[2]{(}{)}{#1\delimsize\vert\mathopen{}#2}
\DeclareMathOperator*{\argmax}{argmax}
\DeclareMathOperator*{\argmin}{argmin}
\DeclarePairedDelimiter{\ceil}{\lceil}{\rceil}

\setlist{topsep=1ex,parsep=1ex,itemsep=0ex}
\setlist[1]{leftmargin=\parindent}
\setlist[enumerate,1]{label=\arabic*.,ref=\arabic*}
\setlist[enumerate,2]{label=(\alph*),ref=(\alph*)}

% For embedding images
\graphicspath{ {./images/} }

% Specifically for paper formatting 
\renewcommand{\baselinestretch}{1.2} % Spaces manuscript for easy reading

% Formatting definitions, propositions, etc. 
\newtheorem{definition}{Definition}
\newtheorem{prop}{Proposition}
\newtheorem{lemma}{Lemma}
\newtheorem{thm}{Theorem}
\newtheorem{corollary}{Corollary}

% Title and author
\title{Laws of Large Numbers and Central Limit Theorems}
\author{Andrew Roberts}

\begin{document}

\maketitle
\tableofcontents
\newpage


% Introduction
\section{Introduction}
In this writeup I banish all but the simplest proofs to the appendix, in an effort to focus the main body of the text on big picture concepts and intuition. 

% Notions of Convergence
\section{Notions of Convergence}
A random variable is simply a measurable function $X: \Omega \to \R$, where $(\Omega, \mathcal{F}, \mathbb{P})$ is a measure space. Therefore, convergence of random variables
really just boils down to convergence of functions. What differentiates such convergence from that studied in real analysis is that the input space $\Omega$ of the function is \textit{weighted}, 
whereas in classical real analysis there is no weighting scheme that gives more emphasis to certain input values $\omega \in \Omega$ and less to others. Ignoring the weights implied by the 
probability measure, we could simply consider the standard notion of \textit{pointwise convergence} of a sequence of random variables $\{X_n\}$ to some limiting random variable $X$:
\begin{align}
\lim_{n \to \infty} X_n(\omega) = X(\omega), \text{ for all } \omega \in \Omega \label{Pointwise_Convergence}
\end{align}
However, as we already discussed, this completely ignores $\mathbb{P}$. Convergence of random variables should naturally incorporate knowledge of this measure, which intuitively tells us 
how much emphasis should be placed on particular $\omega$. The notion of pointwise convergence [\ref{Pointwise_Convergence}] treats all $\omega$ equally. There are many different reasonable
ways that we might incorporate $\mathbb{P}$ into the definition of a limit, yielding different notions of convergence, each of which have strengths and weaknesses with respect to particular applications. 
We consider the most popular notions of convergence below. 

\subsection{Almost Sure Convergence}
Almost sure convergence incorporates knowledge of $\mathbb{P}$ in the simplest manner possible--it simply ignores the set of $\omega$ with zero probability. This is a quite natural slight adjustment 
of pointwise convergence; if $\mathbb{P}$ assigns zero probability to a set $A \in \mathcal{F}$, then requiring $\{X_n\}$ to converge on this set seems to be overly restrictive. Almost sure convergence 
simply removes this restriction. However, beyond this adjustment it still does not consider any sort of weighting for the set of positive measure. It just partitions the set of positive measure from that of zero 
measure, and ignores the latter. 

\begin{definition}
A sequence of random variables $\{X_n\}$ is said to converge almost surely (a.s.) to a random variable $X$ provided that 
\[\mathbb{P}\left(\lim_{n \to \infty} X_n(\omega) = X(\omega)\right) = 1\]
As shorthand, we write $X_n \overset{a.s.}{\to} X$.
\end{definition}  
To be more explicit, we might write out the definition as 
\[\mathbb{P}\left(\left\{\omega \in \Omega: \lim_{n \to \infty} X_n(\omega) = X(\omega)\right\} \right) = 1\]
or equivalently, 
\[\mathbb{P}\left(\left\{\omega \in \Omega: \lim_{n \to \infty} X_n(\omega) \neq X(\omega)\right\} \right) = 0\]
Almost sure convergence says that by going far enough along in the sequence we can make $X_n(\omega)$ and $X(\omega)$ arbitrarily close for almost all $\omega$.

To dig into this a bit deeper, we recall the definition of the standard pointwise limit $\lim_{n \to \infty} X_n(\omega) = X(\omega)$ from real analysis: 
\begin{align*}
\forall \epsilon > 0, \ \exists N_{\epsilon, \omega} \in \mathbb{N} \text{ s.t. } \forall n \geq N_{\epsilon, \omega}, \ \abs{X_n(\omega) - X(\omega)} < \epsilon 
\end{align*} 
where I utilize the subscripts $N_{\epsilon, \omega}$ to emphasize that $N$ depends on both $\epsilon$ and $\omega$. Almost sure convergence says that existence of such an $N$ for 
every $\epsilon > 0$ holds for almost all $\omega \in \Omega$. It can be shown that this implies for any $\epsilon > 0$ the existence of an $N \in \mathbb{N}$ satisfying 
\[\Prob(\abs{X_n(\omega) - X(\omega)} < \epsilon) = 1, \text{ for all } n \geq N\]
I find this statement to provide the best interpretation of almost sure convergence. Let's call the event $\{X_n(\omega) \text{ is outside of an } \epsilon\text{-ball of } X(\omega)\}$ an \textit{unusual event} 
(where the radius $\epsilon$ is a measure of how unusual). Then $X_n(\omega) \overset{a.s.}{\to} X(\omega)$ means that if we go far enough along in the sequence we can drive the probability of 
an unusual event to \textit{zero}. This is a powerful statement, though it gives no sense of how far along in the sequence one must go to achieve this. This interpretation of almost sure convergence is also 
quite helpful when trying to understand its difference from convergence in probability, which is discussed below. 

\subsection{Convergence in Probability}
Another reasonable way to define convergence that takes the probability measure into account is to consider the probability that $X_n$ is ``far from'' $X$ as a function function of $n$, and require that this 
probability converge to $0$ as $n \to \infty$.  
\begin{definition}
A sequence of random variables $\{X_n\}$ is said to converge in probability to a random variable $X$ provided that for any $\epsilon > 0$,
\[\lim_{n \to \infty} \Prob\left(\abs{X_n(\omega) - X(\omega)} \geq \epsilon \right) = 0\]
As shorthand, we write $X_n \overset{p}{\to} X$.
\end{definition}
This is certainly related to almost sure convergence, and I find it quite easy to accidentally conflate the two. Note that in the definition of almost sure convergence the probability measure appears outside of the limit, 
while here it appears inside of the limit. To better understand the difference, we consider manipulating the above definition. 
Plugging in the standard definition of a limit, we obtain the equivalent 
definition that for any $\epsilon > 0$,
\[\forall \epsilon^\prime > 0, \text{ there exists } N \in \mathbb{N} \text{ s.t. for all } n \geq N, \ \Prob\left(\abs{X_n(\omega) - X(\omega)} \geq \epsilon \right) < \epsilon^\prime \]
Now we're in a nice position to compare to almost sure convergence. For $\epsilon > 0$, we found that almost sure convergence guaranteed the existence of an $N \in \mathbb{N}$
satisfying 
\[\Prob(\abs{X_n(\omega) - X(\omega)} \geq \epsilon) = 0, \text{ for all } n \geq N\]
while convergence in probability guarantees the existence of an $N$ satisfying 
\[\Prob(\abs{X_n(\omega) - X(\omega)} \geq \epsilon) < \epsilon^\prime, \text{ for all } n \geq N, \text{ for any } \epsilon^\prime > 0\]
That is, almost sure convergence guarantees a finite number of unusual events--by choosing $N$ large enough the probability of an unusual event may be driven to zero. Convergence in 
probability gives a weaker guarantee; the probability of an unusual event my be driven to be arbitrarily small, but is not guaranteed to ever reach zero. To better understand this distinction, 
we consider a specific example. 

\subsubsection{Example: A sequence that converges in probability but not almost surely}
TODO

% Overview of LLNs
\section{Laws of Large Numbers: The Big Picture}
TODO: the weak vs. the strong law

% WLLN
\section{Weak Laws of Large Numbers}
This section details a variety of weak laws of large numbers (WLLNs), which concern the convergence of sums of random variables in probability. Recall from the previous section that intuitively we can think 
of these theorems as guaranteeing that the ``error rate'' or probability of an ``unusual event'' becomes arbitrarily small as $N$ increases, but they are weak in the sense that the error probability is never guaranteed 
to reach zero. There are many WLLNs that make different assumptions and conclusions, but they are all generally concerned with the convergence in probability of sums or empirical means of random variables. 

\subsection{Simplest Case: Finite Variance}
We begin with the most basic WLLN, which assumes the random variables are iid with finite variance. The finite variance assumption guarantees that the probability distribution is not too ``spread out'', which implies 
that taking empirical means will reduce the randomness in the distribution and tend toward a single value. With this assumption, the WLLN is quite easy to prove, following from a simple application of Chebyshev's 
inequality. 
\begin{thm} 
Let $\{X_k\}_{k = 1}^{\infty}$ be a sequence of iid random variables with finite mean $\mu := \E X_1 < \infty$ and variance $\sigma^2 := \Var(X_1) < \infty$. Define the partial sum $S_N := \sum_{k = 1}^{N} X_k$. Then 
\[\frac{S_N}{N} \overset{p}{\to} \mu\]
\end{thm}

\begin{proof}
Let $\epsilon > 0$. We must show that 
\[\lim_{N \to \infty} \Prob\left(\abs{\frac{S_N}{N} - \mu} > \epsilon \right) = 0\]
To apply Chebyshev, we will require the variance $\Var(S_N)$. This is easily calculated using the iid assumption and finite variance assumption:
\[\Var(S_N) = \Var\left(\sum_{k = 1}^{N} X_k \right) = \sum_{k = 1}^{N} \Var(X_k) =  \sum_{k = 1}^{N} \sigma^2 = N\sigma^2 \]
By Chebyshev, 
\begin{align*}
\Prob\left(\abs{\frac{S_N}{N} - \mu} > \epsilon \right) &\leq \frac{\Var\left(\frac{S_N}{N}\right)}{\epsilon^2} \\
									       &= \frac{\sigma^2}{N\epsilon^2} \to 0 \text{ as } N \to \infty 
\end{align*}
\end{proof}

\subsection{What happens when variance is infinite?}
Speaking very loosely, we described the finite variance assumption as a guarantee that the distribution of the $X_k$ is not too spread out, allowing the distribution of the partial sum $S_N$ to collapse as $N$ is increased. 
It turns out that this assumption is actually not necessary at all--we can conclude the same exact result without the finite variance assumption. However, the finite mean assumption is still required, which plays a similar role by 
bounding how spread out the distribution can be. The proof of this more general result is longer, but still not too bad. It is included below, as the ``truncated Chebyshev'' technique that is employed is a very common strategy to 
deal with cases of infinite variance. This technique is fairly intuitive: modify $X_k$ (which has infinite variance) by chopping off its values outside of a specified interval, leading to a modified random variable with finite support and 
hence finite variance. We can then apply Chebyshev to this truncated random variable. As we extend the truncation interval, the truncated variables approach the original ones, and we hope that in the limit the partial sums of these 
two respective sequences will have the same limit. In order for the truncated Chebyshev argument to work, intuitively the key will be making sure the tails of the $X_k$ decay sufficiently fast. The finite variance assumption guaranteed
this, but we will show below that the finite mean assumption is actually sufficient as well. 

A shorter proof using characteristic functions is given in the appendix, but the truncation technique is presented here as it will sometimes be applicable when the 
characteristic function technique is not. 

\begin{thm}
Let $\{X_k\}_{k = 1}^{\infty}$ be a sequence of iid random variables with finite mean $\mu := \E X_1 < \infty$. Define the partial sum $S_N := \sum_{k = 1}^{N} X_k$. Then 
\[\frac{S_N}{N} \overset{p}{\to} \mu\]
\end{thm}

\begin{proof}
Let $\epsilon > 0$. 
We must show that 
\[\lim_{N \to \infty} \Prob\left(\abs{\frac{S_N}{N} - \mu} > \epsilon \right) = 0\]
Define the truncated random variables 
\[Y_{N,k} := X_k \mathbbm{1}\{\abs{X_k} \leq N \epsilon^3\}\]
The $\epsilon^3$ is chosen precisely because it makes the bound come out correctly, as we will see. Therefore, $Y_{N,k}$ agrees with $X_k$ on the set
 $\{\omega \in \Omega: \abs{X(\omega)} \leq N\epsilon^3\}$, and outside of this set $Y_{N, k}$ is set to $0$. Therefore, in absolute value $\abs{Y_{N,k}} \leq \abs{X_{N,k}}$. Let $S_N^\prime := \sum_{k = 1}^{N} Y_{N,k}$
 denote the partial sums of the sequence $\{Y_{N, k}\}$. Note also that, for a fixed $N$, then each $Y_{N, k}$ is a fixed function of the iid random variables $X_k$ and hence are themselves iid.
 
 The strategy here is to apply the triangle inequality 
 \begin{align*}
 \Prob\left(\abs{\frac{S_N}{N} - \mu} > \epsilon \right) &\leq \Prob\left(\abs{\frac{S_N}{N} - \frac{\E S^\prime_N}{N} } + \abs{\frac{\E S^\prime_N}{N}  - \mu} > \epsilon \right) \\
 										&\leq \Prob\left(\left\{\abs{\frac{S_N}{N} - \frac{\E S^\prime_N}{N} } > \frac{\epsilon}{2} \right\} \bigcup \left\{\abs{\frac{\E S^\prime_N}{N}  - \mu} > \frac{\epsilon}{2}\right\} \right) \\
										&\leq \Prob\left(\abs{\frac{S_N}{N} - \frac{\E S^\prime_N}{N} } > \frac{\epsilon}{2}\right) + \Prob\left(\abs{\frac{\E S^\prime_N}{N}  - \mu} > \frac{\epsilon}{2}\right)
 \end{align*}
 If we show that 
 \begin{enumerate}
 \item $\frac{\E S_N^\prime}{N} \to \mu$ (typical convergence for deterministic sequences)
 \item $\frac{S_N}{N} - \frac{\E S_N^\prime}{N} \overset{p}{\to} 0$
 \end{enumerate}
 then we can drive each of the terms to $0$ to prove the result. Let's begin with the first item, which is the easier of the two. \\
 
 \bigskip
 \noindent
 \textbf{1. Showing $\frac{\E S_N^\prime}{N} \to \mu$.} \\
 Since $\{Y_{N,k}\}_{k = 1}^{\infty}$ form a sequence of iid random variables, then the expectation of $S_N^\prime$ simplifies to 
 \begin{align*}
 \E S_N^\prime &= N \cdot \E\left[X_1 \mathbbm{1}\{\abs{X_1} \leq N\epsilon^3\} \right]
 \end{align*}
 Therefore, 
 \[\frac{\E S_N^\prime}{N} = \E\left[X_1 \mathbbm{1}\{\abs{X_1} \leq N\epsilon^3\}\right]\]
 which I claim converges in probability to $\mu$ by the dominated convergence theorem. Indeed, we have 
 \[X_1 \mathbbm{1}\{\abs{X_1} \leq N\epsilon^3\} \leq \abs{X_1}\]
 and 
 \[X_1 \mathbbm{1}\{\abs{X_1} \leq N\epsilon^3\} \overset{a.s.}{\to} X_1 \]
 The dominated convergence theorem therefore allows interchanging the limit and expectation, which yields
 \[\frac{\E S_N^\prime}{N} = \E\left[X_1 \mathbbm{1}\{\abs{X_1} \leq N\epsilon^3\}\right] \to \E X_1 = \mu \]
 as desired. 
 
 \bigskip
 \noindent
 \textbf{2. Showing $\frac{S_N}{N} - \frac{\E S_N^\prime}{N} \overset{p}{\to} 0$.} \\
 Note that we cannot immediately apply Chebyshev to bound $\Prob\left(\abs{\frac{S_N}{N} - \frac{\E S_N^\prime}{N}} > \epsilon\right) = \Prob\left(\abs{S_N - \E S_N^\prime} > N\epsilon\right)$ since 
 $S_N$ may have infinite variance. We therefore first utilize a crude bound to write the expression in terms of $S_N^\prime$, which does have finite variance. 
 \begin{align*}
 \Prob\left(\abs{S_N - \E S_N^\prime} > N\epsilon\right) &= \Prob\left(\left[\abs{S_N - \E S_N^\prime} > N\epsilon\right] \cap \left[S_N = S_N^\prime\right] \right) +  \Prob\left(\left[\abs{S_N - \E S_N^\prime} > N\epsilon\right] \cap \left[S_N \neq S_N^\prime\right] \right) \\
 										    &\leq \Prob\left(\abs{S^\prime_N - \E S_N^\prime} > N\epsilon \right) + \Prob(S_N \neq S_N^\prime) \\
										    &= \Prob\left(\abs{S^\prime_N - \E S_N^\prime} > N\epsilon \right) + \Prob\left(\sum_{k = 1}^{N} (X_k - Y_{N,k}) \neq 0\right) \\
										    &\leq \frac{\Var(S_N^\prime)}{N^2 \epsilon^2} + \Prob\left(\bigcup_{k = 1}^{N} \{X_k \neq Y_{N,k}\} \right) \\
										    &\leq \frac{\Var(S_N^\prime)}{N^2 \epsilon^2} + \sum_{k = 1}^{N} \Prob(X_k \neq Y_{N,k}) \\
										    &= \frac{\Var(S_N^\prime)}{N^2 \epsilon^2} + \sum_{k = 1}^{N} \Prob(\abs{X_k} > N\epsilon^3) \\
										    &= \frac{\Var(S_N^\prime)}{N^2 \epsilon^2} + N \cdot \Prob(\abs{X_1} > N\epsilon^3)
 \end{align*}
 The second inequality follows from Chebyshev (left term) and the fact that $\left\{\sum_{k = 1}^{N} (X_k - Y_{N,k}) \neq 0\right\} \subset \bigcup_{k = 1}^{N} \{X_k \neq Y_{N,k}\}$ (right term). The latter says that for the sum to 
 be non-zero, at least one of the terms needs to be non-zero. We continue by bounding the first term $\frac{\Var(S_N^\prime)}{N^2 \epsilon^2}$. 
 \begin{align*}
 \frac{\Var(S_N^\prime)}{N^2 \epsilon^2} &= \frac{N \cdot \Var(Y_{N,1})}{N^2 \epsilon^2} \\
 							      &=  \frac{\Var(Y_{N,1})}{N \epsilon^2} \\
							      &= \frac{\E[Y_{N,1}^2] - \E[Y_{N,1}]^2}{N\epsilon^2} \\
							      &\leq  \frac{\E[Y_{N,1}^2]}{N\epsilon^2} \\
							      &\leq \frac{1}{N\epsilon^2} \int_{\{\abs{X_1} \leq N\epsilon^3\}} X_1^2(\omega) d\Prob(\omega) \\
							      &= \frac{1}{N\epsilon^2} \int_{\{\abs{X_1} \leq N\epsilon^3\}} \abs{X_1(\omega)}\abs{X_1(\omega)} d\Prob(\omega) \\
							      &\leq \frac{1}{N\epsilon^2} \int_{\{\abs{X_1} \leq N\epsilon^3\}} \abs{X_1(\omega)} \cdot N\epsilon^3 d\Prob(\omega) \\
							      &= \epsilon \cdot \E\left[\abs{X_1}\mathbbm{1}\{\abs{X_1} \leq N\epsilon^3\} \right] \\ 
							      &\leq \epsilon \cdot \E\abs{X_1}
 \end{align*}
 Note that for the penultimate inequality, we broke $X_1^2$ into two terms and bounded one of the terms by the maximum value the function can assume on the domain of integration. This achieves a tighter bound 
 then if we had bounded both terms. 
 
 To finish the proof we show that 
 \[\lim \sup_{N \to \infty}  \Prob\left(\abs{S_N - \E S_N^\prime} > N\epsilon\right) = 0\]
 Considering the $\lim \sup$ allows us to not to worry about whether the limit actually exists, and is sufficient to conclude $\frac{S_N}{N} - \frac{\E S_N^\prime}{N} \overset{p}{\to} 0$. To this end, we have
 \begin{align*}
 \lim \sup_{N \to \infty}  \Prob\left(\abs{S_N - \E S_N^\prime} > N\epsilon\right) &= \lim \sup_{N \to \infty} \left[\frac{\Var(S_N^\prime)}{N^2 \epsilon^2}  + N \cdot \Prob(\abs{X_1} > N\epsilon^3) \right] \\
 														       &\leq \epsilon \cdot \E\abs{X_1} + \lim \sup_{N \to \infty} N \cdot \Prob(\abs{X_1} > N\epsilon^3) \\
														       &= \epsilon \cdot \E\abs{X_1}  && \text{See below}
 \end{align*}
 Since $\epsilon > 0$ is arbitrary, then this inequality indeed proves that the limit is $0$. To achieve this, we claimed above that 
 \[\lim \sup_{N \to \infty} N \cdot \Prob(\abs{X_1} > N\epsilon^3) = 0\]
 which essentially is a claim that $\Prob(\abs{X_1} > N\epsilon^3)$ tends to $0$ faster than $\frac{1}{N}$. To show this, consider
 \[N \cdot \Prob(\abs{X_1} > N\epsilon^3) = N \cdot \E\left[N \cdot \mathbbm{1}\{\abs{X_1} > N\epsilon^3\} \right] \leq \E\left[\frac{\abs{X_1}}{\epsilon^3} \mathbbm{1}\{\abs{X_1} > N\epsilon^3\} \right]\]
 The bound follows from the fact that whenever the indicator function is non-zero it follows that $N < \frac{\abs{X_1}}{\epsilon^3}$. We are now in a position to apply the dominated convergence theorem. Indeed, 
 we have 
 \[\frac{\abs{X_1}}{\epsilon^3} \mathbbm{1}\{\abs{X_1} > N\epsilon^3\} \leq \frac{\abs{X_1}}{\epsilon^3}\]
 and
 \[\frac{\abs{X_1}}{\epsilon^3} \mathbbm{1}\{\abs{X_1} > N\epsilon^3\} \overset{a.s.}{\to} 0\]
 Therefore, 
 \[\E\left[\frac{\abs{X_1}}{\epsilon^3} \mathbbm{1}\{\abs{X_1} > N\epsilon^3\} \right] \to \E[0] = 0\]
Note that in each of the dominated convergence theorem applications in this proof, we used the dominating function $\abs{X_1}$ which requires the assumption that $\abs{X_1}$ is integrable. Hence, we are using the finite 
mean assumption here. 
 This verifies that $\frac{S_N}{N} - \frac{\E S_N^\prime}{N} \overset{p}{\to} 0$ and thus completes the proof. 
\end{proof}
Notice that in this proof we were able to replace the random variable $S_N^\prime$ with the deterministic value $\E S_N^\prime$, a common strategy in these types of proofs.  

\subsection{What happens when the mean is infinite?}
In the previous section we showed that even without the finite variance assumption, the tails of the $X_k$ decayed sufficiently fast to still conclude the WLLN. The natural next question is whether the finite mean assumption 
is necessary for this to hold, or if this assumption is also unnecessary? It turns out that the finite mean assumption is indeed necessary to prove the WLLN in general, but that there are examples where the WLLN holds despite 
the mean being infinite. To better understand the implications of infinite mean, we explore two examples: 1.) an example where the mean is infinite and the WLLN is not even close to holding, and 2.) an example where the mean is 
infinite but the WLLN still holds. 

\subsubsection{Example: WLLN Fails with Infinite Mean}
Unsurprisingly, for this example we consider the classic pathological fat-tailed distribution, the Cauchy distribution, which has infinite mean and variance.
 In particular, suppose $X_k \overset{iid}{\sim} \text{Cauchy}(0, 1)$ so that the $X_k$ have characteristic functions
\[\varphi_{X_k}(t) = e^{-\abs{t}}\]
As usual, let $S_N := \sum_{k = 1}^{N} X_k$ denote the partial sums. We will utilize the characteristic function $\phi_{\frac{S_N}{N}}$ to analyze the convergence behavior of the sequence of empirical means $\frac{S_N}{N}$. We can 
find the characteristic function of $\phi_{\frac{S_N}{N}}$ by exploiting the fact that the $X_k$ are iid: 
\begin{align*}
\phi_{\frac{S_N}{N}} &= \E e^{it \frac{S_N}{N}} \\
			       &= \E \exp\left\{i\frac{t}{N} \sum_{k = 1}^{N} X_k \right\} \\
			       &= \E \prod_{k = 1}^{N} \exp\left\{i \frac{t}{N} X_k \right\} \\
			       &= \prod_{k = 1}^{N} \E \exp\left\{i \frac{t}{N} X_k \right\} && \text{Independence} \\
			       &=  \prod_{k = 1}^{N} \varphi_{X_k}(t/N) \\
			       &= \prod_{k = 1}^{N} e^{-\abs{\frac{t}{N}}} \\
			       &= \left[ e^{-\abs{\frac{t}{N}}} \right]^N && \text{Identically distributed} \\
			       &= e^{-\abs{t}}
\end{align*} 
So the characteristic function of the empirical mean is the same as that of the individual $X_k$! Intuitively, the Cauchy distribution is so ``spread out'' that taking empirical means doesn't reduct the randomness at all. Not only does 
$\frac{S_N}{N}$ not converge to $\E X_1$, but it has the same exact distribution for any $N$.  

\subsubsection{Example: WLLN Holds with Infinite Mean}
The Cauchy example doesn't tell the whole story; here we consider a difference distribution with infinite mean where the WLLN still holds. Let $X_k \overset{iid}{\sim} f$ where $f$ is a density defined by 
\[f(x) = \frac{C}{x^2 \log \abs{x}} \mathbbm{1}\{\abs{x} > e\}\]
where $C$ is just a normalizing constant. Note that it can be shown that the distributions with the fattest possible tails subject to finite variance on the real line have densities that look like $\frac{1}{x^3}$. We see that 
the above density replaces one of the $x$ terms with $\log \abs{x}$, resulting in a slower rate of convergence, and therefore immediately indicating that the at least the second moment should be infinite, and maybe also the 
first. However, we're getting ahead of ourselves here. First let's verify that this is indeed a valid density. This is easily verified by 
\[\int_{e}^{\infty} \frac{1}{x^2 \log x} < \int_{e}^{\infty} \frac{1}{x^2} dx < \infty\]
and then applying symmetry for the negative half. However, it can be verified that 
\[\E\abs{X} = 2C \int_{e}^{\infty} \frac{1}{x\log x} dx = \infty\]
so the mean is indeed infinite. It remains to show that the WLLN still holds. 

We will use a truncated Chebyshev approach very similar to that used to prove the WLLN with finite mean. As in that proof, define the truncated random variables $Y_{N,k} := X_k \mathbbm{1}\{\abs{X_k} \leq N\}$ and 
$S_N^\prime := \sum_{k = 1}^{N} Y_{N, k}$. The WLLN with finite mean proof used this strategy, but relied on the fact that $\E\abs{X_1} < \infty$. Therefore, we will need to leverage the additional information at our disposal 
in this specific example in order to draw the same conclusion. We have quite a bit of information in the form of the density function of the $X_k$. Indeed, we see that the $X_k$ are symmetric and hence the $Y_{N, k}$ are 
also symmetric, so that $\E[Y_{N, k}]$ = 0. Now, recall that the triangle inequality argument in the WLLN with finite mean proof required verifying two facts: 
 \begin{enumerate}
 \item $\frac{\E S_N^\prime}{N} \to \mu$ (typical convergence for deterministic sequences)
 \item $\frac{S_N}{N} - \frac{\E S_N^\prime}{N} \overset{p}{\to} 0$
 \end{enumerate}
 The symmetry in this example makes the first item trivial! Indeed, $\frac{\E S_N^\prime}{N} = \frac{N \mu}{N} = \mu$. To establish the second, we once again apply the truncated Chebyshev inequality. 
 \begin{align*}
 \Prob(\abs{S_N - \E S_N^\prime} > N\epsilon) &= \Prob(\abs{S_N} > N\epsilon) && \text{Symmetry} \\
 								      &\leq \frac{\Var(S_N^\prime)}{N^2 \epsilon^2} + N \cdot \Prob(\abs{X_1} > N)
 \end{align*}
 To verify the second line, review the analogous steps in the proof of the WLLN with finite mean; the conclusion (which is the truncated Chebyshev inequality) did not rely on the finite mean assumption. I claim that the limit 
 of both terms is $0$, in which case the claim is established. 


We can interpret this example as an ``edge case'', in which the tails do not converge quite fast enough to yield a finite mean, yet still fast enough such that taking empirical means reduces the randomness in the distribution. 

\subsection{More General Weak Laws}
We have so far looked at sufficient conditions that yield the weak law
\[\frac{S_N}{N} \overset{p}{\to} \mu\]
It is natural to take a step back and wonder if this is just a specific case of a more general phenomenon. Some questions we might ask are:
\begin{enumerate}
\item Are there situations in which a scaling other than $\frac{1}{N}$ yields the proper scaling for a weak law? Can we consider other polynomial rates $\frac{1}{N^r}$? Or even more generally, scalings of the 
form $\frac{1}{b_N}$, where $\{b_N\}$ is an arbitrary increasing sequence? 
\item Can we find sufficient \textit{and} necessary conditions for the weak law to hold? In essence, can we find precisely the correct conditions that correspond to a weak law existing? 
\end{enumerate}
Both of these questions are considered below. 

\subsubsection{Different Scaling Rates: The Marcinkiewiz-Zygmund Weak Law}
We first consider the question of viewing the scaling $\frac{S_N}{N}$ as a special case of a more general polynomial scaling $\frac{S_N}{N^r}$. This generalization is provided by the 
Marcinkiewiz-Zygmund weak Law. I will present this law in two parts, although the theorem can be summarized in a single statement. The first part establishes a weak law in a setting where 
one did not exist before. In particular, if the mean does not exist, but a weaker moment exists than we can still conclude a weak law, but will now need to scale the sum by a larger term to account 
for the extra randomness in the tails. 
\begin{thm} 
Suppose $\{X_k\}$ is an iid sequence with $\E\abs{X_1}^r < \infty$ for some $r \in [0, 1)$. Then 
\[\frac{S_N}{N^{1/r}} \overset{p}{\to} \mu\]
\end{thm}

The second part pertains to a setting in which the mean is finite (and hence the WLLN already holds), but states that stronger assumptions can allow us to conclude stronger results. 
In particular, if we make stronger assumptions 
on the decay of the tails (i.e. higher moments exist) then this translates into a smaller scaling (i.e. the denominator required to properly collapse the sum is now smaller). 
\begin{thm} 
Suppose $\{X_k\}$ is an iid sequence with $\E\abs{X_1}^r < \infty$ for some $r \in [1, 2)$. Moreover, assume $\E X_1 = 0$. Then 
\[\frac{S_N}{N^{1/r}} \overset{p}{\to} \mu\]
\end{thm}
Notice that the $r = 1$ case is simply the standard WLLN. In this second case we require the extra assumption $\E X_1 = 0$ for the proof to work, but it is not limiting in practice as we can 
always de-mean random variables. This assumption would not even be meaningful in the first part of the theorem, as the first part pertains to the setting where the mean does not exist. 

\subsubsection{A General Weak Law with Necessary and Sufficient Conditions}
We now consider a very general weak law that considers arbitrary increasing normalizing sequences $\{b_N\}$, and establishes both necessary and sufficient conditions. This theorem is a bit 
more complicated to state, but this is not surprising as it zeros in on the very particular conditions that are required for a weak law to hold in a very general setting. 
\begin{thm}
Suppose $\{X_k\}$ is an independent (not necessarily iid) sequence of random variables and $\{b_N\}$ is an increasing, non-random sequence of positive reals. Define the cutoff random variables 
\begin{align*}
&Y_{N, k} = X_k \mathbbm{1}\{\abs{X_k} \leq b_N\}, &&S_N^\prime = \sum_{k = 1}^{N} Y_{N, k}
\end{align*}
\begin{enumerate}
\item If
\[\lim_{N \to \infty} \sum_{k = 1}^{N} \Prob(\abs{X_k} > b_N) = 0\]
and 
\[\lim_{N \to \infty} \frac{1}{b_N^2} \sum_{k = 1}^{N} \Var(Y_{N, k}) = 0\]
then
\[\frac{S_N - \E S_N^\prime}{b_N} \overset{p}{\to} 0\]

\item Conversely, if 
\[\frac{S_N - \E S_N^\prime}{b_N} \overset{p}{\to} 0\]
and 
\[\frac{S_N^\prime}{b_N} \to 0\]
then 
\[\lim_{N \to \infty} \sum_{k = 1}^{N} \Prob(\abs{X_k} > b_n) = 0\]
and 
\[\lim_{N \to \infty} \frac{1}{b_N^2} \sum_{k = 1}^{N} \Var(Y_{N, k}) = 0\]
hold. 
\end{enumerate}
\end{thm}

\subsection{A Weak Law for Partial Maxima}
All of the weak laws detailed above conclude that the partial sums $S_N$, when properly normalized by some increasing sequence $\{b_N\}$, converge in probability to the mean $\mu$. Can we establish 
similar results for statistics other than sums? Indeed we can--below we  present a weak law for partial maxima, where the normalized partial maxima converge in probability to $0$. We begin by stating the result
for the more-familiar normalizing sequence of polynomials $\{N^{1/r}\}$ before generalizing to arbitrary increasing sequences $\{b_N\}$. 

\begin{thm}
Suppose $\{X_k\}$ is an iid sequence and define $Y_N := \max_{1 \leq k \leq N} \abs{X_k}$. Then, for $r > 0$, 
\[\frac{Y_N}{N^{1/r}} \overset{p}{\to} 0 \iff N \cdot \Prob(\abs{X} > N^{1/r}) \to 0\]
\end{thm}
Therefore, a weak law for partial maxima exists precisely when the tail  probabilities $\Prob(\abs{X} > N^{1/r})$ decay faster than $\frac{1}{N}$. A larger $r$ indicates lighter tails, in which case the 
normalization need not be as strong. I also emphasize that this tail decay condition is necessary and sufficient. 

Generalizing to arbitrary increasing sequences $\{b_N\}$ is relatively straightforward, but requires a ``holds for all $\epsilon > 0$ condition'', as stated below. 
\begin{thm}
Suppose $\{X_k\}$ is an iid sequence of random variables and $\{b_N\}$ a sequence of non-decreasing positive reals. Define $Y_N := \max_{1 \leq k \leq N} \abs{X_k}$. Then, for $r > 0$, 
\[\frac{Y_N}{b_N} \overset{p}{\to} 0 \iff N \cdot \Prob(\abs{X} > b_N \epsilon) \to 0 \text{ for all } \epsilon > 0\]
\end{thm}

\section{The Strong Law of Large Numbers}
Under suitable assumptions, the strong law of large numbers (SLLN) yields convergence of the empirical average $\overline{X}_n = \frac{S_n}{n}$ to the true mean $\mu$ in the iid case just like the WLLN, but now in the stronger 
sense of almost sure convergence
\[\frac{S_n}{n} \overset{a.s.}{\to} \mu\]
As discussed in previous sections, this is the notion of convergence we would intuitively want when considering problems of statistical learning. It turns out that proving this stronger theorem is much more involved than 
proving the weak law. To do so, we show that proving the strong law is equivalent to establishing the convergence of an infinite series of random variables, so we must first develop a theory of random series. Here is the 
link between random series and strong laws (Gut lemma 6.5.1)
\begin{lemma} \label{series_convergence_lemma}
Let $\{X_k\}$ be random variables and $\{a_n\}$ a sequence of positive real numbers increasing to infinity. Set $a_0 = 0$. Then, 
\[\sum_{k = 1}^{\infty} \frac{X_k}{a_k} \text{ converges a.s. } \implies \frac{1}{a_n} \sum_{k = 1}^{n} X_k \overset{a.s.}{\to} 0\]
\end{lemma}
Note that setting $a_n := n$ gives $\frac{S_n}{n}$ on the righthand side, which is the typical sequence of interest in laws of large numbers. The convergence on the righthand side should be interpreted 
as a convergent sequence as $n \to \infty$. This result makes intuitive sense since the $\{a_n\}$ are increasing, so the normalization from dividing by $a_n$ on the righthand side is a stronger effect than 
the normalization on the lefthand side. So the goal of establishing a strong law becomes the goal of showing the series $\sum_{k = 1}^{\infty} \frac{X_k}{a_k}$ is almost surely convergent. 

\subsection{Establishing the Convergence of Random Series}
We will detail a few different results that provide sufficient conditions for a random series to converge, and hence will provide sufficient conditions for the strong law to hold. However, the crown jewel will 
be the \textit{Kolmogorov Three Series theorem}, which provides sufficient and necessary conditions for series convergence and thus will yield an ``if and only if'' strong law.  

\begin{thm} \label{Kolmogorov_convergence_criterion}
(Kolmogorov convergence criterion) Let $\{X_k\}$ be independent with means $\{\mu_k\}$ and variances $\{\sigma_k^2\}$. Then 
\[\sum_{k = 1}^{\infty} \sigma_k^2 < \infty \implies \sum_{k = 1}^{\infty} (X_k - \mu_k) \text{ converges a.s. }\]
If, in addition, $\sum_{k = 1}^{\infty} \mu_k$ converges then $\sum_{k = 1}^{\infty} X_k$ converges almost surely. 
\end{thm}
I emphasize again that convergence of infinite series here should be interpreted as convergence of the sequence of partial sums. This result again makes intuitive sense, since the assumption 
$\sum_{k = 1}^{\infty} \sigma_k^2 < \infty$ essentially means that the variance is eventually $0$, which implies that $X_k$ tends towards its mean so that $X_k - \mu_k$ tends towards $0$. So eventually 
nothing is being added to the partial sums so they converge. To make this a little more precise, it is important to think about the sufficient condition $\sum_{k = 1}^{\infty} \sigma_k^2 < \infty$. In particular, 
this condition implies that $\sigma_k^2 \to 0$ and that the $\sigma_k^2$ are uniformly bounded. However, these two facts are not enough to guarantee almost sure convergence of $\sum_{k = 1}^{\infty} (X_k - \mu_k)$. 
The condition $\sum_{k = 1}^{\infty} \sigma_k^2 < \infty$ is much stronger, as it essentially ensures that the $\sigma_k^2$ converge to $0$ ``fast enough''. Indeed, the Cauchy convergence criterion states that 
\[\sum_{k = 1}^{\infty} \sigma_k^2 < \infty \iff \forall \epsilon > 0 \exists N \text{ s.t. } \forall n > m \geq N, \sum_{k = m + 1}^{n} \sigma_k^2 < \epsilon\]
This is much stronger than just saying $\sigma_k^2 < \epsilon$ for $k$ sufficiently large. The Kolmogorov convergence criterion is actually enough to establish a sufficient condition for the SLLN. However, the below theorem
is what is required to prove the true if and only if version of the strong law. Before stating the theorem, we provide some context. The above lemma assumed that the variances of the random variables were 
finite. It turns out this is not required for the SLLN to hold. To deal with infinite variances we utilize the same strategy as we did for the weak law, considering the truncated random variables
$Y_k := X_k \mathbbm{1}\{\abs{X_k} \leq A\}$ (just like we did for the weak law), since the truncated random variables have finite variance. 
The trick will be to show that if the partial sums of the $Y_k$ converge then so do the partial sums of the $X_k$. Note that this can only 
be true if 
\[\Prob(X_k \neq Y_k \text{ i.o.}) = 0\]
We already know from the Kolmogorov convergence criterion that if the variances of the $Y_k$ are summable and the means of the $Y_k$ converge then the partial sums of the $Y_k$ will converge. These two conditions
correspond to two of the series in the Kolmogorov three series theorem. The third series ensures the condition $\Prob(X_k \neq Y_k \text{ i.o.}) = 0$. Thus, it is really quite a small leap to extend the previous lemma to 
the infinite variance case. The truly remarkable part about the Kolmogorov three series theorem is that the implication also goes the other way. 
\begin{thm}
(Kolmogorov Three Series Theorem) Let $A > 0$ and $\{X_k\}$ independent random variables. Define the truncated random variables $Y_k := X_k \mathbbm{1}\{\abs{X_k} \leq A\}$. Then the sequence of partial sums 
$\{S_n\}$ converges almost surely if and only if
\begin{enumerate}
\item $\sum_{k = 1}^{\infty} \Prob(X_k \neq Y_k) = \sum_{k = 1}^{\infty} \Prob(\abs{X_k} > A) < \infty$ 
\item $\sum_{k = 1}^{\infty} \E Y_k$ converges 
\item $\sum_{k = 1}^{\infty} \Var(Y_k) < \infty$ 
\end{enumerate}
\end{thm}
To be clear here, a sufficient condition for $\{S_n\}$ to converge almost surely is that the three series must converge \textit{for some } $A > 0$. Conversely, if $\{S_n\}$ converges a.s. then 
necessarily the three series converge for any $A > 0$. Following this logic, if the three conditions hold for some $A$ then $\{S_n\}$ converges almost surely, which then implies that they hold 
for all $A$; the point here is that to establish that the SLLN holds we need only verify the conditions for a single $A$. Also note that the first series converging implies 
\[\Prob(X_k \neq Y_k \text{ i.o.}) = 0\]
by a direct application of the first Borel-Cantelli lemma. 

On a final note, we state some perhaps surprising relationships between the different notions of convergence of the sequence of partial sums $\{S_n\}$, in the case where the $X_k$ are 
independent. 
\begin{thm}
Let $\{X_k\}$ be a sequence of independent random variables with associated partial sums $\{S_n\}$. Then 
\begin{enumerate}
\item $S_n$ converges in distribution if and only if $S_n$ converges in probability. 
\item $S_n$ converges in probability if and only if $S_n$ converges almost surely. 
\end{enumerate}
\end{thm}
We know that almost sure convergence always implies convergence in probability, and convergence in probability always implies convergence in distribution. What is new here is the reverse inequalities. 

\subsection{The Strong Laws}
We begin by stating a version of the SLLN that establishes only a sufficient condition. This will be straightforward to prove given the above theorems regarding the convergence of random 
series. We then proceed to state, without proof, the if and only if statement, which relies on the Kolmogorov three series theorem.

\begin{thm}
Let $\{X_k\}$ be independent with mean zero and finite variances $\sigma_k^2$. Then 
\[\sum_{k = 1}^{\infty} \frac{\sigma_k^2}{k^2} < \infty \implies \frac{S_n}{n} \overset{a.s.}{\to} 0\]
\end{thm}

\begin{proof}
From \ref{series_convergence_lemma} it is sufficient to show that $\sum_{k = 1}^{n} \frac{X_k}{k}$ converges almost surely as $n \to \infty$. Since 
\[\sum_{k = 1}^{\infty} \Var\left(\frac{X_k}{k}\right) = \sum_{k = 1}^{\infty} \frac{\sigma_k^2}{k^2} < \infty \]
then by the Kolmogorov convergence criterion (\ref{Kolmogorov_convergence_criterion}), $\sum_{k = 1}^{n} \frac{X_k}{k}$ converges almost surely. 
\end{proof}
It is now quite straightforward to prove the commonly-stated sufficient condition for the SLLN in the finite variance case. 
\begin{corollary} 
Let $\{X_k\}$ be iid with mean $\mu$ and finite variance $\sigma^2$. Then, 
\[\frac{S_n}{n} \overset{a.s.}{\to} \mu\]
\end{corollary}

\begin{proof}
Let $\tilde{X}_k := X_k - \mu$ so that $\E \tilde{X}_k = 0$ and $\Var(\tilde{X}_k) = \sigma^2$. Then 
\[\sum_{k = 1}^{\infty} \frac{\Var(\tilde{X}_k)}{k^2} = \sigma^2 \sum_{k = 1}^{\infty} \frac{1}{k^2} < \infty\]
so we may apply the previous theorem to conclude 
\[\frac{1}{n} \sum_{k = 1}^{n} (X_k - \mu) = \frac{1}{n} \sum_{k = 1}^{n} \tilde{X}_k \overset{a.s.}{\to} \mu \]
which then implies 
\[\frac{S_n}{n} = \frac{1}{n} \sum_{k = 1}^{n} X_k \overset{a.s.}{\to} \mu \]
\end{proof}
Now for the full strong law, which provides implications in both directions and does not require an assumption of finite variance. 
\begin{thm}
Let $\{X_k\}$ be iid random variables. 
\begin{enumerate}
\item If $\E \abs{X_1} < \infty$ and $\E X_1 = \mu$, then $\frac{S_n}{n} \overset{a.s.}{\to} \mu$. 
\item If $\frac{S_n}{n} \overset{a.s.}{\to} c$ for some constant $c$, then $c = \E X_1$ and $\E \abs{X_1} < \infty$. 
\end{enumerate}
\end{thm}
I provide the general structure of the proof below, suppressing some of the details. 
\begin{proof}
$(\implies)$ Establishing the forward implication primarily comes down to an application of the Kolmogorov three series test. However, this test only guarantees almost sure convergence to some 
limiting random variable, so there is more work to be done to ensure the correct limit. The first goal will be to define suitable truncated random variables $Y_k$ and show that 
\[\sum_{k = 1}^{\infty} \frac{Y_k - \E Y_k}{k} \text{ converges a.s.}\]
which we will conclude by an application of the Kolmogorov three series test. Next we will show that this implies that the sequence of empirical averages of the $Y_k$ converge almost 
surely to $\mu$. But since the $X_k$ and $Y_k$ are convergence equivalent (as is established by the first series in the Kolmogorov three series theorem), this means that the 
empirical averages of the $X_k$ converge to the same limit, which is the strong law we seek. We begin with the application of the Kolmogorov three series theorem, tackling the three series
in turn. \textbf{Watch YouTube proof of SLLN instead.}
\begin{enumerate}
\item Show there exists an $A > 0$ such that $\sum_{k = 1}^{\infty} \Prob\left(\frac{\abs{\tilde{X}_k}}{k} > A \right) < \infty$. \\[.2cm]
The assumption that the $X_k$ are integrable takes care of this. Indeed, due to the tail probability formula for expectation we have 
\begin{align*}
\E\abs{X_1} < \infty &\iff \sum_{k = 1}^{\infty} \Prob(\abs{\tilde{X}_1} > k) < \infty \\
			      &\iff \sum_{k = 1}^{\infty} \Prob(\abs{\tilde{X}_k} > k) < \infty \\
			      &\iff \sum_{k = 1}^{\infty} \Prob\left(\frac{\abs{\tilde{X}_k}}{k} > 1 \right) < \infty
\end{align*}
where the second step is due to the iid assumption. Thus, the first condition holds with $A = 1$. 

\item Letting $Y_k := \tilde{X}_k \mathbbm{1}\left\{\frac{\tilde{X}_k}{k} \leq 1\right\}$, show $\sum_{k = 1}^{\infty} \Var\left(\frac{Y_k}{k}\right) < \infty$. \\[.2cm]
This can be shown using the facts that $\E \abs{Y_1} < \infty$ and the $Y_k$ are independent, but it is a bit technical so I will omit it here. 

\item  $\sum_{k = 1}^{\infty} \E \frac{Y_k}{k} < \infty$. \\[.2cm]
This is often the most difficult condition to check in the Kolmogorov three series test. 

\end{enumerate}
\end{proof}

\section{The Central Limit Theorem}
We have seen that under suitable assumptions, for iid random variables the empirical average $\overline{X}_n := \frac{S_n}{n}$ converges to the mean $\mu$ of the random variables. 
The law of large numbers is a qualitative statement; either $\overline{X}_n$ converges to $\mu$ or it doesn't. 
A natural question is then: can we make any probabilistic statements about \textit{how} $\overline{X}_n$ fluctuates about $\mu$? This is the realm of the Central Limit Theorem (CLT). 
Loosely speaking, the CLT states that as $n$ grows large, the distribution of $S_n$ gets closer and closer to a Gaussian. Naturally, the distribution of $S_n$ can shift and spread out 
as $n$ gets larger. Thus, by properly normalizing the distributions to have mean zero and unit variance, we can make the following statement: 
\[\frac{S_n - \E[S_n]}{\text{SD}(S_n)} = \frac{S_n - n\mu}{\sigma \sqrt{n}} \overset{d}{\to} N(0, 1)\]
Expanding the sum, we can equivalently write this as
\[\frac{1}{\sigma \sqrt{n}} \sum_{k = 1}^{n} (X_k - \mu) \overset{d}{\to} N(0, 1)\]
or in terms of the empirical mean as
\[\frac{\sqrt{n}}{\sigma} (\overline{X}_n - \mu) \overset{d}{\to} N(0, 1)\]
We will find that in addition to the finite mean assumption required by the SLLN, a finite variance assumption is necessary for the CLT to hold. 
In addition to this standard CLT for iid random variables, we will see generalizations to the case where the $X_k$ are independent, but not necessarily identically distributed. CLTs of 
this type will look like
\[\frac{1}{s_n} \sum_{k = 1}^{n} (X_k - \mu_k) \overset{d}{\to} N(0, 1)\]
where 
\[s_n^2 := \sum_{k = 1}^{n} \sigma_k^2 \]
In this more general setting, in addition to the variances $\sigma_k^2$ being finite, we will also require assumptions that essentially imply that in the limit, individual $X_k$ cannot exert
too much influence in determining the sums. Very informally, we require that the random variables ``mix together'' in the limit, so that they are all exchangeable in some sense. Clearly this 
holds in the case when they are iid, but the assumptions required are weaker than the identically distributed assumption. 

\subsection{The iid Case.}
\begin{thm}
Let $\{X_k\}$ be iid with finite mean $\mu$ and finite variance $\sigma^2 > 0$. Then, 
\[\frac{S_n - n\mu}{\sigma \sqrt{n}} \overset{d}{\to} N(0, 1)\]
\end{thm}

\subsection{The Non-Identically Distributed Case: Lindeberg-Levy-Feller}
We now consider dropping the assumption that the $X_k$ are identically distributed. The $X_k$ are now free to have different means $\mu_k$ and variances $\sigma_k^2$, both of which 
we assume are finite. In addition, throughout this section we assume that the $\sigma_k^2$ are not all zero, which excludes this uninteresting degenerate case. We will also make use of the 
notation 
\[s_n^2 := \Var(S_n) = \sum_{k = 1}^{n} \sigma_k^2\]
The CLT in the iid case showed that in the limit, no matter what the distribution of the underlying $X_k$, the fluctuations mix together in such a way that the distribution of the sum $S_n$ 
approaches a Gaussian. Relaxing the identically distributed assumption, we now must identify the conditions that allow the $X_k$ to mix together in this way. The two central conditions are 
called the \textit{Lindeberg Conditions}. 
\begin{enumerate}
\item $L_1(n) := \max_{1 \leq k \leq n} \frac{\sigma_k^2}{s_n^2} = \frac{\max_{1 \leq k \leq n} \sigma_k^2}{\sum_{k = 1}^{n} \sigma_k^2} \to 0$
\item $L_2(n) := \frac{1}{s_n^2} \sum_{k = 1}^{n} \E\left[(X_k - \mu_k)^2 \mathbbm{1}\{\abs{X_k - \mu_k} > \epsilon s_n\} \right] \to 0$
\end{enumerate} 
These are not the most intuitive things in the world. For the first Lindeberg condition, we see that $L_1(n)$ considers the most dominant $X_k$ out of the first $n$ random variables in the
sequence, in the sense of having the largest variance. $L_1(n)$ is the proportion of the variance of this dominant random variable over the variance of the sum $S_n$. The first condition thus 
says that in the limit, this dominance proportion must go to $0$. In a certain sense this means that the variability in any one of the $X_k$ becomes negligible in the limit. This second condition 
appears a bit more difficult to parse. Looking inside the indicator function, $\mathbbm{1}\{\abs{X_k - \mu_k} > \epsilon s_n\}$, note that $s_n$ is the typical deviation of the sum $S_n$ 
from its mean. Taking $\epsilon = 1$ for convenience for the moment, the indicator function is thus considering the event where the deviation of $X_k$ from its mean is larger than the typical 
deviation of $S_n$ from its mean. The expectation is then considering the average of square of these ``unusual'' deviations. Then the expected squared unusual deviations for each of the 
first $n$ random variables in the sequence are summed up and this value is compared to the variance of the sum $s_n^2$. The second condition is thus requiring that in the limit, such exceptional 
deviations of the $X_k$ from their means must become negligible compared to the variance of the sum $s_n^2$. 

The two Lindeberg conditions are not necessary and sufficient for the CLT to hold, but they're pretty close. It actually turns out that the second Lindeberg condition implies the first. 
The actual statement of the theorem looks kind of like and if and only if, but is a bit weird due to the possibility of some odd edge cases. We will explore these weird cases a bit after
the statement of the theorem. 
\begin{thm}
Let $\{X_k\}$ be independent random variables with finite means $\mu_k$ and finite variances $\sigma_k^2$ not all equal to zero. 
\begin{enumerate}
\item If the second Lindeberg condition holds, then so does the first, and 
\[\frac{1}{s_n} \sum_{k = 1}^{n} (X_k - \mu_k) \overset{d}{\to} N(0, 1)\]
\item If the first Lindeberg condition holds and the CLT holds, then the second Lindeberg condition holds. 
\end{enumerate}
\end{thm}
What should be immediately clear is that the second Lindeberg condition is sufficient for the CLT to hold. However, the statement of the converse leaves open the possibility that 
it is not strictly necessary; i.e. there are weird cases where the CLT holds despite the second Lindeberg condition failing. The first Lindeberg condition excludes these weird cases. To better
understand this, let's consider some implications of the first Lindeberg condition. Clearly $L_1(n) \to 0$ implies $s_n^2 \to \infty$. If the first Lindeberg condition doesn't hold, this leaves open 
the possibility that $s_n^2$ converges; that is, 
\[\sum_{k = 1}^{\infty} \sigma_k^2 < \infty\]
If this is the case, then by the Kolmogorov convergence criterion (\ref{Kolmogorov_convergence_criterion}), we know that $S_n$ converges almost surely. Thus, 
$S_n - \E S_n = \sum_{k = 1}^{n} (X_k - \mu_k)$ converges almost surely \textit{without normalization}. 

\subsection{The Lyapounov CLT}
The second Lindeberg condition is difficult to check. Lyapounov's condition is a stronger sufficient condition that is often much easier to check in practice. Lyapounov's condition essentially 
requires that any moment higher than the second moment exists (it is common to use the third moment), and then limits the growth rate of these moments in an average sense. 
\begin{thm}
Let $\{X_k\}$ be independent random variables with finite means $\mu_k$ and finite variances $\sigma_k^2$ not all equal to zero. Also assume that for some $\delta > 0$, 
$\E \abs{X_k}^{2 + \delta} < \infty$ for all $k$. If 
\[\frac{1}{s_n^{2 + \delta}} \sum_{k = 1}^{n} \E\left[\abs{X_k - \mu_k}^{2 + \delta}\right] \to 0\]
then the CLT holds. 
\end{thm}
There are cases where the Lyapounov condition doesn't hold, but the Lindeberg condition (and hence the CLT) does, but in general the Lyapounov condition is very useful in practice. 


\end{document}









