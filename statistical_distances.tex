\documentclass[12pt]{article}
\RequirePackage[l2tabu, orthodox]{nag}
\usepackage[main=english]{babel}
\usepackage[rm={lining,tabular},sf={lining,tabular},tt={lining,tabular,monowidth}]{cfr-lm}
\usepackage{amsthm,amssymb,latexsym,gensymb,mathtools,mathrsfs}
\usepackage[T1]{fontenc}
\usepackage[utf8]{inputenc}
\usepackage[pdftex]{graphicx}
\usepackage{epstopdf,enumitem,microtype,dcolumn,booktabs,hyperref,url,fancyhdr}
\usepackage{algorithmic}
\usepackage[ruled,vlined,commentsnumbered,titlenotnumbered]{algorithm2e}

% Plotting
\usepackage{pgfplots}
\usepackage{xinttools} % for the \xintFor***
\usepgfplotslibrary{fillbetween}
\pgfplotsset{compat=1.8}
\usepackage{tikz}

% Custom Commands
\newcommand*{\norm}[1]{\left\lVert#1\right\rVert}
\newcommand*{\abs}[1]{\left\lvert#1\right\rvert}
\newcommand*{\suchthat}{\,\mathrel{\big|}\,}
\newcommand{\E}{\mathbb{E}}
\newcommand{\Var}{\mathrm{Var}}
\newcommand{\R}{\mathbb{R}}
\newcommand{\N}{\mathcal{N}}
\newcommand{\Ker}{\mathrm{Ker}}
\newcommand{\Cov}{\mathrm{Cov}}
\newcommand{\Prob}{\mathbb{P}}
\DeclarePairedDelimiterX\innerp[2]{(}{)}{#1\delimsize\vert\mathopen{}#2}
\DeclareMathOperator*{\argmax}{argmax}
\DeclareMathOperator*{\argmin}{argmin}
\DeclarePairedDelimiter{\ceil}{\lceil}{\rceil}

% Specific to this document. 
\newcommand{\rVec}{\mathbf{X}}
\newcommand{\rVecTwo}{\mathbf{Y}}
\newcommand{\bx}{\mathbf{x}}
\newcommand{\dist}{P}
\newcommand{\distApprox}{Q}

\setlist{topsep=1ex,parsep=1ex,itemsep=0ex}
\setlist[1]{leftmargin=\parindent}
\setlist[enumerate,1]{label=\arabic*.,ref=\arabic*}
\setlist[enumerate,2]{label=(\alph*),ref=(\alph*)}

% For embedding images
\graphicspath{ {./images/} }

% Specifically for paper formatting 
\renewcommand{\baselinestretch}{1.2} % Spaces manuscript for easy reading

% Formatting definitions, propositions, etc. 
\newtheorem{definition}{Definition}
\newtheorem{prop}{Proposition}
\newtheorem{lemma}{Lemma}
\newtheorem{thm}{Theorem}
\newtheorem{corollary}{Corollary}

% Title and author
\title{Notions of Distance Between Probability Measures}
\author{Andrew Roberts}

\begin{document}

\maketitle
\tableofcontents
\newpage

\section{Introduction}
In this document I consider notions of distance (loosely speaking) or ``discrepancy'' between two probability measures $P$ and $Q$, each defined over the same measurable space $(\Omega, \mathcal{F})$. I will use the notation 
$p$ and $q$ to denote the densities (i.e. Radon-Nikodym derivatives) of the respective measures with respect to the Lebesgue measure $\lambda$. I will typically present the notions of discrepancy in their general, 
measure-theoretic formulation, but will primarily focus the discussions on the case where densities exist. We suppose these densities are defined over some subset $\mathcal{D} \subset \R^D$. 

\textbf{TODO:} need to update this; in the discussions below I often consider $P$ and $Q$ being distributions over $\mathcal{D}$, not $\Omega$. 

It will often be useful to think of $P$ as the ``true distribution'' and $Q$ as an 
approximation to $P$, although this distinction will not always be relevant. Some of the distances considered here will be legitimate distances, meaning they satisfy the mathematical definition of a distance \textit{metric}. 
\begin{definition}
TODO
\end{definition}
Others may violate this definition, often by failing to satisfy symmetry or the triangle inequality. Given that I have chosen to designate $P$ as some true distribution, and $Q$ as an approximation to $Q$, then it is not surprising 
that this asymmetry would yield reasonable notions of discrepancy that violate symmetry. 

It is common to abstract away the specifics of different statistical distances, and view them as special cases of more generic notions. To this end, the notion of an \textit{f-divergence} was defined, which encompasses almost 
all of the distances considered here. An alternative route is to consider this distances to be defined as the supremum of differences in expectations of test functions, where choosing different families of test functions results 
in different distances. Metrics defined in this way are called \textit{integral probability metrics}. Both of these abstractions are discussed in detail below. 

\section{Total Variation Distance}
Total variation distance (TV) is a natural candidate to start with, and we will use it to motivate some of the other distances. The TV distance simply asks the question: what is the largest difference in 
probabilities that can occur when measuring the probability according to $P$ versus $Q$? This is quite natural. One nice way to think about this is to sample from the 
two distributions. If $P$ and $Q$ are very similar then we expect the samples generated from each distribution to be almost indistinguishable. On the other hand, if they are completely 
different then we might only need a few samples to distinguish between them with high confidence. The TV distance considers the ``worst-case'' event $A \in \mathcal{F}$, in the sense that $A$ is the event 
on which $P$ and $Q$ disagree the most. Mathematically, the definition is given by 
\begin{align*}
\text{TV}(P, Q) := \sup_{A \in \mathcal{F}} \abs{P(A) - Q(A)}
\end{align*}
where the supremum is taken over all measurable sets $A$. Since $P(A)$ and $Q(A)$ take values in $[0, 1]$ then 
\[0 \leq \text{TV}(P, Q) \leq 1\]
If the distributions have densities $p$ and $q$ with respect to the Lebesgue measure $\lambda$ then this can be 
written 
\begin{align*}
\text{TV}(P, Q) = \sup_{A} \abs{\int_A \left[p(\theta) - q(\theta)\right] d\theta }
\end{align*}
We can immediately establish a connection with $L_1$ distance. 
\begin{align*}
\text{TV}(P, Q) &= \sup_{A} \abs{\int_A \left[p(\theta) - q(\theta)\right] d\theta } \\
				 &\leq \sup_{A} \int_A \abs{p(\theta) - q(\theta)} d\theta \\
				 &\leq \sup_{A} \int \abs{p(\theta) - q(\theta)} d\theta \\
				 &= \int \abs{p(\theta) - q(\theta)} d\theta \\
				 &= \norm{p - q}_{L_1^\lambda}
\end{align*}
so the TV distance between the measures $P$ and $Q$ is no larger than the $L_1$ distance between their densities $p$ and $q$. Trying to establish the opposite 
inequality is tricky since we need to bring the absolute value outside of the integral. The trick is to split the integral into two parts to avoid absolute values. To this end, let 
$B := \{\theta \in \mathcal{D}:p(\theta) \geq q(\theta)\}$. Then 
\begin{align*}
\norm{p - q}_{L_1^\lambda} &= \int \abs{p(\theta) - q(\theta)} d\theta \\
						      &= \int_B \left(p(\theta) - q(\theta)\right) d\theta +  \int_B \left(q(\theta)  - p(\theta)\right) d\theta \\
						      &\leq \sup_A \int_A \left(p(\theta) - q(\theta)\right) d\theta + \sup_A \int_A \left(q(\theta)  - p(\theta)\right) d\theta \\
						      &\leq \sup_A \abs{\int_A \left(p(\theta) - q(\theta)\right) d\theta} + \sup_A \abs{\int_A \left(q(\theta)  - p(\theta)\right) d\theta} \\
						      &= 2 \sup_A \abs{\int_A \left(p(\theta) - q(\theta)\right) d\theta} \\
						      &= 2 \cdot \text{TV}(p, q)
\end{align*}
So we have shown 
\begin{align*}
\text{TV}(p, q) &\leq \norm{p - q}_{L_1^\lambda} \\
\frac{1}{2} \norm{p - q}_{L_1^\lambda} &\leq \text{TV}(p, q)
\end{align*}
This might make us wonder if the constant in the first inequality is $\frac{1}{2}$ so that we have the exact equivalence 
\begin{align*}
\text{TV}(p, q) &= \frac{1}{2} \norm{p - q}_{L_1^\lambda}
\end{align*}
Indeed this is the case. To show this we will need to be a bit more careful with the first bound. To start, note that 
\begin{align*}
\int_\mathcal{D} \left(p(\theta) - q(\theta)\right) d\theta &= \int_B \left(p(\theta) - q(\theta)\right) d\theta + \int_{\mathcal{D} \setminus B} \left(p(\theta) - q(\theta)\right) d\theta \\
											     &=  \int_B \left(p(\theta) - q(\theta)\right) d\theta - \int_{\mathcal{D} \setminus B} \left(q(\theta) - p(\theta)\right) d\theta \\
\end{align*}
But the lefthand side is $0$ since $p$ and $q$ are densities, so 
\begin{align*}
\int_B \left(p(\theta) - q(\theta)\right) d\theta &= \int_{\mathcal{D} \setminus B} \left(q(\theta) - p(\theta)\right) d\theta
\end{align*}
a fact which we will exploit below. Now let $A$ be any measurable set. It follows that 
\begin{align*}
\abs{\int_A \left(p(\theta) - q(\theta)\right) d\theta} &= \max\left\{\int_A \left(p(\theta) - q(\theta)\right) d\theta, -\int_A \left(p(\theta) - q(\theta)\right) d\theta \right\} \\
										      &=  \max\left\{\int_A \left(p(\theta) - q(\theta)\right) d\theta, \int_A \left(q(\theta) - p(\theta)\right) d\theta \right\} \\
										      &\leq \max\left\{\int_{A \cap B} \left(p(\theta) - q(\theta)\right) d\theta, \int_{A \cap (\mathcal{D} \setminus B)} \left(q(\theta) - p(\theta)\right) d\theta \right\} \\
										      &\leq \max\left\{\int_{B} \left(p(\theta) - q(\theta)\right) d\theta, \int_{\mathcal{D} \setminus B} \left(q(\theta) - p(\theta)\right) d\theta \right\} \\
										      &= \int_{B} \left(p(\theta) - q(\theta)\right) d\theta \\
										      &= \frac{1}{2} \left[\int_{B} \left(p(\theta) - q(\theta)\right) d\theta + \int_{\mathcal{D} \setminus B} \left(q(\theta) - p(\theta) \right) d\theta \right] \\
										      &= \frac{1}{2} \left[\int_{B} \abs{p(\theta) - q(\theta)} d\theta + \int_{\mathcal{D} \setminus B} \abs{p(\theta) - q(\theta)} d\theta \right] \\
										      &= \frac{1}{2} \int_{\mathcal{D}} \abs{p(\theta) - q(\theta)} d\theta \\
										      &= \frac{1}{2} \norm{p - q}_{L_1^{\lambda}(\mathcal{D})}
\end{align*}
The first inequality follows from the fact that intersecting the domain of integration of the first integral with $B$ removes values of $\theta$ which results in a negative integrand, and hence the 
integral gets larger. The same reasoning applies to the second integral. The fifth and seventh lines both use the fact that was proved above. This inequality holds for every measurable set $A$ and thus 
\begin{align*}
\text{TV}(p, q) &= \sup_A \abs{\int_A \left(p(\theta) - q(\theta)\right) d\theta} \leq \frac{1}{2} \norm{p - q}_{L_1^{\lambda}(\mathcal{D})}
\end{align*}
which establishes the reverse inequality and thus verifies 
\begin{align*}
\text{TV}(p, q) &= \frac{1}{2} \norm{p - q}_{L_1^\lambda}
\end{align*}
Recalling that $\text{TV}(p, q)$ takes values in $[0, 1]$, then we see that 
\[0 \leq \norm{p - q}_{L_1^\lambda} \leq 2\]
so the two distances $\text{TV}(p, q)$ and $\norm{p - q}_{L_1^\lambda}$ are measuring the same exact thing, just on a different scale. 

To review, we started with the very natural definition of TV distance as the largest possible difference in probabilities that the two measures could assign to any given set $A$. We then showed that this notion could be equivalently defined (up to a multiplicative constant) as the $L_1$ distance between the densities of the measures. One way to see that this makes sense is to consider that the density functions $p$ and $q$ are unit vectors with respect to $\norm{\cdot}_{L_1^\lambda}$ (since they have to integrate to $1$ by definition). Therefore, the distance 
$\norm{p - q}_{L_1^\lambda}$ is not affected by ``size'' of $p$ or $q$. We would not, for example, want to calculate the $L_1$ distance between unnormalized densities, since arbitrarily scaling the densities would change the notion of distance, even though the underlying probability distribution would remain the same. We formalize this notion of invariance below. 
\begin{prop}
Let $T: \mathcal{D} \to \mathcal{D}$ be a smooth, one-to-one mapping. Consider random variables $X_P$, $X_Q$ with distributions $P, Q$ and corresponding densities $p, q$. Define the random variables $Y_P = T(X_P)$ and $Y_Q = T(X_Q)$ and 
let $\tilde{p}, \tilde{q}$ be their respective densities. Then, 
\[\norm{p - q}_{L_1^{\lambda}} = \norm{\tilde{p} - \tilde{q}}_{L_1^{\lambda}}\]
\end{prop}

\begin{proof}
\textbf{TODO}
\end{proof}

\subsection{Variational Characterization and Bounding Differences in Expectations}
Here we derive a so-called \textit{variational} characterization of TV distance, where $d_{TV}$ is shown to be equivalent to the supremum taken over a set of test functions of the difference in expectations. 
We begin by establishing a bound that states that if two distributions are close in TV distance, then expectations computed with respect to the distributions will also be close. 
\begin{lemma} 
Let $f$ be a bounded function (i.e. $\norm{f}_\infty = \sup_{\theta} \abs{f(\theta)} < \infty$). Then, 
\[\abs{\E_p(f) - \E_q(f)} \leq 2\norm{f}_\infty d_{\text{TV}}(p, q)\]
\end{lemma}

\begin{proof}
\begin{align*}
\abs{\E_p(f) - \E_q(f)} &= \abs{\int f(\theta)[p(\theta) - q(\theta)] d\theta} \\
				 &\leq \int \abs{f(\theta)} \abs{p(\theta) - q(\theta)} d\theta \\
				 &\leq \norm{f}_{\infty} \int  \abs{p(\theta) - q(\theta)} d\theta \\
				 &\leq 2\norm{f}_{\infty} d_{TV}(p, q)
\end{align*}
\end{proof}

Thus, small TV distance between distributions implies small distance between expectations calculated using those distributions. 
However, it is important to note that the result only holds for bounded functions $f$, which represents a very strong assumption. For example, if we wanted to compare 
the means of the two distributions, this would correspond to choosing $f(\theta) = \theta$, which is unbounded and thus excluded from the previous result. Note also, that it 
is quite natural that the bounds scales with $\norm{f}_{\infty}$. ``Bigger'' $f$ imply larger expectations, and hence naturally the absolute deviations between the two expectations can 
be larger. We can also think of this bound as stating that the difference in expectations, normalized by the size of $f$, has the upper bound
\[\frac{\abs{\E_p(f) - \E_q(f)} }{\norm{f}_\infty} \leq 2d_{TV}(p, q)\]

\noindent
Now for the variational characterization. 
\begin{thm} 
\[d_{TV}(p, q) = \frac{1}{2} \sup_{\norm{f}_\infty \leq 1} \abs{\E_p(f) - \E_q(f)}\]
\end{thm}

\begin{proof}
If $\norm{f}_\infty \leq 1$, then the above result shows
\[\frac{1}{2} \abs{\E_p(f) - \E_q(f)} \leq d_{TV}(p, q)\]
which implies 
\[\frac{1}{2} \sup_{\norm{f}_\infty \leq 1} \abs{\E_p(f) - \E_q(f)} \leq d_{TV}(p, q)\]
To show the reverse inequality (and hence the desired inequality) it suffices to show the existence of an $f$ with $\norm{f}_\infty = 1$ that satisfies 
\[\frac{1}{2} \abs{\E_p(f) - \E_q(f)} = d_{TV}(p, q)\]
Notice that 
\[\E_p(f) - \E_q(f) = \int f(\theta)\left(p(\theta) - q(\theta) \right) d\theta\]
so the strategy is to choose $f$ such that $f(\theta)$ matches the sign of $p(\theta) - q(\theta)$, hence hence making the integrand as large as possible. 
To this end, let $f(\theta) := \text{sign}\left(p(\theta) - q(\theta)\right)$. Then 
\[\E_p(f) - \E_q(f) = \int f(\theta)\left(p(\theta) - q(\theta) \right) d\theta = \int \abs{p(\theta) - q(\theta)} d\theta = 2d_{TV}(p, q)\]
This achieves the desired inequality and hence proves the result. 
\end{proof}

\section{Hellinger Distance}
The Hellinger distance is perhaps the most popularly used measure of distance between probability distributions in the Bayesian inverse problem literature, for reasons that will be discussed below. I will motivate this notion of distance using the $\text{TV}$/$L_1$ notions discussed previously. We established that $L_1$ notions of distance between probability densities yields a reasonable interpretation of the distance between two probability measures. It is natural to wonder: would replacing $L_1$ with $L_2$ also give us something useful? In short, not really. As mentioned in the previous section, the $L_1$ notion was reasonable since $p$ and $q$ are unit vectors with respect to $L_1$. However, $p$ and $q$ are \textit{not} unit vectors with respect to $L_2$. We see that 
\[\norm{p - q}_{L_2^\lambda} = \int (p(\theta) - q(\theta))^2 d\theta = \int p^2(\theta) d\theta + \int q^2(\theta) d\theta - 2\int p(\theta)q(\theta) d\theta\]
so changes in the first two terms will affect the distance, which is not a desirable property. But what if we just dropped those terms and considered a our notion of distance to be the inner product 
\[- 2\int p(\theta)q(\theta) d\theta\]
But this still doesn't make sense at all. Consider the two cases
\begin{enumerate}
\item $p(0) = \frac{1}{10} \text{ and } q(0) = \frac{2}{10}$
\item $p(0) = \frac{9}{10} \text{ and } q(0) = 1$
\end{enumerate}
If we sampled from these distributions, then in the first case we would expect to observe the sample $0$ about twice as much from the second distribution compared to the first, while $0$ would appear with pretty similar frequency in the second example. The relevant terms in the inner product would be $\frac{1}{10} \cdot \frac{2}{10} = \frac{2}{100}$ and $\frac{9}{10} \cdot 1 = \frac{90}{100}$, so we would actually be \textit{penalizing} the second distribution for being closer, the opposite of what we want. To flip this around, it seems we would actually want to consider the quotient $\frac{p(\theta)}{q(\theta)}$ rather than the product of the densities, which is exactly what many notions of distance between probability measures considers (e.g. KL divergence). Obviously considering the quotient instead abandons the idea of $L_2$ distance. However, it turns out that with a slight tweak we can still define an $L_2$ measure that actually makes sense. 
\textbf{Also discuss difference vs quotient wrt normalizing constants:} e.g. $\abs{\frac{p(\theta)}{Z_1} - \frac{q(\theta)}{Z_2}}$ (norm constants affect results). vs. 
$\abs{\frac{\frac{p(\theta)}{Z_1}}{\frac{q(\theta)}{Z_2}}} = \frac{Z_2}{Z_1} \frac{p(\theta)}{q(\theta)}$ so norm constants will just scale the integral which is fine. 

The idea is to transform the densities $p$ and $q$ so that they actually are unit vectors with respect to $L_2$. The transformation that accomplishes this is the square root function, since 
\[\norm{\sqrt{p}}^2_{L_2^\lambda} = \int \sqrt{p(\theta)}^2 d\theta =  \int p(\theta) d\theta = 1 \]
Applying this idea yields the Hellinger distance:
\begin{align}
d_H(P, Q) = \left(\frac{1}{2} \int \abs{\sqrt{p(\theta)} - \sqrt{q(\theta)}}^2 d\theta \right)^{1/2} = \frac{1}{\sqrt{2}} \norm{\sqrt{p} - \sqrt{q}}_{L_2^\lambda}
\end{align} 
As we will show below, the scaling by $\frac{1}{2}$ ensures that 
\[0 \leq d_H(P, Q) \leq 1\]
which means it is on the same scale as $d_{TV}(P, Q)$. 

\begin{prop}
The Hellinger distance satisfies 
\[0 \leq d_H(P, Q) \leq 1\]
Moreover,  $d_H(P, Q) = 0$ if and only if the supports of $P$ and $Q$ are disjoint.  
\end{prop}

\begin{proof}
Clearly, $d_H(P, Q) \geq 0$. For the upper bound, we have 
\begin{align*}
2d^2_H(P, Q) &= \int \abs{\sqrt{p(\theta)} - \sqrt{q(\theta)}}^2 d\theta \\
		      &= \int \left[ p(\theta) + q(\theta) - 2\sqrt{p(\theta) q(\theta)} \right] d\theta \\
		      &\leq \int \left[ p(\theta) + q(\theta) \right] d\theta \\
		      &= 2
\end{align*} 
which implies $d_H(P, Q) \leq 1$. Notice that the equality holds iff
\[\int \sqrt{p(\theta) q(\theta)} d\theta = 0\]
If there is a measurable set $A \subset \mathcal{D}$ satisfying $P(A) > 0$ and $Q(A) > 0$ then the above integral will be strictly positive. 
Thus, it must be the case that $P(A) = 0$ or $Q(A) = 0$ for all $A$; that is, the supports of the two distributions must be disjoint. 
\end{proof}

\begin{prop} 
Let $\mathcal{H}$ be the space of probability measures on $\mathcal{D}$. Then $d_H(\cdot, \cdot): \mathcal{H} \times \mathcal{H} \to [0, 1]$ forms a bounded metric on this space. 
\end{prop}

\begin{proof}
From the previous proposition we have $d_H(P, Q) \geq 0$ and already showed that $d_H$ is bounded. Clearly, $d_H(P, Q) = d_H(Q, P)$ and 
$d_H(P, Q) = 0 \iff P = Q$. \textbf{TODO: } show triangle inequality. 
\end{proof}

\subsection{Equivalence of Hellinger and Total Variation Distance}
We now prove bounds relating TV and Hellinger distance, showing they are \textit{equivalent} metrics (in the topological sense). 

\begin{lemma} 
For densities $p$ and $q$, 
\[\frac{1}{\sqrt{2}} d_{TV}(p, q) \leq d_H(p, q) \leq \sqrt{d_{TV}(p, q)}\]
\end{lemma}

\begin{proof}
For the lower bound on $d_H$, consider 
\begin{align*}
d_{TV}(p, q) &= \frac{1}{2} \int \abs{p(\theta) - q(\theta)} d\theta \\
		    &= \frac{1}{2} \int \abs{\left(\sqrt{p(\theta)} - \sqrt{q(\theta)}\right) \left(\sqrt{p(\theta)} + \sqrt{q(\theta)}\right)} d\theta \\
		    &= \frac{1}{2} \int \langle \abs{\sqrt{p} - \sqrt{q}}, \abs{\sqrt{p} + \sqrt{q}} \rangle_{L_2^\lambda} \\
		    &\leq \frac{1}{\sqrt{2}} \norm{\sqrt{p} - \sqrt{q}}_{L_2^\lambda} \cdot \frac{1}{\sqrt{2}} \norm{\sqrt{p} + \sqrt{q}}_{L_2^\lambda} && \text{Cauchy-Schwarz} \\
		    &= d_H(p, q) \cdot \frac{1}{\sqrt{2}} \norm{\sqrt{p} + \sqrt{q}}_{L_2^\lambda} 
\end{align*}
To bound the second term, we expand the squared sum and apply the AM-GM inequality. 
\begin{align*}
 \norm{\sqrt{p} + \sqrt{q}}_{L_2^\lambda}  &= \left[ \int (\sqrt{p} + \sqrt{q})^2 d\theta  \right]^{1/2} \\
 							        &=  \left[ 2 + 2\int \sqrt{p(\theta)q(\theta)}  d\theta  \right]^{1/2} \\
							        &\leq \left[ 2 + \int (p(\theta) + q(\theta)) d\theta  \right]^{1/2} && \text{AM-GM} \\
							        &= \left[2 +  2 \right]^{1/2} = 2
\end{align*}
Plugging this back in, we obtain
\[d_{TV}(p, q) \leq \sqrt{2} d_H(p, q)\]
For the other bound, consider 
\begin{align*}
d^2_{H}(p, q) &= \frac{1}{2} \int \abs{\sqrt{p(\theta)} - \sqrt{q(\theta)}}^2 d\theta \\
		     &=  \frac{1}{2} \int \abs{\sqrt{p(\theta)} - \sqrt{q(\theta)}} \abs{\sqrt{p(\theta)} - \sqrt{q(\theta)}} d\theta \\
		     &\leq \frac{1}{2} \int \abs{\sqrt{p(\theta)} - \sqrt{q(\theta)}} \abs{\sqrt{p(\theta)} + \sqrt{q(\theta)}} d\theta  && \left(\abs{\sqrt{p(\theta)} - \sqrt{q(\theta)}}  \leq \abs{\sqrt{p(\theta)} + \sqrt{q(\theta)}}\right) \\
		     &\leq \frac{1}{2} \int \abs{p(\theta) - q(\theta)} d\theta \\
		     &= d_{TV}(p, q)
\end{align*}
\end{proof}

\subsection{Bounding Differences in Expectations}
We prove an analog of the result for TV distance, in which closeness of the distributions implies closeness of expectations computed using those distributions. The 
result below establishes such a result for the class of functions $f$ that have finite second moments with respect to both distributions. This result applies much more broadly 
than the analog for TV distance, which only applied to the class of bounded functions $f$. Recall that this restriction excluded, in particular, the function $f(\theta) = \theta$, which 
is the function used for a comparison of means of the two distributions. In the Hellinger result, this function will be included so long as the two distributions in question have 
finite variance. 

\begin{lemma} 
Suppose $f$ satisfies $C := \sqrt{\E_p\abs{f}^2 + \E_q\abs{f}^2} < \infty$. Then 
\[\abs{\E_p(f) - \E_q(f)} \leq 2C d_H(p, q)\]
\end{lemma}

\begin{proof}
We utilize the standard strategy for these sort of Hellinger proofs--writing $p(\theta) - q(\theta) = \left(\sqrt{p(\theta)} + \sqrt{q(\theta)} \right) \left(\sqrt{p(\theta)} - \sqrt{q(\theta)} \right)$ and then applying 
the Cauchy-Schwarz inequality. 
\begin{align*}
\abs{\E_p(f) - \E_q(f)}^2 &= \abs{\int f(\theta) \left(p(\theta) - q(\theta)\right) d\theta}^2 \\
				     &= \abs{\int f(\theta) \left(\sqrt{p(\theta)} + \sqrt{q(\theta)} \right) \left(\sqrt{p(\theta)} - \sqrt{q(\theta)} \right) d\theta}^2 \\
				     &\leq \norm{\sqrt{p} - \sqrt{q}}_{L_2^\lambda}^2 \norm{(\sqrt{p} + \sqrt{q})f}_{L_2^\lambda}^2 \\
				     &= d^2_H(p, q) \norm{(\sqrt{p} + \sqrt{q})f}_{L_2^\lambda}^2
\end{align*}

In order to establish the bound on $\norm{(\sqrt{p} + \sqrt{q})f}_{L_2^\lambda}^2$, first note that 
\[\left(\sqrt{p(\theta)} + \sqrt{q(\theta)}\right)^2 = p(\theta) + q(\theta) + 2\sqrt{p(\theta)q(\theta)} \leq p(\theta) + q(\theta) + 2\frac{p(\theta) + q(\theta)}{2} = 2\left[p(\theta) + q(\theta) \right]\]
where the inequality applies the AM-GM inequality. We thus have, 
\begin{align*}
\norm{(\sqrt{p} + \sqrt{q})f}_{L_2^\lambda}^2 &= \int \left(\sqrt{p(\theta) }+ \sqrt{q(\theta)}\right)^2 f(\theta)^2 d\theta \\
								    &\leq 2 \int \left[p(\theta) + q(\theta) \right] f(\theta)^2 d\theta \\
								    &= 2 \left[\E_p\abs{f}^2 + \E_q\abs{f}^2 \right] \\
								    &= 2C^2
\end{align*}
Taking square roots then yields the desired inequality. \textbf{TODO:} I'm getting factor of $\sqrt{2}$ in my bound, rather than 2 as in the book. 
\end{proof}

Note that the analogous constant $C$ in the TV case is $\norm{f}_\infty$. In this case, $C$ is a sort of sum of the $L_2$ size of $f$ under the two distributions. 




\section{Kullback-Leibler }
\begin{itemize}
\item Show equivalence to MLE
\item Show sensitivity to tails 
\end{itemize}

\section{Scoring Rules}

\subsection{Introduction to Scoring Rules}
We now consider the related topic of evaluating probabilistic forecasts. This is a tricky business; for example, if a weather forecasters predicts a 10\% chance of rain tomorrow and then it rains, did the forecaster do a poor 
job or did they just get unlucky. After all, they did say there was a non-zero chance of rain. Another application arises from statistical and machine learning models whose predictions are themselves distributions. For example, 
the prediction of a Gaussian process (GP) at some test point is a Gaussian distribution. We could ignore the distribution and simply measure the error of the prediction relative to the mean of the GP, but this would fail to take 
into account much of the information in a GP prediction. How to incorporate the whole distribution into a measure of error is not at all obvious. 

The toolkit that has been developed to deal with situations such as these are known as \textit{scoring rules}. Loosely speaking, a scoring rule is a function that takes as arguments 1.) a distribution (the forecaster's prediction) and 
2.) the outcome that actually occurred and it returns a value that can be thought of as ``reward'' or ``utility''. A higher value should mean the forecaster did a better job. Although much of the early literature on scoring rules was focused 
on the case of discrete distributions, we only focus on the continuous case here. In particular, we will consider a set $\mathcal{P}$ of possible probability densities over a space $\Omega$. The forecaster makes a prediction $p \in \mathcal{P}$ which we will 
refer to as the \textit{predictive density}. For example, this could be a GP predictive density. If the true realized outcome is $x \in \Omega$ then the scoring rule compares the predictive density $p$ with the outcome $x$ and returns a score. More formally, 
a scoring rule is a function
\[S: \mathcal{P} \times \Omega \to \R \cup \{-\infty, +\infty\}\]
meaning that the forecaster would be assigned score $S(p, x)$. An intuitive first guess at a good scoring rule might be 
\[S(p, x) = p(x)\]
After all, if the forecaster assigned high probability density $p(x)$ to the realized outcome $x$ then it should be rewarded, and vice versa. It turns out, as we will see later, that despite its intuitive appeal this so-called \textit{linear scoring rule} 
actually has some undesirable properties. Before discussing this, we discuss the basic properties that we would like a scoring rule to satisfy. 

\subsection{(Strictly) Proper Scoring Rules}
If we imagine that we are actually compensating a forecaster based on their evaluation by the criterion $S(p, x)$ then we want to ensure that the incentives of the scoring rule are actually encouraging the behavior we would want. 
In particular, if the forecaster believes the true distribution is $p$ then the scoring rule should incentivize them to actually report this distribution. What we don't want is a scoring rule that can be gamed by the forecaster; i.e. a rule that 
incentivizes them to report some other distribution $q \in \mathcal{P}$ that does not align with their true beliefs, but will nonetheless allow them to yield more reward. In order to define this more formally, we need to define what objective 
function the forecaster is actually trying to maximize. We suppose the forecaster cares about their expected score 
\[\max_{q \in \mathcal{P}}\E_{x \sim p}\left[S(q, x) \right] \] 
where the expectation is taken with respect to the forecaster's true belief $p$. We want to design rules $S(p, x)$ that encourage honesty
\[p = \text{argmax}_{q \in \mathcal{P}}\E_{x \sim p}\left[S(q, x) \right] \]
Such scoring rules are known as \textit{strictly proper}. 

\begin{definition}
A scoring rule $S(p, x)$ is called \textbf{proper} if 
\[\E_{x \sim p}\left[S(p, x) \right] \geq \E_{x \sim p}\left[S(q, x) \right] \text{ for all } q  \in \mathcal{P}\]
It is called \textbf{strictly proper} if equality holds only when $q = p$. 
\end{definition}

It turns out that our intuitively appealing linear scoring rule $S(p, x) = p(x)$ is not even proper, let alone strictly proper. With respect to the goal of maximizing expected score, it encourages forecasters to report dishonest forecasts, 
by inflating the probability density at the modes of their true belief $p$. In the discrete case of distributions over finitely many outcomes, the forecaster's best strategy is actually to place all of the probability mass on the outcome that 
they think is most likely, regardless of their true beliefs. 

\subsection{Logarithmic Scoring Rule}

\subsection{Continuous Ranked Probability Score}
The continuous ranked probability score (CRPS) is a popular strictly proper scoring rule that applies to continuous random variables. It is defined as the $L^2$ error between the CDF $F$ of the predictive distribution and the 
empirical distribution function (EDF) based on the observed outcome $x$
\begin{align} 
\text{CRPS}(F, x) = -\int_{-\infty}^{\infty} \left(F(y) - 1\{x \leq y\} \right)^2 dy. \label{CRPS}
\end{align}
As opposed to the logarithmic scoring rule, the CRPS has the advantage of being defined in terms of the CDF instead of the density, making it much more amenable to estimation via samples. It also depends on the predictive 
distribution over all $y$, not just the predictive distribution at the value $x$. This allows the CRPS to take into account the quality of nearby predictions when assessing the forecast, which may be a desirable property. It can be 
shown that the above integral is equivalent to 
\begin{align}
\text{CRPS}(F, x) = \frac{1}{2} \E_F \abs{X - X^\prime} - \E_F \abs{X - x}, \label{CRPS2}
\end{align}
where $X, X^\prime \overset{ind}{\sim} F$. The second term is simply the mean absolute error, meaning that large errors reduce the score (note the negative sign). The first term is a measure of spread in the forecast; this term 
will be large when the spread in the forecast is small, rewarding confident predictions. Therefore, the CRPS can be seen as balancing prediction accuracy and confidence. 

Closed-form expressions for CRPS: http://cran.nexr.com/web/packages/scoringRules/vignettes/crpsformulas.html

\subsection{Energy Score}
The \textit{energy score} is the multivariate generalization of the CRPS. Suppose now that we are considering distributions over $\R^D$. We directly generalize \ref{CRPS2} by defining the energy score
\begin{align}
\text{ES}(P, \bx) &= \frac{1}{2} \E_{\rVec, \rVec^\prime \sim P} \norm{\rVec - \rVec^\prime}_2 - \E_{\rVec \sim P}  \norm{\rVec - \bx}_2. 
\end{align}
This scoring rule is strictly proper and, moreover, the generalization 
\begin{align}
\text{ES}_{\beta}(P, \bx) &= \frac{1}{2} \E_{\rVec, \rVec^\prime \sim P} \norm{\rVec - \rVec^\prime}^\beta_2 - \E_{\rVec \sim P}  \norm{\rVec - \bx}^\beta_2
\end{align}
is strictly proper when $\beta \in (0, 2)$. We focus on the common $\beta = 1$ case. The expected energy score with respect to baseline distribution $Q$ is thus 
\begin{align}
\text{ES}_{\beta}(P, Q) := \E_{\bx \sim Q} \left[\text{ES}(P, \bx)\right] = \frac{1}{2} \E_{\rVec, \rVec^\prime \sim P} \norm{\rVec - \rVec^\prime}_2 - \E_{\rVec \sim P, \bx \sim Q}  \norm{\rVec - \bx}_2. \label{expected_ES}
\end{align}
It turns out that the expected energy score has close connections with the \textit{energy distance} defined, for example, in \cite{Mak}. To recover the energy distance, we recall the 
divergence function associated with the score \text{ES}
\begin{align}
d_{\text{ES}}(P, Q) &= \text{ES}(Q, Q) - \text{ES}(P, Q),
\end{align}
where we note that $Q$ is interpreted as the true or baseline distribution here. We denote samples from $P$ and $Q$ as $\rVec \sim P$ and $\rVecTwo \sim Q$, respectively. 
However, it turns out that in the special case of the energy score, this divergence will actually be symmetric. 
Indeed, plugging in \ref{expected_ES} we find 
\begin{align}
d_{\text{ES}}(P, Q) &=  \E_{\rVec \sim P, \rVecTwo \sim Q}  \norm{\rVec - \rVecTwo}_2 - \frac{1}{2} \E_{\rVec, \rVec^\prime \sim P} \norm{\rVec - \rVec^\prime}_2 - \frac{1}{2} \E_{\rVecTwo, \rVecTwo^\prime \sim Q} \norm{\rVecTwo - \rVecTwo^\prime}_2,
\end{align}
which is precisely two times the energy distance, as defined in \ref{Mak}. It is shown in \ref{Mak} that the energy distance is a valid metric on probability distributions. Thus, the divergence function of the energy score actually induces a valid probability 
metric, which is not the case in general. 

\section{TODOs:}
\begin{enumerate}
\item Show $\ell_1$ distance is invariant to changes in scale (show using the fact that it is equivalent to TV distance). 
\item Show that $\ell_2$ distance is not invariant to changes in scale. 
\item Larry Wasserstein's intermediate statistics notes https://www.stat.cmu.edu/~larry/=stat705/Lecture27.pdf have a lot of good content, including MMD, integral probability metrics, f-divergences. 
\item Good posts of TV and Hellinger distance: https://djalil.chafai.net/blog/2010/10/14/back-to-basics-total-variation-distance/
\end{enumerate}

% Bibliography
\begin{thebibliography}{20}
\bibitem{Gretton1} Arthur Gretton, Karsten M. Borgwardt, Malte Rasch, Bernhard Schölkopf, and Alexander J. Smola. 2006. A kernel method for the two-sample-problem. In Proceedings of the 19th International Conference on Neural Information Processing Systems (NIPS'06). MIT Press, Cambridge, MA, USA, 513–520.
\bibitem{Gretton2} A. Gretton, K. Borgwardt, M. Rasch, B. Scholkopf, and A. Smola. A kernel method for the two sample
problem. Technical Report 157, MPI for Biological Cybernetics, 2007.
\bibitem{Muandet} Krikamol Muandet; Kenji Fukumizu; Bharath Sriperumbudur; Bernhard Schölkopf, Kernel Mean Embedding of Distributions: A Review and Beyond , now, 2017.
\bibitem{Hofmann} Thomas Hofmann, Bernhard Schölkopf, Alexander J. Smola. "Kernel methods in machine learning." The Annals of Statistics, 36(3) 1171-1220 June 2008.
\bibitem{Mak} Mak, Simon \& Joseph, V. Roshan. (2017). Support points. Annals of Statistics. 46. 10.1214/17-AOS1629. 
\end{thebibliography}

\end{document}



